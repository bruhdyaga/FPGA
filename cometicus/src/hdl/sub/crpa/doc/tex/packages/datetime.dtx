%\iffalse
% datetime.dtx generated using makedtx version 0.94b (c) Nicola Talbot
% Command line args:
%   -src "(.+)\.(sty)=>\1.\2"
%   -src "(.+)\.(def)=>\1.\2"
%   -doc "manual.tex"
%   -author "Nicola Talbot"
%   -dir "source"
%   datetime
% Created on 2007/8/20 15:37
%\fi
%\iffalse
%<*package>
%% \CharacterTable
%%  {Upper-case    \A\B\C\D\E\F\G\H\I\J\K\L\M\N\O\P\Q\R\S\T\U\V\W\X\Y\Z
%%   Lower-case    \a\b\c\d\e\f\g\h\i\j\k\l\m\n\o\p\q\r\s\t\u\v\w\x\y\z
%%   Digits        \0\1\2\3\4\5\6\7\8\9
%%   Exclamation   \!     Double quote  \"     Hash (number) \#
%%   Dollar        \$     Percent       \%     Ampersand     \&
%%   Acute accent  \'     Left paren    \(     Right paren   \)
%%   Asterisk      \*     Plus          \+     Comma         \,
%%   Minus         \-     Point         \.     Solidus       \/
%%   Colon         \:     Semicolon     \;     Less than     \<
%%   Equals        \=     Greater than  \>     Question mark \?
%%   Commercial at \@     Left bracket  \[     Backslash     \\
%%   Right bracket \]     Circumflex    \^     Underscore    \_
%%   Grave accent  \`     Left brace    \{     Vertical bar  \|
%%   Right brace   \}     Tilde         \~}
%</package>
%\fi
% \iffalse
% Doc-Source file to use with LaTeX2e
% Copyright (C) 2007 Nicola Talbot, all rights reserved.
% \fi
% \iffalse
%<*driver>
\documentclass{ltxdoc}

\usepackage[colorlinks,
            bookmarks,
            bookmarksopen,
            pdfauthor={N.L.C. Talbot},
            pdftitle={datetime.sty: A Date and Time Package},
            pdfkeywords={date,time,LaTeX}]{hyperref}

\CheckSum{4044}

\newcommand{\sty}[1]{\textsf{#1}}
\begin{document}
\DocInput{datetime.dtx}
\end{document}
%</driver>
%\fi
%\RecordChanges
%\OnlyDescription
%
%\title{datetime.sty v2.55: Formatting Current Date and 
%Time}
% \author{Nicola L. C. Talbot\\[10pt]
% School of Computing Sciences\\
% University of East Anglia\\
% Norwich.  NR4 7TJ.\\
% United Kingdom.\\
% \url{http://theoval.cmp.uea.ac.uk/~nlct/}}
% \date{20 Aug 2007}
% \maketitle
% \tableofcontents
% \section{Introduction}
%\changes{1.0}{2000/08/08}{First release}
%\changes{1.01}{2000/09/18}{Documentation added}
%The \sty{datetime} package is a \LaTeXe\ package that 
%provides various different formats for \cs{today},
%and provides commands for displaying the current time.  
%If you only want the 
%time commands but not the date changing commands, you can pass 
%the option \texttt{nodate} to the package.
%\changes{2.3}{2004/05/01}{nodate package option added}
%
%\changes{2.41}{2004/10/22}{split package into two files: 
%datetime.sty and fmtcount.sty}
%Since version 2.4, the \sty{datetime} package has been 
%separated into two packages: \sty{datetime} and 
%\sty{fmtcount}.  When I originally created this package, 
%I defined the commands, \cs{ordinal} etc which could be used 
%in the definition of \cs{today}.  Since then, I have extended 
%the number of commands available that can be used to display the 
%value of a \LaTeX\ counter, however it seems more appropriate to 
%define all these counter-related commands in a separate package. 
%The \sty{fmtcount} package is now distributed separately 
%from the \sty{datetime} package, and will also need to 
%be installed.
%
%\changes{2.42}{2004/11/01}{made package compatible with babel}
%As from version 2.42, the \sty{datetime} package is now 
%compatible with \sty{babel}, however you must load the 
%\sty{datetime} package \emph{after} the \sty{babel} package.  
%For example:
%\begin{verbatim}
%\usepackage[francais]{babel}
%\usepackage{datetime}
%\end{verbatim}
%
% \section{Date Declarations}
% There are various declarations that change the effect of 
%\cs{today}.  The change can be localised by placing the 
%declaration within a group.
%
% \vspace{10pt}\noindent  \meta{Day} \meta{Month} \meta{Year} formats:\\
% \noindent\DescribeMacro{\longdate}
% The declaration \verb"\longdate" will redefine 
%\verb"\today" to produce the current date displayed in the form 
%Wednesday 8\textsuperscript{th} March, 2000
% if the package option \texttt{dayofweek} is used, or 
%8\textsuperscript{th} March, 2000 if the package option
% \texttt{nodayofweek} is used.\\
% \DescribeMacro{\shortdate}
% The declaration \verb"\shortdate" will redefine 
%\verb"\today" to produce the current date displayed in the form 
%Wed 8\textsuperscript{th} Mar, 2000 if the package option 
%\texttt{dayofweek} is used, or 8\textsuperscript{th} Mar, 2000 
%if the package option \texttt{nodayofweek} is used.\\
% \DescribeMacro{\ddmmyyyydate}
% The declaration \verb"\ddmmyyyydate" will redefine 
%\verb"\today" to produce
% the current date displayed in the form 08/03/2000\\
% \DescribeMacro{\dmyyyydate}
% The declaration \verb"\dmyyyydate" will redefine 
%\verb"\today" to produce
% the current date displayed in the form 8/3/2000\\
% \DescribeMacro{\ddmmyydate}
% The declaration \verb"\ddmmyydate" will redefine 
%\verb"\today" to produce
% the current date displayed in the form 08/03/00\\
% \DescribeMacro{\dmyydate}
% The declaration \verb"\dmyydate" will redefine 
%\verb"\today" to produce
% the current date displayed in the form 8/3/00\\
% \DescribeMacro{\textdate}
% The declaration \verb"\textdate" will redefine 
%\verb"\today" to produce the current date displayed in the form: 
%Wednesday the Eighth of March, Two Thousand if the package option 
%\texttt{dayofweek} is used, or Eighth of March, Two Thousand if 
%the package option \texttt{nodayofweek} is used. Note that 
%\verb"\textdate" is defined for use with English, it won't
%look right if it is used when another language has been
%selected\footnote{in fact, you may get an error from the
%\sty{fmtcount} package if you are using a language that it
%doesn't support.}.  If you want to define a similar command for
%another language, you will first need to check that the
%\sty{fmtcount} package supports that language.
%
% \vspace{10pt}\noindent  \meta{Month} \meta{Day} \meta{Year} formats:\\
% \DescribeMacro{\usdate}
% The declaration \verb"\usdate" will redefine 
%\verb"\today" to produce the current date displayed in the form 
%March 8, 2000. (As \TeX\ and \LaTeX\ do by default.)\\
% \DescribeMacro{\mmddyyyydate}
% The declaration \verb"\mmddyyyydate" will redefine 
%\verb"\today" to produce the current date displayed in the form 
%03/08/2000\\
% \DescribeMacro{\mdyyyydate}
% The declaration \verb"\mdyyyydate" will redefine 
%\verb"\today" to produce the current date displayed in the form 
%3/8/2000\\
% \DescribeMacro{\mmddyydate}
% The declaration \verb"\mmddyydate" will redefine 
%\verb"\today" to produce the current date displayed in the form 
%03/08/00\\
% \DescribeMacro{\mdyydate}
% The declaration \verb"\mdyydate" will redefine 
%\verb"\today" to produce the current date displayed in the form 
%3/8/00
%
%\vspace{10pt}\noindent In addition, the declarations \verb"\date"\meta{lang} are 
%available for all languages defined either by
%calling \sty{babel} prior to \sty{datetime} or by 
%passing the language name as an option to \sty{datetime}.
%See~\autoref{sec:newdate} if you want to define your own customised date format.
%
%As from version 2.43, the numerical date formats (such as 
%\verb"\ddmmyyyydate") use the command
%\verb"\dateseparator" to separate the numbers.  So, for example, 
%if you want to hyphens instead of slashes, you can do:
%\begin{verbatim}
%\renewcommand{\dateseparator}{-}
%\end{verbatim}
%
% \section{Time Commands}
%\DescribeMacro{\currenttime}
% The current time is displayed using the command 
%\verb"\currenttime".
%\DescribeMacro{\settimeformat}
% The format can be changed using the declaration
%\cs{settimeformat}\marg{style}, where \meta{style} is the name of the
%format\footnote{Note that the commands \cs{xxivtime}, \cs{ampmtime}
%and \cs{oclock} are still available, \cs{settimeformat} redefines
%\cs{currenttime} to the command given by placing a backslash in front
%of \meta{style}.  So \cs{settimeformat\{xxivtime\}} sets
%\cs{currenttime} to \cs{xxivtime} and so on.}.  Available formats
%are:
%\begin{description}
%\item[xxivtime] Twenty-four hour time in the form 22:28 (Default)
%\item[ampmtime] Twelve hour time in the form 10:28pm
%\item[oclock] Displays the current time as a string, e.g.\ 
%Twenty-Eight minutes past Ten in the afternoon.
%\end{description}
%
% \DescribeMacro{\newtimeformat}
% New time formats can be defined using the command:\\[5pt]
%\cs{newtimeformat}\marg{name}\marg{format}\\[5pt]
% where \meta{name} is the name of the new format (used in 
%\cs{settimeformat}), and \meta{format} is how to format the 
%time.  Within \meta{format} you can use the counters 
%\texttt{HOUR} (number of hours after midnight), \texttt{MINUTE} 
%(number of minutes past the hour), \texttt{HOURXII} (number of 
%hours after midnight/midday), \texttt{TOHOUR} (the next hour) and
%\texttt{TOMINUTE} (number of minutes to the next hour), and the
%corresponding commands: \cs{THEHOUR}, \cs{THEMINUTE},
%\cs{THEHOURXII}, \cs{THETOHOUR} and \cs{THETOMINUTE}.
%
%For example, to define a new time format that uses a dot instead of a
%colon:
%\begin{verbatim}
%\newtimeformat{dottime}{\twodigit{\THEHOUR}.\twodigit{\THEMINUTE}}
%\end{verbatim}
%You then need to switch to this new format before you can use it:
%\begin{verbatim}
%\settimeformat{dottime}
%\currenttime
%\end{verbatim}
%
%As from version 2.43, if you only want to change the separator, 
%you can simply redefine \cs{timeseparator}
%instead of defining a new time format.  For example:
%\begin{verbatim}
%\renewcommand{\timeseparator}{.}
%\end{verbatim}
%The \texttt{xxivtime} format will now work like the \texttt{dottime} 
%format defined above.
%
% \section{Formating Dates}
% \DescribeMacro{\pdfdate}
% The command \verb"\pdfdate"\footnote{thanks to Ulrich Dirr for 
%asking about this} prints the date in the format required for
% PDF files, e.g.\ if the date is 1 May 2004 and time is
% 22:02, \verb"\pdfdate" will print 20040501220200.  The reason 
%this date format is separate from all the others is because the 
%other form doesn't get properly expanded by PDF\TeX. (This 
%command is defined regardless of whether the package option 
%\texttt{nodate} is called.)
%Example:
%\begin{verbatim}
%\pdfinfo{
%   /Author (Me)
%   /Title (A Sample Document)
%   /CreationDate (D:20040501215500)
%   /ModDate (D:\pdfdate)
%}
%\end{verbatim}
% 
% There are two commands that print the name of the current
% month:
% \DescribeMacro{\monthname}
% \verb"\monthname" prints the current month name in full, 
%e.g.\ August, and 
% \DescribeMacro{\shortmonthname}
% \verb"\shortmonthname" prints the abbreviated month name, 
%e.g.\ Aug.  Both \verb"\monthname" and 
%\verb"\shortmonthname" take an optional argument (a number from 
%1 to 12) if the name of a specific month is required.  For 
%example, \verb"\monthname[6]" will produced the output: June.
%
% The day of the week is computed using the algorithm documented at
% \url{http://userpages.wittenburg.edu/bshelburne/Comp150/DayOfWeek.htm}.
% This algorithm works for any date between 1\textsuperscript{st} 
%Jan, 1901 and 31\textsuperscript{st} Dec, 2099.
% The following macros display the day of week for a given date:
%
% \noindent\DescribeMacro{\dayofweekname}
%\cs{dayofweekname}\marg{day}\marg{month}\marg{year} prints the
% day of week for the specified date.  For example, 
%\verb"\dayofweekname{31}{10}{2002}"
% will produce the output: Thursday.\\
% \DescribeMacro{\shortdayofweekname}
%\cs{shortdayofweekname}\meta{day}\marg{month}\marg{year} prints the 
%abbreviated name for the
% day of week for the specified date.  For example\\
% \verb"\shortdayofweekname{31}{10}{2002}"\\
% will produce the output: Thu.
%
% \noindent\DescribeMacro{\ifshowdow}
% The \TeX\ conditional \verb"\ifshowdow" can be used to determine
% whether or not the option \texttt{dayofweek} has been passed to 
%the package.
% For example:
%\begin{verbatim}
%\ifshowdow\dayofweekname{31}{10}{2002} \fi
%\end{verbatim}
% will only display the day of week if the \texttt{dayofweek} 
%option was passed to \sty{datetime}.
% Alternatively, you can use David~Carlisle's \sty{ifthen} 
%package:
%\begin{verbatim}
%\ifthenelse{\boolean{showdow}}{\dayofweekname{31}{10}{2002} }{}
%\end{verbatim}
%
%\noindent\DescribeMacro{\ordinaldate}
%The command \cs{ordinaldate}\marg{number}
%displays \meta{number} as a date-type ordinal.  If the
%current language is English, this will simply pass
%the argument to \cs{ordinalnum} (defined in the 
%\sty{fmtcount} package), 
%if the current language is Breton, Welsh or French, a superscript 
%will only be added if \meta{number} is 1, otherwise only 
%\meta{number} will be displayed.
%
% \DescribeMacro{\formatdate}
%The macro
%\cs{formatdate}\marg{day}\marg{month}\marg{year}\footnote{Note the
%name change since version 1.1.  The command name was changed from
%\cs{thedate} to \cs{formatdate} to avoid a name clash when using the
%\sty{seminar} class file.} formats the specified date according to
%the current format of \cs{today}\footnote{To be more precise,
%\cs{today} is defined to be
%\cs{formatdate\{\cs{day}\}\{\cs{month}\}\{\cs{year}\}} where
%\cs{longdate} etc change the definition of \cs{formatdate}}.
%(Arguments must all be integers.) For example, in combination with
%\verb"\longdate", the command
%\begin{verbatim}
%\formatdate{27}{9}{2004}
%\end{verbatim}
%will produce the output: Monday 27\textsuperscript{th} September,
%2004.
%
% \DescribeMacro{\twodigit}
% You can ensure that a number is displayed with at least two 
%digits by using the command 
%\cs{twodigit}\marg{num}.
% This is of use if you want to define your own date or time 
%formats.
%
% \section{Defining New Date Formats}\label{sec:newdate}
%
% \DescribeMacro{\newdateformat}
% New date formats can be defined using the command:\\[5pt]
% \cs{newdateformat}\marg{name}\marg{format}\\[5pt]
% where \meta{name} is the name of the new format, and
% \meta{format} is how to format the date.  Within the
% argument \meta{format} you can use the commands \cs{THEDAY}, 
%\cs{THEMONTH}
% and \cs{THEYEAR} to represent the relevant day, month and 
%year, or you can use the counters 
% \texttt{DAY}, \texttt{MONTH} and \texttt{YEAR} if you want to 
%use \cs{ordinal} etc. Once you have defined the
% new date format, you can then switch to it using the declaration 
%\verb'\'\meta{name}
% (i.e.\ the name you specified preceded by a backslash), and 
% subsequent calls to \cs{today} and \cs{formatdate} will 
%use your new format.
%
% For example, suppose you want to define a new date format called,
% say, \texttt{mydate}, that will typeset the date in the form: 
%8-3-2002,
% then you can do:
%\begin{verbatim}
%\newdateformat{mydate}{\THEDAY-\THEMONTH-\THEYEAR}
%\end{verbatim}
%\cs{newdateformat} will then define the declaration 
%\cs{mydate} which can be used to
% switch to your new format. In the following example, 
% two new date formats are defined, and they are then
% selected to produce two different formats for the current date:
%\begin{verbatim}
%\newdateformat{dashdate}{%
%\twodigit{\THEDAY}-\twodigit{\THEMONTH}-\THEYEAR}
%
%\newdateformat{usvardate}{%
%\monthname[\THEMONTH] \ordinal{DAY}, \THEYEAR}
%
%Dash: \dashdate\today.
%US: \usvardate\today.
%\end{verbatim}
%If the current date is, say, 8th March, 2002, the above code will 
% produce the following:
%Dash: 08-03-2002.
%US: March 8\textsuperscript{th}, 2002.
%
%Note that \cs{THEDAY} etc and \texttt{DAY} etc have no real 
%meaning outside \cs{newdateformat} (this is why they 
%are in uppercase). Incidentally, the \texttt{dashdate} format
%is not really necessary, as you can achieve this format
%using:
%\begin{verbatim}
%\renewcommand{\dateseparator}{-}
%\ddmmyyyydate
%\end{verbatim}
%
%Another note: in the above code, \cs{ordinal} was
%used to illustrate the use of the \texttt{DAY} counter.  It
%is better to use \cs{ordinaldate} instead:
%\begin{verbatim}
%\newdateformat{usvardate}{%
%\monthname[\THEMONTH] \ordinaldate{\THEDAY}, \THEYEAR}
%\end{verbatim}
%
%\section{Saving Dates}
%
%It is possible to save a date for later use using the command:
%\footnote{Thanks to Denis Bitouz\'e for asking about this}\par
%\DescribeMacro{\newdate}
%\cs{newdate}\marg{name}\marg{day}\marg{month}\marg{year}
%
%This date can later be displayed using the same format as that
%used by \cs{formatdate} using the command:\par
%\DescribeMacro{\displaydate}
%\cs{displaydate}\marg{name}
%
%Individual elements of the date can be extracted using the
%commands:\par
%\DescribeMacro{\getdateday}
%\cs{getdateday}\marg{name}\par
%\DescribeMacro{\getdatemonth}
%\cs{getdatemonth}\marg{name}\par
%\DescribeMacro{\getdateyear}
%\cs{getdateyear}\marg{name}
%
%\section{Predefined Names}
%
%The following commands are defined by the \sty{datetime} 
%package:
%
%\begin{tabular}{ll}
%\bfseries Command Name & \bfseries Default Value\\
%\cs{dateseparator} & \verb'/'\\
%\cs{timeseparator} & \verb':'\\
%\cs{amname} & \texttt{am}\\
%\cs{pmname} & \texttt{pm}\\
%\cs{amorpmname} & \cs{amname} if morning, otherwise \cs{pmname}\\
%\cs{amstring} & \texttt{in the morning}\\
%\cs{pmstring} & \texttt{in the afternoon}\\
%\cs{amorpmstring} & \cs{amstring} if morning, otherwise 
%\cs{pmstring}\\
%\cs{halfpast} & \texttt{Half past}\\
%\cs{quarterpast} & \texttt{Quarter past}\\
%\cs{quarterto} & \texttt{Quarter to}\\
%\cs{noon} & \texttt{Noon}\\
%\cs{midnight} & \texttt{Midnight}\\
%\cs{oclockstring} & \texttt{O'Clock}
%\end{tabular}
%
% \section{Package Options}
%
% The following options may be passed to this package:\\[10pt]
% \begin{tabular}{@{\ttfamily}ll}
% long     & make \cs{today} produce long date\\
% short    & make \cs{today} produce short date\\
% ddmmyyyy & make \cs{today} produce DD/MM/YYYY date\\
% dmyyyy   & make \cs{today} produce D/M/YYYY date\\
% ddmmyy   & make \cs{today} produce DD/MM/YY date\\
% dmyy     & make \cs{today} produce D/M/YY date\\
% text     & make \cs{today} produce text date\\
% us       & make \cs{today} produce US style date\\
% mmddyyyy & make \cs{today} produce MM/DD/YYYY date\\
% mdyyyy   & make \cs{today} produce M/D/YYYY date\\
% mmddyy   & make \cs{today} produce MM/DD/YY date\\
% mdyy     & make \cs{today} produce M/D/YY date\\
% raise    & make ordinal st,nd,rd,th appear as superscript\\
% level    & make ordinal st,nd,rd,th appear level with rest of 
%text\\
% dayofweek & make the day of week appear for \cs{longdate}, 
%\cs{shortdate} \\
%           & or \cs{textdate}\\
% nodayofweek & don't display the day of week.\\
% 24hr     & make \cs{currenttime} produce \texttt{xxivtime} 
%format\\
% 12hr     & make \cs{currenttime} produce \texttt{ampmtime} 
%format\\
% oclock   & make \cs{currenttime} produce \texttt{oclock} 
%format\\
% nodate   & Don't redefine \cs{today} or define the month or 
%day of week commands\\
%          & (useful if you only want the time commands or 
%\verb"\pdfdate")
% \end{tabular}
%
%The default options are: \texttt{long}, \texttt{raise}, 
%\texttt{dayofweek} and \texttt{24hr}.
%
%\section{Multilingual Support}
%
%If you want to use the \sty{babel} package, you must load
%it \emph{before} you load the \sty{datetime} package. This
%is because the \sty{babel} \cs{date}\meta{lang} commands
%redefine \cs{today}, whereas the \sty{datetime} package
%redefines \cs{today} to use \cs{formatdate}, and the
%date formatting commands (such as \cs{longdate}) redefine
%\cs{formatdate} rather than \cs{today}. This ensures 
%consistent formatting of the dates whether you use \cs{today} or
%\cs{formatdate}.  Therefore, the \sty{datetime} package
%has to redefine all the \cs{date}\meta{lang} commands accordingly.
%Thus the multilingual date support is mostly limited to that provided
%by \sty{babel}. Additional support, such as the day of
%week names and abbreviations, are only supplied for those
%languages that I know, or that other people have been able to supply
%for me.
%
%
%The commands \cs{monthname} and \cs{shortmonthname},
%will produce the month name in the current language.
%If you want the month name in a specific language, you
%can use the command \cs{monthname}\meta{lang}.
%For example, \verb"\monthnamefrench[6]" will produce the output: 
%juin. Note that \cs{textdate} is formatted for English dates,
%and won't look right if used with another language setting. If you
%want a textual date, the \sty{fmtcount} package (which is 
%loaded by \sty{datetime}) defines some commands which display
%a number or ordinal as a word, but it only has very limited 
%multilingual support. See the \sty{fmtcount} documentation 
%for further details.
%
%There is currently only \emph{limited} multilingual support for 
%\cs{dayofweekname} and \cs{shortdayofweekname} (just
%English, French, Portuguese, Spanish and German\footnote{thanks
%to Uwe Bieling for supplying the German names}). You can add support 
%for other languages by defining the commands
%\cs{dayofweeknameid}\meta{lang} and 
%\cs{shortdayofweeknameid}\meta{lang}. Note that these
%commands only take \emph{one} argument which should be
%a number from 1 to 7 indicating the day of the week.
%
%You can use the following as templates.  Replace
%\texttt{english} with the name of your language (as given
%by \cs{languagename}) and replace \texttt{Sunday}
%etc as appropriate:
%\begin{verbatim}
%\providecommand*{\dayofweeknameidenglish}[1]{%
%\ifcase#1\relax
%\or Sunday%
%\or Monday%
%\or Tuesday%
%\or Wednesday%
%\or Thursday%
%\or Friday%
%\or Saturday%
%\fi}
%                                                          
%\providecommand*{\shortdayofweeknameidenglish}[1]{%
%\ifcase#1\relax
%\or Sun%
%\or Mon%
%\or Tue%
%\or Wed%
%\or Thu%
%\or Fri%
%\or Sat%
%\fi}
%\end{verbatim}
%If you want them added to future versions of 
%\sty{datetime}, please email me the code.
%
%\section{Configuration File}
%
%As from Version 2.4, the \sty{datetime} package will read in settings
%from the configuration file \texttt{datetime.cfg}, if it exists,
%which will override the default package options.  For example,
%suppose you prefer a short date without the day of week by default,
%you will need to create a file called \texttt{datetime.cfg} that
%contains the line:
%\begin{verbatim}
%\shortdate\showdowfalse
%\end{verbatim}
%The file \texttt{datetime.cfg} should then go somewhere on the 
%\TeX\ path.  Now all you need to do is:
%\begin{verbatim}
%\usepackage{datetime}
%\end{verbatim}
%without having to specify the \texttt{short} and 
%\texttt{nodayofweek} options.
%
%You can also use this file to define and set your own date 
%styles.  For example, you could create
%a configuration file that has the following lines:
%\begin{verbatim}
%\newdateformat{dashdate}{\twodigit{\THEDAY}-\twodigit{\THEMONTH}-\THEYEAR}
%\dashdate
%\end{verbatim}
%Whenever you use the \sty{datetime} package, it will now 
%use this format by default.
%
%\section{LaTeX2HTML styles}
%
%\changes{2.43}{2005/02/25}{Added LaTeX2HTML support}%
%\changes{2.44}{2005/03/03}{Fixed minor bugs in Perl scripts}
%Version 2.43 and above of the \sty{datetime} bundle 
%supplies the LaTeX2HTML style file \texttt{datetime.perl}.  
%This file should be placed in a 
%directory searched by LaTeX2HTML.  The following limitations 
%apply to the LaTeX2HTML styles:
%
%\begin{itemize}
%\item The configuration file \texttt{datetime.cfg}
%is currently ignored. (This is because
%I can't work out the correct code to do this.  If you
%know how to do this, please let me know.)  You can however
%do:
%\begin{verbatim}
%\usepackage{datetime}
%\html{% datetime.ins generated using makedtx version 0.94b 2007/8/20 15:37
\input docstrip

\preamble

 datetime.dtx
 Copyright 2007 Nicola Talbot

 This work may be distributed and/or modified under the
 conditions of the LaTeX Project Public License, either version 1.3
 of this license of (at your option) any later version.
 The latest version of this license is in
   http://www.latex-project.org/lppl.txt
 and version 1.3 or later is part of all distributions of LaTeX
 version 2005/12/01 or later.

 This work has the LPPL maintenance status `maintained'.

 The Current Maintainer of this work is Nicola Talbot.

 This work consists of the files datetime.dtx and datetime.ins and the derived files datetime.sty, dt-austrian.def, dt-bahasa.def, dt-basque.def, dt-breton.def, dt-british.def, dt-bulgarian.def, dt-catalan.def, dt-croatian.def, dt-czech.def, dt-danish.def, dt-dutch.def, dt-esperanto.def, dt-estonian.def, dt-finnish.def, dt-french.def, dt-galician.def, dt-german.def, dt-greek.def, dt-hebrew.def, dt-icelandic.def, dt-irish.def, dt-italian.def, dt-latin.def, dt-lsorbian.def, dt-magyar.def, dt-naustrian.def, dt-ngerman.def, dt-norsk.def, dt-polish.def, dt-portuges.def, dt-romanian.def, dt-russian.def, dt-samin.def, dt-scottish.def, dt-serbian.def, dt-slovak.def, dt-slovene.def, dt-spanish.def, dt-swedish.def, dt-turkish.def, dt-UKenglish.def, dt-ukraineb.def, dt-USenglish.def, dt-usorbian.def, dt-welsh.def.

\endpreamble

\askforoverwritefalse

\generate{\file{datetime.sty}{\usepreamble\defaultpreamble
\usepostamble\defaultpostamble\from{datetime.dtx}{datetime.sty,package}}
\file{dt-austrian.def}{\usepreamble\defaultpreamble
\usepostamble\defaultpostamble\from{datetime.dtx}{dt-austrian.def,package}}
\file{dt-bahasa.def}{\usepreamble\defaultpreamble
\usepostamble\defaultpostamble\from{datetime.dtx}{dt-bahasa.def,package}}
\file{dt-basque.def}{\usepreamble\defaultpreamble
\usepostamble\defaultpostamble\from{datetime.dtx}{dt-basque.def,package}}
\file{dt-breton.def}{\usepreamble\defaultpreamble
\usepostamble\defaultpostamble\from{datetime.dtx}{dt-breton.def,package}}
\file{dt-british.def}{\usepreamble\defaultpreamble
\usepostamble\defaultpostamble\from{datetime.dtx}{dt-british.def,package}}
\file{dt-bulgarian.def}{\usepreamble\defaultpreamble
\usepostamble\defaultpostamble\from{datetime.dtx}{dt-bulgarian.def,package}}
\file{dt-catalan.def}{\usepreamble\defaultpreamble
\usepostamble\defaultpostamble\from{datetime.dtx}{dt-catalan.def,package}}
\file{dt-croatian.def}{\usepreamble\defaultpreamble
\usepostamble\defaultpostamble\from{datetime.dtx}{dt-croatian.def,package}}
\file{dt-czech.def}{\usepreamble\defaultpreamble
\usepostamble\defaultpostamble\from{datetime.dtx}{dt-czech.def,package}}
\file{dt-danish.def}{\usepreamble\defaultpreamble
\usepostamble\defaultpostamble\from{datetime.dtx}{dt-danish.def,package}}
\file{dt-dutch.def}{\usepreamble\defaultpreamble
\usepostamble\defaultpostamble\from{datetime.dtx}{dt-dutch.def,package}}
\file{dt-esperanto.def}{\usepreamble\defaultpreamble
\usepostamble\defaultpostamble\from{datetime.dtx}{dt-esperanto.def,package}}
\file{dt-estonian.def}{\usepreamble\defaultpreamble
\usepostamble\defaultpostamble\from{datetime.dtx}{dt-estonian.def,package}}
\file{dt-finnish.def}{\usepreamble\defaultpreamble
\usepostamble\defaultpostamble\from{datetime.dtx}{dt-finnish.def,package}}
\file{dt-french.def}{\usepreamble\defaultpreamble
\usepostamble\defaultpostamble\from{datetime.dtx}{dt-french.def,package}}
\file{dt-galician.def}{\usepreamble\defaultpreamble
\usepostamble\defaultpostamble\from{datetime.dtx}{dt-galician.def,package}}
\file{dt-german.def}{\usepreamble\defaultpreamble
\usepostamble\defaultpostamble\from{datetime.dtx}{dt-german.def,package}}
\file{dt-greek.def}{\usepreamble\defaultpreamble
\usepostamble\defaultpostamble\from{datetime.dtx}{dt-greek.def,package}}
\file{dt-hebrew.def}{\usepreamble\defaultpreamble
\usepostamble\defaultpostamble\from{datetime.dtx}{dt-hebrew.def,package}}
\file{dt-icelandic.def}{\usepreamble\defaultpreamble
\usepostamble\defaultpostamble\from{datetime.dtx}{dt-icelandic.def,package}}
\file{dt-irish.def}{\usepreamble\defaultpreamble
\usepostamble\defaultpostamble\from{datetime.dtx}{dt-irish.def,package}}
\file{dt-italian.def}{\usepreamble\defaultpreamble
\usepostamble\defaultpostamble\from{datetime.dtx}{dt-italian.def,package}}
\file{dt-latin.def}{\usepreamble\defaultpreamble
\usepostamble\defaultpostamble\from{datetime.dtx}{dt-latin.def,package}}
\file{dt-lsorbian.def}{\usepreamble\defaultpreamble
\usepostamble\defaultpostamble\from{datetime.dtx}{dt-lsorbian.def,package}}
\file{dt-magyar.def}{\usepreamble\defaultpreamble
\usepostamble\defaultpostamble\from{datetime.dtx}{dt-magyar.def,package}}
\file{dt-naustrian.def}{\usepreamble\defaultpreamble
\usepostamble\defaultpostamble\from{datetime.dtx}{dt-naustrian.def,package}}
\file{dt-ngerman.def}{\usepreamble\defaultpreamble
\usepostamble\defaultpostamble\from{datetime.dtx}{dt-ngerman.def,package}}
\file{dt-norsk.def}{\usepreamble\defaultpreamble
\usepostamble\defaultpostamble\from{datetime.dtx}{dt-norsk.def,package}}
\file{dt-polish.def}{\usepreamble\defaultpreamble
\usepostamble\defaultpostamble\from{datetime.dtx}{dt-polish.def,package}}
\file{dt-portuges.def}{\usepreamble\defaultpreamble
\usepostamble\defaultpostamble\from{datetime.dtx}{dt-portuges.def,package}}
\file{dt-romanian.def}{\usepreamble\defaultpreamble
\usepostamble\defaultpostamble\from{datetime.dtx}{dt-romanian.def,package}}
\file{dt-russian.def}{\usepreamble\defaultpreamble
\usepostamble\defaultpostamble\from{datetime.dtx}{dt-russian.def,package}}
\file{dt-samin.def}{\usepreamble\defaultpreamble
\usepostamble\defaultpostamble\from{datetime.dtx}{dt-samin.def,package}}
\file{dt-scottish.def}{\usepreamble\defaultpreamble
\usepostamble\defaultpostamble\from{datetime.dtx}{dt-scottish.def,package}}
\file{dt-serbian.def}{\usepreamble\defaultpreamble
\usepostamble\defaultpostamble\from{datetime.dtx}{dt-serbian.def,package}}
\file{dt-slovak.def}{\usepreamble\defaultpreamble
\usepostamble\defaultpostamble\from{datetime.dtx}{dt-slovak.def,package}}
\file{dt-slovene.def}{\usepreamble\defaultpreamble
\usepostamble\defaultpostamble\from{datetime.dtx}{dt-slovene.def,package}}
\file{dt-spanish.def}{\usepreamble\defaultpreamble
\usepostamble\defaultpostamble\from{datetime.dtx}{dt-spanish.def,package}}
\file{dt-swedish.def}{\usepreamble\defaultpreamble
\usepostamble\defaultpostamble\from{datetime.dtx}{dt-swedish.def,package}}
\file{dt-turkish.def}{\usepreamble\defaultpreamble
\usepostamble\defaultpostamble\from{datetime.dtx}{dt-turkish.def,package}}
\file{dt-UKenglish.def}{\usepreamble\defaultpreamble
\usepostamble\defaultpostamble\from{datetime.dtx}{dt-UKenglish.def,package}}
\file{dt-ukraineb.def}{\usepreamble\defaultpreamble
\usepostamble\defaultpostamble\from{datetime.dtx}{dt-ukraineb.def,package}}
\file{dt-USenglish.def}{\usepreamble\defaultpreamble
\usepostamble\defaultpostamble\from{datetime.dtx}{dt-USenglish.def,package}}
\file{dt-usorbian.def}{\usepreamble\defaultpreamble
\usepostamble\defaultpostamble\from{datetime.dtx}{dt-usorbian.def,package}}
\file{dt-welsh.def}{\usepreamble\defaultpreamble
\usepostamble\defaultpostamble\from{datetime.dtx}{dt-welsh.def,package}}
}

\endbatchfile
}
%\end{verbatim}
%This, I agree, is an unpleasant cludge.
%
%\item The commands \cs{monthname}\meta{language} are not 
%implemented.
%
%\item Some of the languages are not implemented.
%
%\item The package option \texttt{nodate} is not implemented.
%
%\end{itemize}
%
%\section{Troubleshooting}
%
%There is a \sty{datetime} FAQ available at:
%\url{http://theoval.cmp.uea.ac.uk/~nlct/latex/packages/faq/}
%
%
%
%\StopEventually{}
%\section{The Code}
%\iffalse
%    \begin{macrocode}
%<*datetime.sty>
%    \end{macrocode}
%\fi
%\subsection{datetime.sty}
% This section documents the code for \texttt{datetime.sty}
%    \begin{macrocode}
\NeedsTeXFormat{LaTeX2e}
\ProvidesPackage{datetime}[2007/08/20 v2.55 Date Time Package]
%    \end{macrocode}
%\texttt{fmtcount.sty} needs to be loaded here as it defines the 
% command \cs{fmtord} which may be redefined later:
%    \begin{macrocode}
\RequirePackage{fmtcount}
%    \end{macrocode}
%\begin{macro}{\if@dtl@nodate}
% Define a new conditional \cs{if@dt@nodate}.  If it is true, 
% \cs{today} will not be redefined, nor will
% \cs{monthname}, \cs{shortmonthname}, \cs{dayofweek} and 
% \cs{shortdayofweek} be defined.
% Set it to false.
%    \begin{macrocode}
\newif\if@dt@nodate
\@dt@nodatefalse
%    \end{macrocode}
%\end{macro}
%\begin{macro}{\dateseparator}
% Define the character used to separate the numbers in the formats 
% defined by \cs{ddmmyyyy} etc
%\changes{2.43}{2005/02/25}{new}
%    \begin{macrocode}
\newcommand{\dateseparator}{/}
%    \end{macrocode}
%\end{macro}
%\begin{macro}{\if@dt@multilingual}
% Define switch to determine whether to enable multilingual support.
% This check to see if babel package is
% loaded instead of testing 'languagename (following suggestions on
% comp.text.tex)
%\changes{v2.49}{5 Dec 2006}{checks if babel package has been loaded}
% Note that babel must be loaded \emph{before} datetime, otherwise
% it will change the definitions of \cs{date}\meta{lang}.
%\changes{2.55}{2007/08/20}{fixed bug preventing multilingual support}
%    \begin{macrocode}
\newif\if@dt@multilingual
\@ifpackageloaded{babel}{%
\@dt@multilingualtrue}{%
\@ifpackageloaded{ngerman}{%
\@dt@multilingualtrue}{\@dt@multilingualfalse}}
%    \end{macrocode}
%\end{macro}
% Define the \cs{ordinaldate}\meta{language} macros. These are 
% needed because some
% languages only use an ordinal for the first day of
% the month (such as french). This isn't really needed
% here, but the LaTeX2HTML style file needs \cs{ordinaldate}.
%\begin{macro}{\ordinaldateenglish}
% English version:
%    \begin{macrocode}
\newcommand*{\ordinaldateenglish}[1]{\ordinalnum{#1}}
%    \end{macrocode}
%\end{macro}
%\begin{macro}{\ordinaldatewelsh}
% Welsh version:
%    \begin{macrocode}
\newcommand*{\ordinaldatewelsh}[1]{%
#1\ifnum#1=1\/\textsuperscript{a\~n}\fi}
%    \end{macrocode}
%\end{macro}
%\begin{macro}{\ordinaldatebreton}
% Breton version:
%    \begin{macrocode}
\newcommand*{\ordinaldatebreton}[1]{%
#1\ifnum#1=1\/\textsuperscript{a\~n}\fi}
%    \end{macrocode}
%\end{macro}
%\begin{macro}{\ordinaldatefrench}
% French:
%    \begin{macrocode}
\newcommand*{\ordinaldatefrench}[1]{%
#1\ifnum#1=1\ier\fi}
%    \end{macrocode}
%\end{macro}
%\begin{macro}{\ordinaldate}
% If |\ordinaldate|\meta{language} is not defined, then just display
% the number.
%\changes{2.45}{2005/05/23}{new}
%    \begin{macrocode}
\newcommand*{\ordinaldate}[1]{%
\if@dt@multilingual
\@ifundefined{ordinaldate\languagename}{#1}{%
\csname ordinaldate\languagename\endcsname{#1}}%
\else
\ordinalnum{#1}%
\fi}
%    \end{macrocode}
%\end{macro}
%\begin{macro}{\ier}
% In case \verb|\ier| hasn't been defined:%
%\changes{v2.47}{27 Oct 2005}{defined if it doesn't 
% already exist}
%    \begin{macrocode}
\providecommand*{\ier}{\textsuperscript{er}}
%    \end{macrocode}
%\end{macro}
%\begin{macro}{\ifshowdow}
% Now define the declarations that redefine |\formatdate| as they
% are used by the package options. Need a conditional to determine
% whether or not to show the day of week name.
%    \begin{macrocode}
\newif\ifshowdow
%    \end{macrocode}
%\end{macro}
%\changes{1.1}{2002/04/20}{\cs{thedate} added}
%\begin{macro}{\formatdate}
% Initially |\formatdate| does nothing. It will be redefined later.
%\changes{2.0}{2002/10/30}{changes \cs{thedate} to \cs{formatdate}
% to avoid name conflict with other packages/class files.}
%    \begin{macrocode}
\providecommand*{\formatdate}[3]{}
%    \end{macrocode}
%\end{macro}
% Provide counters to store the specified date:
%    \begin{macrocode}
\newcount\@day
\newcount\@month
\newcount\@year
%    \end{macrocode}
%\begin{macro}{\longdate}
% Long date format. (This is the default in the absense of package
% options, babel and datetime.cfg settings.)
%    \begin{macrocode}
\DeclareRobustCommand*{\longdate}{%
\renewcommand*{\formatdate}[3]{%
\ifshowdow\dayofweekname{##1}{##2}{##3} \fi
\@day=##1\relax\@month=##2\relax\@year=##3\relax
\ordinaldate{\the\@day}\ \monthname[\@month], \the\@year}}
%    \end{macrocode}
%\end{macro}
%\begin{macro}{\shortdate}
% Abbreviated version of above
%    \begin{macrocode}
\DeclareRobustCommand*{\shortdate}{%
\renewcommand*{\formatdate}[3]{%
\ifshowdow\shortdayofweekname{##1}{##2}{##3} \fi
\@day=##1\relax\@month=##2\relax\@year=##3\relax
\ordinaldate{\the\@day}\ \shortmonthname[\@month], \the\@year}}
%    \end{macrocode}
%\end{macro}
%\begin{macro}{\twodigit}
% Define |\twodigit| to display a number as two digits. \LaTeX\
% already defines the internal command |\two@digits|, but need
% a command that can be used in |\newdateformat| in the document.
%\changes{2.2}{2004/04/27}{new}
%    \begin{macrocode}
\let\twodigit\two@digits
%    \end{macrocode}
%\end{macro}
%\begin{macro}{\ddmmyyyydate}
% Day/month/year format. (Day and month displayed as two digits,
% year displayed as is.)
%    \begin{macrocode}
\DeclareRobustCommand*{\ddmmyyyydate}{%
\renewcommand*{\formatdate}[3]{%
\@day=##1\relax\@month=##2\relax\@year=##3\relax
\twodigit\@day\dateseparator \twodigit\@month\dateseparator 
\the\@year}}
%    \end{macrocode}
%\end{macro}
%\begin{macro}{\dmyyyydate}
% Day/month/year format. (Numbers all displayed as is.)
%    \begin{macrocode}
\DeclareRobustCommand*{\dmyyyydate}{%
\renewcommand*{\formatdate}[3]{%
\@day=##1\relax\@month=##2\relax\@year=##3\relax
\the\@day\dateseparator \the\@month\dateseparator \the\@year}}
%    \end{macrocode}
%\end{macro}
%\begin{macro}{\ddmmyydate}
% Day/month/year format. (All numbers displayed as two digits.)
%    \begin{macrocode}
\DeclareRobustCommand*{\ddmmyydate}{\renewcommand*{\formatdate}[3]{%
\@day=##1\relax\@month=##2\relax\@year=##3\relax
\@dtctr=\@year%
\@modulo{\@dtctr}{100}%
\twodigit\@day\dateseparator \twodigit\@month\dateseparator 
\twodigit\@dtctr}}
%    \end{macrocode}
%\end{macro}
%\begin{macro}{\dmyydate}
% Day/month/year format. (Day and month displayed as is, year 
% abbreviated to two digits.)
%    \begin{macrocode}
\DeclareRobustCommand*{\dmyydate}{\renewcommand*{\formatdate}[3]{%
\@day=##1\relax\@month=##2\relax\@year=##3\relax
\@dtctr=\@year%
\@modulo{\@dtctr}{100}%
\the\@day\dateseparator \the\@month\dateseparator \twodigit\@dtctr}}
%    \end{macrocode}
%\end{macro}
%\begin{macro}{\textdate}
% Full textual date (English).
%    \begin{macrocode}
\DeclareRobustCommand*{\textdate}{%
\renewcommand*{\formatdate}[3]{%
\ifshowdow\dayofweekname{##1}{##2}{##3} the \fi
\@day=##1\relax\@month=##2\relax\@year=##3\relax
\Ordinalstringnum{\@day}\ of \monthname[\@month], 
\Numberstringnum{\@year}%
}}
%    \end{macrocode}
%\end{macro}
%\begin{macro}{\usdate}
% US format (as per original definition of |\today|)
%    \begin{macrocode}
\DeclareRobustCommand*{\usdate}{%
\renewcommand*{\formatdate}[3]{%
\@day=##1\relax\@month=##2\relax\@year=##3\relax
\monthname[\@month]\ \the\@day, \the\@year}}
%    \end{macrocode}
%\end{macro}
%\begin{macro}{\mmddyyyydate}
% Month/day/year format. (Month and day displayed as two digits,
% year displayed as is.)
%    \begin{macrocode}
\DeclareRobustCommand*{\mmddyyyydate}{%
\renewcommand*{\formatdate}[3]{%
\@day=##1\relax\@month=##2\relax\@year=##3\relax
\twodigit\@month\dateseparator \twodigit\@day\dateseparator 
\the\@year}}
%    \end{macrocode}
%\end{macro}
%\begin{macro}{\mdyyyydate}
% Month/day/year format. (All numbers displayed as is.)
%    \begin{macrocode}
\DeclareRobustCommand*{\mdyyyydate}{%
\renewcommand*{\formatdate}[3]{%
\@day=##1\relax\@month=##2\relax\@year=##3\relax
\the\@month\dateseparator \the\@day\dateseparator \the\@year}}
%    \end{macrocode}
%\end{macro}
%\begin{macro}{\mmddyydate}
% Month/day/year format. (All numbers displayed with two digits.)
%    \begin{macrocode}
\DeclareRobustCommand*{\mmddyydate}{\renewcommand*{\formatdate}[3]{%
\@day=##1\relax\@month=##2\relax\@year=##3\relax
\@dtctr=\@year%
\@modulo{\@dtctr}{100}%
\twodigit\@month\dateseparator \twodigit\@day\dateseparator 
\twodigit\@dtctr}}
%    \end{macrocode}
%\end{macro}
%\begin{macro}{\mdyydate}
% Month/day/year format. (Month and day displayed as is, year 
% abbreviated to two digits.)
%    \begin{macrocode}
\DeclareRobustCommand*{\mdyydate}{\renewcommand*{\formatdate}[3]{%
\@day=##1\relax\@month=##2\relax\@year=##3\relax
\@dtctr=\@year%
\@modulo{\@dtctr}{100}%
\the\@month\dateseparator \the\@day\dateseparator \twodigit\@dtctr}}
%    \end{macrocode}
%\end{macro}
%\begin{macro}{\newdate}
% Define commands to save dates
% and later format them. Store a given date:
%\changes{2.45}{2005/05/01}{new}
%    \begin{macrocode}
\newcommand*{\newdate}[4]{%
\@ifundefined{date@#1@y}{%
\@namedef{date@#1@d}{#2}%
\@namedef{date@#1@m}{#3}%
\@namedef{date@#1@y}{#4}}{%
\PackageError{datetime}{Date `#1' already defined}{}}}
%    \end{macrocode}
%\end{macro}
%\begin{macro}{\getdateyear}
% Display year from previously stored date
%    \begin{macrocode}
\newcommand*{\getdateyear}[1]{%
\@ifundefined{date@#1@y}{%
\PackageError{datetime}{Date `#1' not defined}{}}{%
\csname date@#1@y\endcsname}}
%    \end{macrocode}
%\end{macro}
%\begin{macro}{\getdatemonth}
% Display month from previously stored date
%    \begin{macrocode}
\newcommand*{\getdatemonth}[1]{%
\@ifundefined{date@#1@m}{%
\PackageError{datetime}{Date `#1' not defined}{}}{%
\csname date@#1@m\endcsname}}
%    \end{macrocode}
%\end{macro}
%\begin{macro}{\getdateday}
% Display day from previously stored date
%    \begin{macrocode}
\newcommand{\getdateday}[1]{%
\@ifundefined{date@#1@d}{%
\PackageError{datetime}{Date `#1' not defined}{}}{%
\csname date@#1@d\endcsname}}
%    \end{macrocode}
%\end{macro}
%\begin{macro}{\displaydate}
% Display a previously stored date using current date format
%    \begin{macrocode}
\newcommand*{\displaydate}[1]{%
\@ifundefined{date@#1@y}{%
\PackageError{datetime}{Date `#1' not defined}{}}{%
\formatdate{\csname date@#1@d\endcsname}{%
\csname date@#1@m\endcsname}{%
\csname date@#1@y\endcsname}}}
%    \end{macrocode}
%\end{macro}
%\begin{macro}{\currenttime}
% (New to v2.3) Define |\currenttime| which will print the
% time according to the current format. Set it to 24hr time by default
%    \begin{macrocode}
\DeclareRobustCommand*{\currenttime}{\xxivtime}
%    \end{macrocode}
%\end{macro}
%\begin{macro}{\timeseparator}
%\changes{2.43}{2005/02/25}{new}
% Define separator for numerical times:
%    \begin{macrocode}
\newcommand*{\timeseparator}{:}
%    \end{macrocode}
%\end{macro}
%\begin{macro}{\settimeformat}
% Switch to specified time format:
%\changes{2.3}{2004/05/01}{new}
%    \begin{macrocode}
\providecommand*{\settimeformat}[1]{%
\DeclareRobustCommand*\currenttime{\csname#1\endcsname}}
%    \end{macrocode}
%\end{macro}
% Set defaults:
%    \begin{macrocode}
\longdate
\showdowtrue
%    \end{macrocode}
% Load in specifications from configuration file:
%\changes{2.4}{2004/09/25}{added provision for configuration file
%datetime.cfg}
%    \begin{macrocode}
\InputIfFileExists{datetime.cfg}{\PackageInfo{datetime}{%
Loading local datetime configurations}}{%
\PackageInfo{datetime}{No datetime.cfg file found, using default
settings}}
%    \end{macrocode}
% Specify the package options, specify default options and process
%    \begin{macrocode}
\RequirePackage{fmtcount}
\DeclareOption{long}{\longdate}
\DeclareOption{short}{\shortdate}
\DeclareOption{ddmmyyyy}{\ddmmyyyydate}
\DeclareOption{dmyyyy}{\dmyyyydate}
\DeclareOption{ddmmyy}{\ddmmyydate}
\DeclareOption{dmyy}{\dmyydate}
\DeclareOption{text}{\textdate}
\DeclareOption{us}{\usdate}
\DeclareOption{mmddyyyy}{\mmddyyyydate}
\DeclareOption{mdyyyy}{\mdyyyydate}
\DeclareOption{mmddyy}{\mmddyydate}
\DeclareOption{mdyy}{\mdyydate}
\DeclareOption{level}{\fmtcountsetoptions{fmtord=level}}
\DeclareOption{raise}{\fmtcountsetoptions{fmtord=raise}}
\DeclareOption{dayofweek}{\showdowtrue}
\DeclareOption{nodayofweek}{\showdowfalse}
\DeclareOption{nodate}{\@dt@nodatetrue}
\DeclareOption{24hr}{\settimeformat{xxivtime}}
\DeclareOption{12hr}{\settimeformat{ampmtime}}
\DeclareOption{oclock}{\settimeformat{oclock}}
%    \end{macrocode}
% Multilingual support. These package options shouldn't really be
% needed if babel has already been loaded.
%    \begin{macrocode}
\DeclareOption{austrian}{\input{dt-austrian.def}}
\DeclareOption{bahasa}{\input{dt-bahasa.def}}
\DeclareOption{basque}{\input{dt-basque.def}}
\DeclareOption{breton}{\input{dt-breton.def}}
\DeclareOption{bulgarian}{\input{dt-bulgarian.def}}
\DeclareOption{catalan}{\input{dt-catalan.def}}
\DeclareOption{croatian}{\input{dt-croatian.def}}
\DeclareOption{czech}{\input{dt-czech.def}}
\DeclareOption{danish}{\input{dt-danish.def}}
\DeclareOption{dutch}{\input{dt-dutch.def}}
\DeclareOption{esperanto}{\input{dt-esperanto.def}}
\DeclareOption{estonian}{\input{dt-estonian.def}}
\DeclareOption{finnish}{\input{dt-finnish.def}}
\DeclareOption{french}{\input{dt-french.def}}
\DeclareOption{galician}{\input{dt-galician.def}}
\DeclareOption{german}{\input{dt-german.def}}
\DeclareOption{greek}{\input{dt-greek.def}}
\DeclareOption{hebrew}{\input{dt-hebrew.def}}
\DeclareOption{icelandic}{\input{dt-icelandic.def}}
\DeclareOption{irish}{\input{dt-irish.def}}
\DeclareOption{italian}{\input{dt-italian.def}}
\DeclareOption{latin}{\input{dt-latin.def}}
\DeclareOption{lsorbian}{\input{dt-lsorbian.def}}
\DeclareOption{magyar}{\input{dt-magyar.def}}
\DeclareOption{naustrian}{\input{dt-naustrian.def}}
\DeclareOption{ngerman}{\input{dt-ngerman.def}}
\DeclareOption{norsk}{\input{dt-norsk.def}}
\DeclareOption{polish}{\input{dt-polish.def}}
\DeclareOption{portuges}{\input{dt-portuges.def}}
\DeclareOption{romanian}{\input{dt-romanian.def}}
\DeclareOption{russian}{%%
%% This is file `dt-russian.def',
%% generated with the docstrip utility.
%%
%% The original source files were:
%%
%% datetime.dtx  (with options: `dt-russian.def,package')
%% 
%%  datetime.dtx
%%  Copyright 2007 Nicola Talbot
%% 
%%  This work may be distributed and/or modified under the
%%  conditions of the LaTeX Project Public License, either version 1.3
%%  of this license of (at your option) any later version.
%%  The latest version of this license is in
%%    http://www.latex-project.org/lppl.txt
%%  and version 1.3 or later is part of all distributions of LaTeX
%%  version 2005/12/01 or later.
%% 
%%  This work has the LPPL maintenance status `maintained'.
%% 
%%  The Current Maintainer of this work is Nicola Talbot.
%% 
%%  This work consists of the files datetime.dtx and datetime.ins and the derived files datetime.sty, dt-austrian.def, dt-bahasa.def, dt-basque.def, dt-breton.def, dt-british.def, dt-bulgarian.def, dt-catalan.def, dt-croatian.def, dt-czech.def, dt-danish.def, dt-dutch.def, dt-esperanto.def, dt-estonian.def, dt-finnish.def, dt-french.def, dt-galician.def, dt-german.def, dt-greek.def, dt-hebrew.def, dt-icelandic.def, dt-irish.def, dt-italian.def, dt-latin.def, dt-lsorbian.def, dt-magyar.def, dt-naustrian.def, dt-ngerman.def, dt-norsk.def, dt-polish.def, dt-portuges.def, dt-romanian.def, dt-russian.def, dt-samin.def, dt-scottish.def, dt-serbian.def, dt-slovak.def, dt-slovene.def, dt-spanish.def, dt-swedish.def, dt-turkish.def, dt-UKenglish.def, dt-ukraineb.def, dt-USenglish.def, dt-usorbian.def, dt-welsh.def.
%% 
%% \CharacterTable
%%  {Upper-case    \A\B\C\D\E\F\G\H\I\J\K\L\M\N\O\P\Q\R\S\T\U\V\W\X\Y\Z
%%   Lower-case    \a\b\c\d\e\f\g\h\i\j\k\l\m\n\o\p\q\r\s\t\u\v\w\x\y\z
%%   Digits        \0\1\2\3\4\5\6\7\8\9
%%   Exclamation   \!     Double quote  \"     Hash (number) \#
%%   Dollar        \$     Percent       \%     Ampersand     \&
%%   Acute accent  \'     Left paren    \(     Right paren   \)
%%   Asterisk      \*     Plus          \+     Comma         \,
%%   Minus         \-     Point         \.     Solidus       \/
%%   Colon         \:     Semicolon     \;     Less than     \<
%%   Equals        \=     Greater than  \>     Question mark \?
%%   Commercial at \@     Left bracket  \[     Backslash     \\
%%   Right bracket \]     Circumflex    \^     Underscore    \_
%%   Grave accent  \`     Left brace    \{     Vertical bar  \|
%%   Right brace   \}     Tilde         \~}
\ProvidesFile{dt-russian.def}[2004/10/31]
\providecommand{\monthnamerussian}[1][\month]{%
\@orgargctr=#1\relax
\ifcase\@orgargctr
\PackageError{datetime}{Invalid Month number \the\@orgargctr}{%
Month numbers should go from 1 to 12}%
\or \cyrya\cyrn\cyrv\cyra\cyrr\cyrya\or
    \cyrf\cyre\cyrv\cyrr\cyra\cyrl\cyrya\or
    \cyrm\cyra\cyrr\cyrt\cyra\or
    \cyra\cyrp\cyrr\cyre\cyrl\cyrya\or
    \cyrm\cyra\cyrya\or
    \cyri\cyryu\cyrn\cyrya\or
    \cyri\cyryu\cyrl\cyrya\or
    \cyra\cyrv\cyrg\cyru\cyrs\cyrt\cyra\or
    \cyrs\cyre\cyrn\cyrt\cyrya\cyrb\cyrr\cyrya\or
    \cyro\cyrk\cyrt\cyrya\cyrb\cyrr\cyrya\or
    \cyrn\cyro\cyrya\cyrb\cyrr\cyrya\or
    \cyrd\cyre\cyrk\cyra\cyrb\cyrr\cyrya%
\else
\PackageError{datetime}{Invalid Month number \the\@orgargctr}{%
Month numbers should go from 1 to 12}%
\fi}
\DeclareRobustCommand*\daterussian{%
\renewcommand{\formatdate}[3]{%
\@day=##1\relax\@month=##2\relax\@year=##3\relax
\number\@day~\monthnamerussian[\@month]\ \number\@year~\cyrg.}}
\endinput
%%
%% End of file `dt-russian.def'.
}
\DeclareOption{samin}{\input{dt-samin.def}}
\DeclareOption{scottish}{\input{dt-scottish.def}}
\DeclareOption{serbian}{\input{dt-serbian.def}}
\DeclareOption{slovak}{\input{dt-slovak.def}}
\DeclareOption{slovene}{\input{dt-slovene.def}}
\DeclareOption{spanish}{\input{dt-spanish.def}}
\DeclareOption{swedish}{\input{dt-swedish.def}}
\DeclareOption{turkish}{\input{dt-turkish.def}}
\DeclareOption{ukraineb}{\input{dt-ukraineb.def}}
\DeclareOption{usorbian}{\input{dt-usorbian.def}}
\DeclareOption{welsh}{\input{dt-welsh.def}}
%    \end{macrocode}
% Process package options
%    \begin{macrocode}
\ProcessOptions
%    \end{macrocode}
% Need ifthen package for conditional stuff.
%    \begin{macrocode}
\RequirePackage{ifthen}
%    \end{macrocode}
% \subsubsection{Date Macros}
% Define the macro that prints the month name.
% (Only define this command if @dt@nodate is false)
%    \begin{macrocode}
\if@dt@nodate
%    \end{macrocode}
% The |nodate| option was used, so just print informative message,
% and do nothing else.
%    \begin{macrocode}
\PackageInfo{datetime}{option "nodate" used, so not defining 
\string\monthname}
\else
%    \end{macrocode}
%\begin{macro}{\monthnameenglish}
%    \begin{macrocode}
\providecommand*{\monthnameenglish}[1][\month]{%
\@orgargctr=#1\relax
\ifcase\@orgargctr
\PackageError{datetime}{Invalid Month number \the\@orgargctr}{Month 
numbers should go from 1 (January) to 12 (December)}%
\or January%
\or February%
\or March%
\or April%
\or May%
\or June%
\or July%
\or August%
\or September%
\or October%
\or November%
\or December%
\else \PackageError{datetime}{Invalid Month number \the\@orgargctr}{%
Month numbers should go from 1 (January) to 12 (December)}%
\fi}
%    \end{macrocode}
%\end{macro}
%\begin{macro}{\monthname}
%Define \verb|\monthname| to be language dependent. If there
%is no \verb|\monthname|\meta{language}, defaults to English.
%    \begin{macrocode}
\newcommand*{\monthname}[1][\month]{%
\if@dt@multilingual
\@ifundefined{monthname\languagename}{%
\PackageWarning{datetime}{No month names provided for language
'\languagename'}%
\monthnameenglish[#1]}{\csname monthname\languagename\endcsname[#1]}%
\else
\monthnameenglish[#1]%
\fi}
%    \end{macrocode}
%\end{macro}
% End of |\if@dt@nodate| else part:
%    \begin{macrocode}
\fi
%    \end{macrocode}
% Define the macro that prints the abbreviated month name
% (Again, only do this if @dt@nodate is false)
%    \begin{macrocode}
\if@dt@nodate
%    \end{macrocode}
% The |nodate| option was used, so just print informative message,
% and do nothing else.
%    \begin{macrocode}
\PackageInfo{datetime}{option "nodate" used, so not defining 
\protect\shortmonthname}
\else
%    \end{macrocode}
%\begin{macro}{\shortmonthnameenglish}
%\changes{2.1}{2003/12/17}{fixed bug producing an error message
% in December}
%    \begin{macrocode}
\providecommand*{\shortmonthnameenglish}[1][\month]{%
\@orgargctr=#1\relax
\ifcase\@orgargctr
\PackageError{datetime}{Invalid Month number \the\@orgargctr}{Month 
numbers should go from 1 (jan) to 12 (dec)}%
\or Jan%
\or Feb%
\or Mar%
\or Apr%
\or May%
\or Jun%
\or Jul%
\or Aug%
\or Sept%
\or Oct%
\or Nov%
\or Dec%
\else%
\PackageError{datetime}{Invalid Month number \the\@orgargctr}{Month 
numbers should go from 1 (jan) to 12 (dec)}%
\fi}
%    \end{macrocode}
%\end{macro}
%\begin{macro}{\shortmonthname}
%Define \verb|\shortmonthname| to be language dependent. If there
%is no \verb|\shortmonthname|\meta{language}, defaults to English.
%    \begin{macrocode}
\newcommand*{\shortmonthname}[1][\month]{%
\if@dt@multilingual
\@ifundefined{shortmonthname\languagename}{%
\PackageWarning{datetime}{No abbreviated month name defined for
language '\languagename', using full version instead}%
\monthname[#1]}{%
\csname shortmonthname\languagename\endcsname[#1]}%
\else
\shortmonthnameenglish[#1]%
\fi}
%    \end{macrocode}
%\end{macro}
% End of |\if@dt@nodate| else part:
%    \begin{macrocode}
\fi
%    \end{macrocode}
% Define macros needed to compute the weekday
% (Again, only do this if @dt@nodate is false)
%\begin{macro}{\ifleapyear}
% Need to define |\ifleapyear| regardless of @dt@nodate otherwise \LaTeX\ won't
% match |\ifleapyear| with |\fi|
%    \begin{macrocode}
\newif\ifleapyear
%    \end{macrocode}
%\end{macro}
% Define temporary counter for arithmetic.
%    \begin{macrocode}
\newcount\@dtctr
%    \end{macrocode}
% If nodate, add a reminder in the log file that \verb|\dayofweek|
% is not defined.
%    \begin{macrocode}
\if@dt@nodate
\PackageInfo{datetime}{option "nodate" used, so not defining 
\string\dayofweek \space or \string\shortdayofweek}
\else
%    \end{macrocode}
% Using the algorithm documented at
% http://userpages.wittenberg.edu/bshelburne/Comp150/DayofWeek.htm
% Syntax: \verb|testifleapyear{|\meta{year}\verb|}| sets 
% conditional \verb|\ifleapyear|.
%    \begin{macrocode}
\providecommand*{\testifleapyear}[1]{%
\leapyearfalse
\@year=#1\relax
\@dtctr=\@year
\@modulo{\@dtctr}{400}%
\ifnum\@dtctr=0\relax
\leapyeartrue %         year mod 400 = 0 => leap year
\else
\@dtctr=\@year
\@modulo{\@dtctr}{100}%
\ifnum\@dtctr=0\relax
\leapyearfalse %        year mod 100 = 0 && year mod 400 != 0 => not a leap year
\else
\@dtctr=\@year
\@modulo{\@dtctr}{4}%
\ifnum\@dtctr=0\relax
\leapyeartrue %         year mod 4 = 0 && year mod 100 != 0 => leap year
\fi
\fi
\fi
}
%    \end{macrocode}
%\begin{macro}{\dayofyear}
% Count register in which to store the day of the year.
%    \begin{macrocode}
\newcount\dayofyear
%    \end{macrocode}
%\end{macro}
%\begin{macro}{\computedayofyear}
% \cs{computedayofyear}\marg{day}\marg{month}\marg{year}\par
% Computes the day of year. Result will be stored in \verb|\dayofyear|
%    \begin{macrocode}
\providecommand*{\computedayofyear}[3]{%
\testifleapyear{#3}%
\dayofyear=0\relax
\@day=#1\relax \@month=#2\relax \@year=#3\relax
\ifcase\@month
\or
\or \advance\dayofyear by 31\relax
\or \advance\dayofyear by 59\relax
\or \advance\dayofyear by 90\relax
\or \advance\dayofyear by 120\relax
\or \advance\dayofyear by 151\relax
\or \advance\dayofyear by 181\relax
\or \advance\dayofyear by 212\relax
\or \advance\dayofyear by 243\relax
\or \advance\dayofyear by 273\relax
\or \advance\dayofyear by 304\relax
\or \advance\dayofyear by 334\relax
\else
\PackageError{datetime}{Invalid month number}{The second argument to 
\string\computedayofyear \space should lie in the range 1-12}%
\fi
\ifnum\@month>2\relax
\ifleapyear\advance\dayofyear by 1\relax\fi
\fi
\advance\dayofyear by \@day\relax
}
%    \end{macrocode}
%\end{macro}
%\begin{macro}{\dayofweek}
% Count register in which to store the day of the week.
%    \begin{macrocode}
\newcount\dayofweek
%    \end{macrocode}
%\end{macro}
%\begin{macro}{\computedayofweek}
% \cs{computedayofweek}\marg{day}\marg{month}\marg{year}\par
% Computes the day of week index. The result is stored in 
% |\dayofweek|.
%    \begin{macrocode}
\providecommand*{\computedayofweek}[3]{%
\computedayofyear{#1}{#2}{#3}%
\@dtctr=#3\relax
\advance\@dtctr by -1901\relax
\@modulo{\@dtctr}{28}%
\dayofweek=\@dtctr
\divide\dayofweek by 4\relax
\advance\dayofweek by \@dtctr
\advance\dayofweek by 2\relax
\@modulo{\dayofweek}{7}%
\advance\dayofweek by \dayofyear
\advance\dayofweek by -1\relax
\@modulo{\dayofweek}{7}%
\advance\dayofweek by 1\relax}
%    \end{macrocode}
%\end{macro}
%\begin{macro}{\dayofweeknameidenglish}
% Given the day of week index, print associated the English name.
%    \begin{macrocode}
\providecommand*{\dayofweeknameidenglish}[1]{%
\ifcase#1\relax
\or Sunday%
\or Monday%
\or Tuesday%
\or Wednesday%
\or Thursday%
\or Friday%
\or Saturday%
\fi}
%    \end{macrocode}
%\end{macro}
%\begin{macro}{\dayofweeknameid}
% Given the day of week index, print the associated name in the
% current language. If there is none defined for that language
% default to English.
%    \begin{macrocode}
\providecommand*{\dayofweeknameid}[1]{%
\if@dt@multilingual
\@ifundefined{dayofweeknameid\languagename}{%
\ifthenelse{\equal{\languagename}{nohyphenation}}{}{%
\PackageWarning{datetime}{No week day names defined for language 
'\languagename', defaulting to English}}%
\dayofweeknameidenglish{#1}}{%
\csname dayofweeknameid\languagename\endcsname{#1}}%
\else
\dayofweeknameidenglish{#1}%
\fi
}
%    \end{macrocode}
%\end{macro}
%\begin{macro}{\dayofweekname}
% Print the day of week name for the specified date.
%    \begin{macrocode}
\providecommand*{\dayofweekname}[3]{%
\computedayofweek{#1}{#2}{#3}%
\dayofweeknameid{\dayofweek}%
}
%    \end{macrocode}
%\end{macro}
%\begin{macro}{\thisdayofweekname}
% Print today's day of week name.
%    \begin{macrocode}
\providecommand*{\thisdayofweekname}{%
\dayofweekname{\day}{\month}{\year}}
%    \end{macrocode}
%\end{macro}
%\begin{macro}{\shortdayofweeknameidenglish}
% As before, but for abbreviated day of week name. English version:
%    \begin{macrocode}
\providecommand*{\shortdayofweeknameidenglish}[1]{%
\ifcase#1\relax
\or Sun%
\or Mon%
\or Tue%
\or Wed%
\or Thu%
\or Fri%
\or Sat%
\fi}
%    \end{macrocode}
%\end{macro}
%\begin{macro}{\shortdayofweekname}
% Language dependant version:
%    \begin{macrocode}
\providecommand*{\shortdayofweekname}[3]{%
\computedayofweek{#1}{#2}{#3}%
\if@dt@multilingual
\@ifundefined{shortdayofweeknameid\languagename}{%
\ifthenelse{\equal{\languagename}{nohyphenation}}{}{%
\PackageWarning{datetime}{No abbreviated week day names defined for 
language '\languagename', defaulting to long version}}%
\dayofweeknameid{\dayofweek}}{%
\csname shortdayofweeknameid\languagename\endcsname\dayofweek}%
\else
\shortdayofweeknameidenglish{\dayofweek}%
\fi
}
%    \end{macrocode}
%\end{macro}
%\begin{macro}{\thisshortdayofweekname}
% Today's week day name, abbreviated:
%    \begin{macrocode}
\providecommand*{\thisshortdayofweekname}{%
\dayofweekname{\day}{\month}{\year}}
%    \end{macrocode}
%\end{macro}
% End of |\if@dt@nodate| else part.
%    \begin{macrocode}
\fi
%    \end{macrocode}
%\begin{macro}{\today}
% Redefine |\today| so that it uses |\formatdate|.
% (Only do this if @dt@nodate is false)
%\changes{2.46}{2005/09/30}{defined using \cs{DeclareRobustCommand}}
%\changes{2.41}{2004/10/22}{defined using \cs{providecommand}}
%    \begin{macrocode}
\if@dt@nodate
\else
\DeclareRobustCommand*{\today}{\formatdate{\day}{\month}{\year}}
\fi
%    \end{macrocode}
%\end{macro}
%Check to see if babel package has redefined |\today|.
%\changes{2.53}{11 June 2007}{added check for \cs{dateUKenglish} and 
%\cs{dateUSenglish}}
%\changes{2.54}{15 June 2007}{added check for \cs{datebritish}}
%    \begin{macrocode}
\if@dt@nodate
\else
\@ifundefined{dateenglish}{}{\let\dateenglish\longdate}
\@ifundefined{dateUKenglish}{}{%%
%% This is file `dt-UKenglish.def',
%% generated with the docstrip utility.
%%
%% The original source files were:
%%
%% datetime.dtx  (with options: `dt-UKenglish.def,package')
%% 
%%  datetime.dtx
%%  Copyright 2007 Nicola Talbot
%% 
%%  This work may be distributed and/or modified under the
%%  conditions of the LaTeX Project Public License, either version 1.3
%%  of this license of (at your option) any later version.
%%  The latest version of this license is in
%%    http://www.latex-project.org/lppl.txt
%%  and version 1.3 or later is part of all distributions of LaTeX
%%  version 2005/12/01 or later.
%% 
%%  This work has the LPPL maintenance status `maintained'.
%% 
%%  The Current Maintainer of this work is Nicola Talbot.
%% 
%%  This work consists of the files datetime.dtx and datetime.ins and the derived files datetime.sty, dt-austrian.def, dt-bahasa.def, dt-basque.def, dt-breton.def, dt-british.def, dt-bulgarian.def, dt-catalan.def, dt-croatian.def, dt-czech.def, dt-danish.def, dt-dutch.def, dt-esperanto.def, dt-estonian.def, dt-finnish.def, dt-french.def, dt-galician.def, dt-german.def, dt-greek.def, dt-hebrew.def, dt-icelandic.def, dt-irish.def, dt-italian.def, dt-latin.def, dt-lsorbian.def, dt-magyar.def, dt-naustrian.def, dt-ngerman.def, dt-norsk.def, dt-polish.def, dt-portuges.def, dt-romanian.def, dt-russian.def, dt-samin.def, dt-scottish.def, dt-serbian.def, dt-slovak.def, dt-slovene.def, dt-spanish.def, dt-swedish.def, dt-turkish.def, dt-UKenglish.def, dt-ukraineb.def, dt-USenglish.def, dt-usorbian.def, dt-welsh.def.
%% 
%% \CharacterTable
%%  {Upper-case    \A\B\C\D\E\F\G\H\I\J\K\L\M\N\O\P\Q\R\S\T\U\V\W\X\Y\Z
%%   Lower-case    \a\b\c\d\e\f\g\h\i\j\k\l\m\n\o\p\q\r\s\t\u\v\w\x\y\z
%%   Digits        \0\1\2\3\4\5\6\7\8\9
%%   Exclamation   \!     Double quote  \"     Hash (number) \#
%%   Dollar        \$     Percent       \%     Ampersand     \&
%%   Acute accent  \'     Left paren    \(     Right paren   \)
%%   Asterisk      \*     Plus          \+     Comma         \,
%%   Minus         \-     Point         \.     Solidus       \/
%%   Colon         \:     Semicolon     \;     Less than     \<
%%   Equals        \=     Greater than  \>     Question mark \?
%%   Commercial at \@     Left bracket  \[     Backslash     \\
%%   Right bracket \]     Circumflex    \^     Underscore    \_
%%   Grave accent  \`     Left brace    \{     Vertical bar  \|
%%   Right brace   \}     Tilde         \~}
\ProvidesFile{dt-UKenglish.def}[2007/06/11]
\let\dateUKenglish\longdate

\let\monthnameUKenglish\monthnameenglish
\let\shortmonthnameUKenglish\shortmonthnameenglish

\let\dayofweeknameidUKenglish\dayofweeknameidenglish
\let\shortdayofweeknameidUKenglish\shortdayofweeknameidenglish
\endinput
%%
%% End of file `dt-UKenglish.def'.
}
\@ifundefined{dateUSenglish}{}{%%
%% This is file `dt-USenglish.def',
%% generated with the docstrip utility.
%%
%% The original source files were:
%%
%% datetime.dtx  (with options: `dt-USenglish.def,package')
%% 
%%  datetime.dtx
%%  Copyright 2007 Nicola Talbot
%% 
%%  This work may be distributed and/or modified under the
%%  conditions of the LaTeX Project Public License, either version 1.3
%%  of this license of (at your option) any later version.
%%  The latest version of this license is in
%%    http://www.latex-project.org/lppl.txt
%%  and version 1.3 or later is part of all distributions of LaTeX
%%  version 2005/12/01 or later.
%% 
%%  This work has the LPPL maintenance status `maintained'.
%% 
%%  The Current Maintainer of this work is Nicola Talbot.
%% 
%%  This work consists of the files datetime.dtx and datetime.ins and the derived files datetime.sty, dt-austrian.def, dt-bahasa.def, dt-basque.def, dt-breton.def, dt-british.def, dt-bulgarian.def, dt-catalan.def, dt-croatian.def, dt-czech.def, dt-danish.def, dt-dutch.def, dt-esperanto.def, dt-estonian.def, dt-finnish.def, dt-french.def, dt-galician.def, dt-german.def, dt-greek.def, dt-hebrew.def, dt-icelandic.def, dt-irish.def, dt-italian.def, dt-latin.def, dt-lsorbian.def, dt-magyar.def, dt-naustrian.def, dt-ngerman.def, dt-norsk.def, dt-polish.def, dt-portuges.def, dt-romanian.def, dt-russian.def, dt-samin.def, dt-scottish.def, dt-serbian.def, dt-slovak.def, dt-slovene.def, dt-spanish.def, dt-swedish.def, dt-turkish.def, dt-UKenglish.def, dt-ukraineb.def, dt-USenglish.def, dt-usorbian.def, dt-welsh.def.
%% 
%% \CharacterTable
%%  {Upper-case    \A\B\C\D\E\F\G\H\I\J\K\L\M\N\O\P\Q\R\S\T\U\V\W\X\Y\Z
%%   Lower-case    \a\b\c\d\e\f\g\h\i\j\k\l\m\n\o\p\q\r\s\t\u\v\w\x\y\z
%%   Digits        \0\1\2\3\4\5\6\7\8\9
%%   Exclamation   \!     Double quote  \"     Hash (number) \#
%%   Dollar        \$     Percent       \%     Ampersand     \&
%%   Acute accent  \'     Left paren    \(     Right paren   \)
%%   Asterisk      \*     Plus          \+     Comma         \,
%%   Minus         \-     Point         \.     Solidus       \/
%%   Colon         \:     Semicolon     \;     Less than     \<
%%   Equals        \=     Greater than  \>     Question mark \?
%%   Commercial at \@     Left bracket  \[     Backslash     \\
%%   Right bracket \]     Circumflex    \^     Underscore    \_
%%   Grave accent  \`     Left brace    \{     Vertical bar  \|
%%   Right brace   \}     Tilde         \~}
\ProvidesFile{dt-USenglish.def}[2007/06/11]
\let\dateUSenglish\usdate

\let\monthnameUSenglish\monthnameenglish
\let\shortmonthnameUSenglish\shortmonthnameenglish

\let\dayofweeknameidUSenglish\dayofweeknameidenglish
\let\shortdayofweeknameidUSenglish\shortdayofweeknameidenglish
\endinput
%%
%% End of file `dt-USenglish.def'.
}
\@ifundefined{datebritish}{}{\input{dt-british.def}}
\@ifundefined{dateaustrian}{}{\input{dt-austrian.def}}
\@ifundefined{datebahasa}{}{\input{dt-bahasa.def}}
\@ifundefined{datebasque}{}{\input{dt-basque.def}}
\@ifundefined{datebreton}{}{\input{dt-breton.def}}
\@ifundefined{datebulgarian}{}{\input{dt-bulgarian.def}}
\@ifundefined{datecatalan}{}{\input{dt-catalan.def}}
\@ifundefined{datecroatian}{}{\input{dt-croatian.def}}
\@ifundefined{dateczech}{}{\input{dt-czech.def}}
\@ifundefined{datedanish}{}{\input{dt-danish.def}}
\@ifundefined{datedutch}{}{\input{dt-dutch.def}}
\@ifundefined{dateesperanto}{}{\input{dt-esperanto.def}}
\@ifundefined{dateestonian}{}{\input{dt-estonian.def}}
\@ifundefined{datefinnish}{}{\input{dt-finnish.def}}
\@ifundefined{datefrench}{}{\input{dt-french.def}}
\@ifundefined{dategalician}{}{\input{dt-galician.def}}
\@ifundefined{dategerman}{}{\input{dt-german.def}}
\@ifundefined{dategreek}{}{\input{dt-greek.def}}
\@ifundefined{datehebrew}{}{\input{dt-hebrew.def}}
\@ifundefined{dateicelandic}{}{\input{dt-icelandic.def}}
\@ifundefined{dateirish}{}{\input{dt-irish.def}}
\@ifundefined{dateitalian}{}{\input{dt-italian.def}}
\@ifundefined{datelatin}{}{\input{dt-latin.def}}
\@ifundefined{datelsorbian}{}{\input{dt-lsorbian.def}}
\@ifundefined{datemagyar}{}{\input{dt-magyar.def}}
\@ifundefined{datenaustrian}{}{\input{dt-naustrian.def}}
\@ifundefined{datengerman}{}{\input{dt-ngerman.def}}
\@ifundefined{datenorsk}{}{\input{dt-norsk.def}}
\@ifundefined{datepolish}{}{\input{dt-polish.def}}
\@ifundefined{dateportuges}{}{\input{dt-portuges.def}}
\@ifundefined{dateromanian}{}{\input{dt-romanian.def}}
\@ifundefined{daterussian}{}{%%
%% This is file `dt-russian.def',
%% generated with the docstrip utility.
%%
%% The original source files were:
%%
%% datetime.dtx  (with options: `dt-russian.def,package')
%% 
%%  datetime.dtx
%%  Copyright 2007 Nicola Talbot
%% 
%%  This work may be distributed and/or modified under the
%%  conditions of the LaTeX Project Public License, either version 1.3
%%  of this license of (at your option) any later version.
%%  The latest version of this license is in
%%    http://www.latex-project.org/lppl.txt
%%  and version 1.3 or later is part of all distributions of LaTeX
%%  version 2005/12/01 or later.
%% 
%%  This work has the LPPL maintenance status `maintained'.
%% 
%%  The Current Maintainer of this work is Nicola Talbot.
%% 
%%  This work consists of the files datetime.dtx and datetime.ins and the derived files datetime.sty, dt-austrian.def, dt-bahasa.def, dt-basque.def, dt-breton.def, dt-british.def, dt-bulgarian.def, dt-catalan.def, dt-croatian.def, dt-czech.def, dt-danish.def, dt-dutch.def, dt-esperanto.def, dt-estonian.def, dt-finnish.def, dt-french.def, dt-galician.def, dt-german.def, dt-greek.def, dt-hebrew.def, dt-icelandic.def, dt-irish.def, dt-italian.def, dt-latin.def, dt-lsorbian.def, dt-magyar.def, dt-naustrian.def, dt-ngerman.def, dt-norsk.def, dt-polish.def, dt-portuges.def, dt-romanian.def, dt-russian.def, dt-samin.def, dt-scottish.def, dt-serbian.def, dt-slovak.def, dt-slovene.def, dt-spanish.def, dt-swedish.def, dt-turkish.def, dt-UKenglish.def, dt-ukraineb.def, dt-USenglish.def, dt-usorbian.def, dt-welsh.def.
%% 
%% \CharacterTable
%%  {Upper-case    \A\B\C\D\E\F\G\H\I\J\K\L\M\N\O\P\Q\R\S\T\U\V\W\X\Y\Z
%%   Lower-case    \a\b\c\d\e\f\g\h\i\j\k\l\m\n\o\p\q\r\s\t\u\v\w\x\y\z
%%   Digits        \0\1\2\3\4\5\6\7\8\9
%%   Exclamation   \!     Double quote  \"     Hash (number) \#
%%   Dollar        \$     Percent       \%     Ampersand     \&
%%   Acute accent  \'     Left paren    \(     Right paren   \)
%%   Asterisk      \*     Plus          \+     Comma         \,
%%   Minus         \-     Point         \.     Solidus       \/
%%   Colon         \:     Semicolon     \;     Less than     \<
%%   Equals        \=     Greater than  \>     Question mark \?
%%   Commercial at \@     Left bracket  \[     Backslash     \\
%%   Right bracket \]     Circumflex    \^     Underscore    \_
%%   Grave accent  \`     Left brace    \{     Vertical bar  \|
%%   Right brace   \}     Tilde         \~}
\ProvidesFile{dt-russian.def}[2004/10/31]
\providecommand{\monthnamerussian}[1][\month]{%
\@orgargctr=#1\relax
\ifcase\@orgargctr
\PackageError{datetime}{Invalid Month number \the\@orgargctr}{%
Month numbers should go from 1 to 12}%
\or \cyrya\cyrn\cyrv\cyra\cyrr\cyrya\or
    \cyrf\cyre\cyrv\cyrr\cyra\cyrl\cyrya\or
    \cyrm\cyra\cyrr\cyrt\cyra\or
    \cyra\cyrp\cyrr\cyre\cyrl\cyrya\or
    \cyrm\cyra\cyrya\or
    \cyri\cyryu\cyrn\cyrya\or
    \cyri\cyryu\cyrl\cyrya\or
    \cyra\cyrv\cyrg\cyru\cyrs\cyrt\cyra\or
    \cyrs\cyre\cyrn\cyrt\cyrya\cyrb\cyrr\cyrya\or
    \cyro\cyrk\cyrt\cyrya\cyrb\cyrr\cyrya\or
    \cyrn\cyro\cyrya\cyrb\cyrr\cyrya\or
    \cyrd\cyre\cyrk\cyra\cyrb\cyrr\cyrya%
\else
\PackageError{datetime}{Invalid Month number \the\@orgargctr}{%
Month numbers should go from 1 to 12}%
\fi}
\DeclareRobustCommand*\daterussian{%
\renewcommand{\formatdate}[3]{%
\@day=##1\relax\@month=##2\relax\@year=##3\relax
\number\@day~\monthnamerussian[\@month]\ \number\@year~\cyrg.}}
\endinput
%%
%% End of file `dt-russian.def'.
}
\@ifundefined{datesamin}{}{\input{dt-samin.def}}
\@ifundefined{datescottish}{}{\input{dt-scottish.def}}
\@ifundefined{dateserbian}{}{\input{dt-serbian.def}}
\@ifundefined{dateslovak}{}{\input{dt-slovak.def}}
\@ifundefined{dateslovene}{}{\input{dt-slovene.def}}
\@ifundefined{datespanish}{}{\input{dt-spanish.def}}
\@ifundefined{dateswedish}{}{\input{dt-swedish.def}}
\@ifundefined{dateturkish}{}{\input{dt-turkish.def}}
\@ifundefined{dateukraineb}{}{\input{dt-ukraineb.def}}
\@ifundefined{dateusorbian}{}{\input{dt-usorbian.def}}
\@ifundefined{datewelsh}{}{\input{dt-welsh.def}}
\fi
%    \end{macrocode}
%
% Define |\newdateformat| which defines a declaration that redefines |\formatdate| so that
% it uses |\dateformat|.  |\dateformat| takes four arguments, the first defines the format
% the last three arguments are the three arguments that effectively get passed to |\formatdate|.
% |\dateformat| sets |\@day|, |\@month| and |\@year|.  |\c@DAY|, |\c@MONTH| and |\c@YEAR| 
% are set as synonims for |\@day|, |\@month| and |\@year|, to that the uses can use the counters
% |DAY|, |MONTH| and |YEAR| as part of the format.
%    \begin{macrocode}
\if@dt@nodate
\PackageInfo{datetime}{option "nodate" used, so not 
defining \string\newdateformat}
\else
%    \end{macrocode}
% The commands |\THEDAY|, |\THEMONTH| and |\THEYEAR| should only be
% used in the argument to |\newdateformat|. This is done partly to
% assist the \LaTeX2HTML version.
%    \begin{macrocode}
\providecommand*\THEDAY{\the\@day}
\providecommand*\THEMONTH{\the\@month}
\providecommand*\THEYEAR{\the\@year}
%    \end{macrocode}
% Simulate a \LaTeX\ counter.
%    \begin{macrocode}
\let\c@DAY=\@day
\let\c@MONTH=\@month
\let\c@YEAR=\@year
%    \end{macrocode}
%\begin{macro}{\dateformat}
% Format the given date in the given format.
%    \begin{macrocode}
\providecommand*{\dateformat}[4]{%
\@day=#2\relax\@month=#3\relax\@year=#4\relax#1}
%    \end{macrocode}
%\end{macro}
%\begin{macro}{\newdateformat}
% Provide a means to define a new date format.
%\changes{2.2}{2004/04/27}{new}
%    \begin{macrocode}
\providecommand{\newdateformat}[2]{%
\@ifundefined{#1}{%
\expandafter\DeclareRobustCommand\csname#1\endcsname{%
\renewcommand{\formatdate}{\dateformat{#2}}}}{%
\PackageError{datetime}{Can't create new date format, command 
\textbackslash#1 already defined}{You will need to 
give your new date format a different name}}}
%    \end{macrocode}
%\end{macro}
% End of |\if@no@date| else part:
%    \begin{macrocode}
\fi
%    \end{macrocode}
%
% \subsubsection{Time Macros}
% Define a command to create a new time format, similar to the new 
% date format. Again this is done in a way that makes it easier to
% code the \LaTeX2HTML version.
%    \begin{macrocode}
\newcount\c@HOUR
\newcount\c@HOURXII
\newcount\c@MINUTE
\newcount\c@TOHOUR
\newcount\c@TOMINUTE
%    \end{macrocode}
% These commands should only be used in |\newtimeformat|.
%    \begin{macrocode}
\def\THEHOUR{\the\c@HOUR}
\def\THEHOURXII{\the\c@HOURXII}
\def\THEMINUTE{\the\c@MINUTE}
\def\THETOHOUR{\the\c@TOHOUR}
\def\THETOMINUTE{\the\c@TOMINUTE}
%    \end{macrocode}
%\begin{macro}{\newtimeformat}
% Provide a means to define a new time format.
%\changes{2.3}{2004/05/01}{new}
%    \begin{macrocode}
\providecommand*{\newtimeformat}[2]{%
\@ifundefined{#1}{%
\expandafter\def\csname#1\endcsname{%
\c@HOUR=\time%
\divide\c@HOUR by 60\relax
\c@HOURXII=\c@HOUR
\ifnum\c@HOURXII>12
\advance\c@HOURXII by -12\relax
\fi
\c@MINUTE=\time%
\@modulo{\c@MINUTE}{60}%
\c@TOHOUR=\c@HOURXII
\advance\c@TOHOUR by 1\relax
\@modulo{\c@TOHOUR}{12}%
\c@TOMINUTE=\c@MINUTE
\advance\c@TOMINUTE by -60\relax
\multiply\c@TOMINUTE by -1\relax
#2\relax
}}{%
\PackageError{datetime}{Command \textbackslash#1  already defined}{%
You can't create a new time format called "#1" as the command 
\textbackslash#1 already exists}}}
%    \end{macrocode}
%\end{macro}
%\begin{macro}{\xxivtime}
% Define commands to print the current time. Twenty-four hour time:
%    \begin{macrocode}
\newtimeformat{xxivtime}{%
\twodigit\THEHOUR\timeseparator\twodigit\THEMINUTE}
%    \end{macrocode}
%\end{macro}
%\begin{macro}{\ampmtime}
% 12-hour time:
%\changes{2.3}{2004/05/01}{fixed minor bug}
%\changes{2.51}{2007/01/30}{fixed bug between midnight and 1am}
%    \begin{macrocode}
\newtimeformat{ampmtime}{%
\ifthenelse{\value{HOUR}=0}{12}{\THEHOURXII}\timeseparator
\twodigit\THEMINUTE
\ifthenelse{\value{HOUR}<12}{\amname}{%
\ifthenelse{\time=720}{ \noon}{\pmname}}}
%    \end{macrocode}
%\end{macro}
% Textual time:
%\begin{macro}{\hourstring}
% \cs{hourstring}\marg{count} will print |\midnight| if 
% \meta{count} is 0, otherwise will do \cs{Numberstring}\marg{count}.
%    \begin{macrocode}
\newcommand*{\hourstring}[1]{%
\ifthenelse{\value{#1}=0}{\midnight}{\Numberstring{#1}}}
%    \end{macrocode}
%\end{macro}
%\begin{macro}{\oclock}
%\changes{2.43}{2005/02/23}{fixed bug causing an infinite loop on the hour}
%    \begin{macrocode}
\newtimeformat{oclock}{\ifthenelse{\time=0 \or \time=720}{%
%    \end{macrocode}
% Midnight or Midday:
%    \begin{macrocode}
\ifthenelse{\time=0}{\midnight}{\noon}}{%
%    \end{macrocode}
% Neither Midnight nor Midday.
% Do minutes first
%    \begin{macrocode}
\ifthenelse{\value{MINUTE}=0}{%
%    \end{macrocode}
% On the hour: don't print any minutes, just do the hour and 
% |\oclockstring|
%    \begin{macrocode}
\Numberstring{HOUR} \oclockstring}{%
\ifthenelse{\value{MINUTE}=15}{%
\quarterpast\ \hourstring{HOUR}}{%
\ifthenelse{\value{MINUTE}=30}{%
\halfpast\ \hourstring{HOUR}}{%
\ifthenelse{\value{MINUTE}=45}{%
\quarterto\ \hourstring{TOHOUR}}{%
\ifthenelse{\value{MINUTE}<30}{%
\Numberstring{MINUTE}\ \ifthenelse{\value{MINUTE}=1}{minute}{minutes} 
past \hourstring{HOURXII}}{%
\Numberstring{TOMINUTE}\ \ifthenelse{%
\value{TOMINUTE}=1}{minute}{minutes} to \hourstring{TOHOUR}}}}}}% 
%    \end{macrocode}
% Now say whether it is morning or afternoon
%    \begin{macrocode}
\ifthenelse{\value{HOUR}<12}{%
%    \end{macrocode}
% Morning
%    \begin{macrocode}
\ifthenelse{\value{HOUR}=0}{}{\ \amstring}}{%
%    \end{macrocode}
% Afternoon
%    \begin{macrocode}
\ifthenelse{\value{TOHOUR}=0}{}{\ \pmstring}}}}
%    \end{macrocode}
%\end{macro}
% Define textual strings used in the above.
%\begin{macro}{\amname}
%    \begin{macrocode}
\providecommand*{\amname}{am}
%    \end{macrocode}
%\end{macro}
%\begin{macro}{\pmname}
%    \begin{macrocode}
\providecommand*{\pmname}{pm}
%    \end{macrocode}
%\end{macro}
%\begin{macro}{\amorpmname}
%    \begin{macrocode}
\providecommand*{\amorpmname}{%
\ifthenelse{\value{HOUR}>12}{\pmname}{\amname}}
%    \end{macrocode}
%\end{macro}
%\begin{macro}{\amstring}
%    \begin{macrocode}
\providecommand*{\amstring}{in the morning}
%    \end{macrocode}
%\end{macro}
%\begin{macro}{\pmstring}
%    \begin{macrocode}
\providecommand*{\pmstring}{in the afternoon}
%    \end{macrocode}
%\end{macro}
%\begin{macro}{\amorpmstring}
%    \begin{macrocode}
\providecommand*{\amorpmstring}{%
\ifthenelse{\value{HOUR}>12}{\pmstring}{\amstring}}
%    \end{macrocode}
%\end{macro}
%\begin{macro}{\halfpast}
%    \begin{macrocode}
\providecommand*{\halfpast}{Half past}
%    \end{macrocode}
%\end{macro}
%\begin{macro}{\quarterpast}
%    \begin{macrocode}
\providecommand*{\quarterpast}{Quarter past}
%    \end{macrocode}
%\end{macro}
%\begin{macro}{\quarterto}
%    \begin{macrocode}
\providecommand*{\quarterto}{Quarter to}
%    \end{macrocode}
%\end{macro}
%\begin{macro}{\noon}
%    \begin{macrocode}
\providecommand*{\noon}{Noon}
%    \end{macrocode}
%\end{macro}
%\begin{macro}{\midnight}
%    \begin{macrocode}
\providecommand*{\midnight}{Midnight}
%    \end{macrocode}
%\end{macro}
%\begin{macro}{\oclockstring}
%    \begin{macrocode}
\providecommand*{\oclockstring}{O'Clock}
%    \end{macrocode}
%\end{macro}
%\begin{macro}{\pdfdate}
% Finally create command that will typeset the date in PDF format
% e.g. 20040501215500. This is defined regardless of |\if@no@date|
% as it's provided for use in |\pdfinfo|. Initially |\pdfdate| is set 
% to the year.
%\changes{2.31}{2004/05/01}{new}
%\changes{2.32}{2004/05/04}{fixed bug}
%    \begin{macrocode}
\newtoks\dt@a \newtoks\dt@b
\edef\pdfdate{\the\year}
%    \end{macrocode}
% Append the month
%    \begin{macrocode}
\dt@b=\expandafter{\pdfdate}
\dt@a=\expandafter{\the\month}
\ifnum\month<10\relax
\edef\pdfdate{\the\dt@b0\the\dt@a}
\else
\edef\pdfdate{\the\dt@b\the\dt@a}
\fi
%    \end{macrocode}
% Append the day
%    \begin{macrocode}
\dt@b=\expandafter{\pdfdate}
\dt@a=\expandafter{\the\day}
\ifnum\day<10\relax
\edef\pdfdate{\the\dt@b0\the\dt@a}
\else
\edef\pdfdate{\the\dt@b\the\dt@a}
\fi
%    \end{macrocode}
% Append the hour
%    \begin{macrocode}
\@dtctr=\time%
\divide\@dtctr by 60\relax
\dt@b=\expandafter{\pdfdate}
\dt@a=\expandafter{\the\@dtctr}
\ifnum\@dtctr<10
\edef\pdfdate{\the\dt@b0\the\dt@a}
\else
\edef\pdfdate{\the\dt@b\the\dt@a}
\fi
%    \end{macrocode}
% Append the minute.
%    \begin{macrocode}
\@dtctr=\time%
\@modulo{\@dtctr}{60}%
\dt@b=\expandafter{\pdfdate}
\dt@a=\expandafter{\the\@dtctr}
\ifnum\@dtctr<10\relax
\edef\pdfdate{\the\dt@b0\the\dt@a}
\else
\edef\pdfdate{\the\dt@b\the\dt@a}
\fi
%    \end{macrocode}
% Append the seconds. \TeX\ doesn't store the seconds, so
% set to zero.
%    \begin{macrocode}
\dt@a={00}
\dt@b=\expandafter{\pdfdate}
\edef\pdfdate{\the\dt@b\the\dt@a}
%    \end{macrocode}
%\end{macro}
%\iffalse
%    \begin{macrocode}
%</datetime.sty>
%    \end{macrocode}
%\fi
%\iffalse
%    \begin{macrocode}
%<*dt-austrian.def>
%    \end{macrocode}
%\fi
%\subsection{Compatibility with Babel (Language definition files)}
%\subsubsection{Austrian}
% Identify file
%    \begin{macrocode}
\ProvidesFile{dt-austrian.def}[2004/10/31]
%    \end{macrocode}
% Define month names.
%    \begin{macrocode}
\providecommand{\monthnameaustrian}[1][\month]{%
\@orgargctr=#1\relax
\ifcase\@orgargctr
\PackageError{datetime}{Invalid Month number \the\@orgargctr}{%
Month numbers should go from 1 to 12}%
\or J\"anner%
\or Februar%
\or M\"arz%
\or April%
\or Mai%
\or Juni%
\or Juli%
\or August%
\or September%
\or Oktober%
\or November%
\or Dezember%
\else
\PackageError{datetime}{Invalid Month number \the\@orgargctr}{%
Month numbers should go from 1 (janvier) to 12 (decembre)}%
\fi}
%    \end{macrocode}
% (Re)Define declaration to switch to this format.
%    \begin{macrocode}
\DeclareRobustCommand*\dateaustrian{%
\renewcommand{\formatdate}[3]{%
\@day=##1\relax\@month=##2\relax\@year=##3\relax
\number\@day.~\monthnameaustrian[\@month]\space\number\@year}}
%    \end{macrocode}
%\iffalse
%    \begin{macrocode}
%</dt-austrian.def>
%    \end{macrocode}
%\fi
%\iffalse
%    \begin{macrocode}
%<*dt-bahasa.def>
%    \end{macrocode}
%\fi
%\subsubsection{Bahasa}
% Identify file
%    \begin{macrocode}
\ProvidesFile{dt-bahasa.def}[2004/10/31]
%    \end{macrocode}
% Define month names.
%    \begin{macrocode}
\providecommand{\monthnamebahasa}[1][\month]{%
\@orgargctr=#1\relax
\ifcase\@orgargctr
\PackageError{datetime}{Invalid Month number \the\@orgargctr}{%
Month numbers should go from 1 (Januari) to 12 (Desember)}%
\or Januari%
\or Pebruari%
\or Maret%
\or April%
\or Mei%
\or Juni%
\or Juli%
\or Agustus%
\or September%
\or Oktober%
\or Nopember%
\or Desember%
\else
\PackageError{datetime}{Invalid Month number \the\@orgargctr}{%
Month numbers should go from 1 (Januari) to 12 (Desember)}%
\fi}
%    \end{macrocode}
% Define abbreviated month names. This currently does the full 
% name, because I don't know the abbreviated versions.
%    \begin{macrocode}
\providecommand{\shortmonthnamebahasa}[1][\month]{%
\@orgargctr=#1\relax
\ifcase\@orgargctr
\PackageError{datetime}{Invalid Month number \the\@orgargctr}{%
Month numbers should go from 1 (Januari) to 12 (Desember)}%
\or Januari%
\or Pebruari%
\or Maret%
\or April%
\or Mei%
\or Juni%
\or Juli%
\or Agustus%
\or September%
\or Oktober%
\or Nopember%
\or Desember%
\else
\PackageError{datetime}{Invalid Month number \the\@orgargctr}{%
Month numbers should go from 1 (Januari) to 12 (Desember)}%
\fi}
%    \end{macrocode}
% (Re)Define declaration to switch to this format.
%    \begin{macrocode}
\DeclareRobustCommand*\datebahasa{%
\renewcommand{\formatdate}[3]{%
\@day=##1\relax\@month=##2\relax\@year=##3\relax
\number\@day}~\monthnamebahasa[\@month]\space \number\@year}}
%    \end{macrocode}
%\iffalse
%    \begin{macrocode}
%</dt-bahasa.def>
%    \end{macrocode}
%\fi
%\iffalse
%    \begin{macrocode}
%<*dt-basque.def>
%    \end{macrocode}
%\fi
%\subsubsection{Basque}
% Identify file
%    \begin{macrocode}
\ProvidesFile{dt-basque.def}[2004/10/31]
%    \end{macrocode}
% Define month names.
%    \begin{macrocode}
\providecommand{\monthnamebasque}[1][\month]{%
\@orgargctr=#1\relax
\ifcase\@orgargctr
\PackageError{datetime}{Invalid Month number \the\@orgargctr}{%
Month numbers should go from 1 (urtarrilaren) to 12 (abenduaren)}%
\or urtarrilaren%
\or otsailaren%
\or martxoaren%
\or apirilaren%
\or maiatzaren%
\or ekainaren%
\or uztailaren%
\or abuztuaren%
\or irailaren%
\or urriaren%
\or azaroaren%
\or abenduaren%
\else \PackageError{datetime}{Invalid Month number \the\@orgargctr}{%
Month numbers should go from 1 (urtarrilaren) to 12 (abenduaren)}%
\fi}
%    \end{macrocode}
% Define abbreviated month names. This currently does the full 
% name, because I don't know the abbreviated versions.
%    \begin{macrocode}
\providecommand{\shortmonthnamebasque}[1][\month]{%
\@orgargctr=#1\relax
\ifcase\@orgargctr
\PackageError{datetime}{Invalid Month number \the\@orgargctr}{%
Month numbers should go from 1 (urtarrilaren) to 12 (abenduaren)}%
\or urtarrilaren%
\or otsailaren%
\or martxoaren%
\or apirilaren%
\or maiatzaren%
\or ekainaren%
\or uztailaren%
\or abuztuaren%
\or irailaren%
\or urriaren%
\or azaroaren%
\or abenduaren%
\else
\PackageError{datetime}{Invalid Month number \the\@orgargctr}{%
Month numbers should go from 1 (urtarrilaren) to 12 (abenduaren)}%
\fi}
%    \end{macrocode}
% (Re)Define declaration to switch to this format.
%    \begin{macrocode}
\DeclareRobustCommand*\datebasque{%
\renewcommand{\formatdate}[3]{%
\@day=##1\relax\@month=##2\relax\@year=##3\relax
\number\@year.eko\space\monthnamebasque[\@month]~\number\@day}}
%    \end{macrocode}
%\iffalse
%    \begin{macrocode}
%</dt-basque.def>
%    \end{macrocode}
%\fi
%\iffalse
%    \begin{macrocode}
%<*dt-breton.def>
%    \end{macrocode}
%\fi
%\subsubsection{Breton}
% Identify file
%    \begin{macrocode}
\ProvidesFile{dt-breton.def}[2004/10/31]
%    \end{macrocode}
% Define month names.
%    \begin{macrocode}
\providecommand{\monthnamebreton}[1][\month]{%
\@orgargctr=#1\relax
\ifcase\@orgargctr
\PackageError{datetime}{Invalid Month number \the\@orgargctr}{%
Month numbers should go from 1 (Genver) to 12 (Kerzu)}%
\or Genver%
\or C'hwevrer%
\or Meurzh%
\or Ebrel%
\or Mae%
\or Mezheven%
\or Gouere%
\or Eost%
\or Gwengolo%
\or Here%
\or Du%
\or Kerzu%
\else
\PackageError{datetime}{Invalid Month number \the\@orgargctr}{%
Month numbers should go from 1 (Genver) to 12 (Kerzu)}%
\fi}
%    \end{macrocode}
% Define abbreviated month names. This currently does the full 
% name, because I don't know the abbreviated versions.
%    \begin{macrocode}
\providecommand{\shortmonthnamebreton}[1][\month]{%
\@orgargctr=#1\relax
\ifcase\@orgargctr
\PackageError{datetime}{Invalid Month number \the\@orgargctr}{%
Month numbers should go from 1 (Genver) to 12 (Kerzu)}%
\or Genver%
\or C'hwevrer%
\or Meurzh%
\or Ebrel%
\or Mae%
\or Mezheven%
\or Gouere%
\or Eost%
\or Gwengolo%
\or Here%
\or Du%
\or Kerzu%
\else
\PackageError{datetime}{Invalid Month number \the\@orgargctr}{%
Month numbers should go from 1 (Genver) to 12 (Kerzu)}%
\fi}
%    \end{macrocode}
% (Re)Define declaration to switch to this format.
%    \begin{macrocode}
\DeclareRobustCommand*\datebreton{%
\renewcommand{\formatdate}[3]{%
\@day=##1\relax\@month=##2\relax\@year=##3\relax
\ifnum\@day=1\relax 1\/\textsuperscript{a\~n}\else\number\@day\fi
\space a\space viz\space\monthnamebreton[\@month]\space
\number\@year
}}
%    \end{macrocode}
%\iffalse
%    \begin{macrocode}
%</dt-breton.def>
%    \end{macrocode}
%\fi
%\iffalse
%    \begin{macrocode}
%<*dt-british.def>
%    \end{macrocode}
%\fi
%\subsubsection{british}
% Identify file
%    \begin{macrocode}
\ProvidesFile{dt-british.def}[2007/06/15]
%    \end{macrocode}
\let\datebritish\longdate

\let\monthnamebritish\monthnameenglish
\let\shortmonthnamebritish\shortmonthnameenglish

\let\dayofweeknameidbritish\dayofweeknameidenglish
\let\shortdayofweeknameidbritish\shortdayofweeknameidenglish
%\iffalse
%    \begin{macrocode}
%</dt-british.def>
%    \end{macrocode}
%\fi
%\iffalse
%    \begin{macrocode}
%<*dt-bulgarian.def>
%    \end{macrocode}
%\fi
%\subsubsection{Bulgarian}
% Identify file
%    \begin{macrocode}
\ProvidesFile{dt-bulgarian.def}[2004/10/31]
%    \end{macrocode}
% Define month names.
%    \begin{macrocode}
\providecommand{\monthnamebulgarian}[1][\month]{%
\@orgargctr=#1\relax
\ifcase\@orgargctr
\PackageError{datetime}{Invalid Month number \the\@orgargctr}{%
Month numbers should go from 1 to 12}%
    \or\cyrya\cyrn\cyru\cyra\cyrr\cyri\or
    \cyrf\cyre\cyrv\cyrr\cyru\cyra\cyrr\cyri\or
    \cyrm\cyra\cyrr\cyrt\or
    \cyra\cyrp\cyrr\cyri\cyrl\or
    \cyrm\cyra\cyrishrt\or
    \cyryu\cyrn\cyri\or
    \cyryu\cyrl\cyri\or
    \cyra\cyrv\cyrg\cyru\cyrs\cyrt\or
    \cyrs\cyre\cyrp\cyrt\cyre\cyrm\cyrv\cyrr\cyri\or
    \cyro\cyrk\cyrt\cyro\cyrm\cyrv\cyrr\cyri\or
    \cyrn\cyro\cyre\cyrm\cyrv\cyrr\cyri\or
    \cyrd\cyre\cyrk\cyre\cyrm\cyrv\cyrr\cyri
\else
\PackageError{datetime}{Invalid Month number \the\@orgargctr}{%
Month numbers should go from 1 to 12}%
\fi}
%    \end{macrocode}
% Define abbreviated month names. This currently does the full 
% name, because I don't know the abbreviated versions.
%    \begin{macrocode}
\providecommand{\shortmonthnamebulgarian}[1][\month]{%
\@orgargctr=#1\relax
\ifcase\@orgargctr
\PackageError{datetime}{Invalid Month number \the\@orgargctr}{%
Month numbers should go from 1 to 12}%
    \or\cyrya\cyrn\cyru\cyra\cyrr\cyri\or
    \cyrf\cyre\cyrv\cyrr\cyru\cyra\cyrr\cyri\or
    \cyrm\cyra\cyrr\cyrt\or
    \cyra\cyrp\cyrr\cyri\cyrl\or
    \cyrm\cyra\cyrishrt\or
    \cyryu\cyrn\cyri\or
    \cyryu\cyrl\cyri\or
    \cyra\cyrv\cyrg\cyru\cyrs\cyrt\or
    \cyrs\cyre\cyrp\cyrt\cyre\cyrm\cyrv\cyrr\cyri\or
    \cyro\cyrk\cyrt\cyro\cyrm\cyrv\cyrr\cyri\or
    \cyrn\cyro\cyre\cyrm\cyrv\cyrr\cyri\or
    \cyrd\cyre\cyrk\cyre\cyrm\cyrv\cyrr\cyri
\else
\PackageError{datetime}{Invalid Month number \the\@orgargctr}{%
Month numbers should go from 1 to 12}%
\fi}
%    \end{macrocode}
% (Re)Define declaration to switch to this format.
%    \begin{macrocode}
\DeclareRobustCommand*\datebulgarian{%
\renewcommand{\formatdate}[3]{%
\@day=##1\relax\@month=##2\relax\@year=##3\relax
\number\@day~\monthnamebulgarian[\@month]\ \number\@year~\cyrg.}}
%    \end{macrocode}
%\iffalse
%    \begin{macrocode}
%</dt-bulgarian.def>
%    \end{macrocode}
%\fi
%\iffalse
%    \begin{macrocode}
%<*dt-catalan.def>
%    \end{macrocode}
%\fi
%\subsubsection{Catalan}
% Identify file
%    \begin{macrocode}
\ProvidesFile{dt-catalan.def}[2004/10/31]
%    \end{macrocode}
% Define month names.
%    \begin{macrocode}
\providecommand{\monthnamecatalan}[1][\month]{%
\@orgargctr=#1\relax
\ifcase\@orgargctr
\PackageError{datetime}{Invalid Month number \the\@orgargctr}{%
Month numbers should go from 1 (de gener) to 12 (de desembre)}%
\or de gener%
\or de febrer%
\or de mar\c{c}%
\or d'abril%
\or de maig%
\or de juny%
\or de juliol%
\or d'agost%
\or de setembre%
\or d'octubre%
\or de novembre%
\or de desembre%
\else
\PackageError{datetime}{Invalid Month number \the\@orgargctr}{%
Month numbers should go from 1 (de gener) to 12 (de desembre)}%
\fi}
%    \end{macrocode}
% Define abbreviated month names. This currently does the full 
% name, because I don't know the abbreviated versions.
%    \begin{macrocode}
\providecommand{\shortmonthnamecatalan}[1][\month]{%
\@orgargctr=#1\relax
\ifcase\@orgargctr
\PackageError{datetime}{Invalid Month number \the\@orgargctr}{%
Month numbers should go from 1 (de gener) to 12 (de desembre)}%
\or de gener%
\or de febrer%
\or de mar\c{c}%
\or d'abril%
\or de maig%
\or de juny%
\or de juliol%
\or d'agost%
\or de setembre%
\or d'octubre%
\or de novembre%
\or de desembre%
\else
\PackageError{datetime}{Invalid Month number \the\@orgargctr}{%
Month numbers should go from 1 (de gener) to 12 (de desembre)}%
\fi}
%    \end{macrocode}
% (Re)Define declaration to switch to this format.
%    \begin{macrocode}
\DeclareRobustCommand*\datecatalan{%
\renewcommand{\formatdate}[3]{%
\@day=##1\relax\@month=##2\relax\@year=##3\relax
\number\@day~\monthnamecatalan[\@month]\ de~\number\@year
}}
%    \end{macrocode}
%\iffalse
%    \begin{macrocode}
%</dt-catalan.def>
%    \end{macrocode}
%\fi
%\iffalse
%    \begin{macrocode}
%<*dt-croatian.def>
%    \end{macrocode}
%\fi
%\subsubsection{Croatian}
% Identify file
%    \begin{macrocode}
\ProvidesFile{dt-croatian.def}[2004/10/31]
%    \end{macrocode}
% Define month names.
%    \begin{macrocode}
\providecommand{\monthnamecroatian}[1][\month]{%
\@orgargctr=#1\relax
\ifcase\@orgargctr
\PackageError{datetime}{Invalid Month number \the\@orgargctr}{%
Month numbers should go from 1 (sijecnja) to 12 (prosinca)}%
\or sije\v{c}nja%
\or velja\v{c}e%
\or o\v{z}ujka%
\or travnja%
\or svibnja%
\or lipnja%
\or srpnja%
\or kolovoza%
\or rujna%
\or listopada%
\or studenog%
\or prosinca%
\else
\PackageError{datetime}{Invalid Month number \the\@orgargctr}{%
Month numbers should go from 1 (sijecnja) to 12 (prosinca)}%
\fi}
%    \end{macrocode}
% Define abbreviated month names. This currently does the full 
% name, because I don't know the abbreviated versions.
%    \begin{macrocode}
\providecommand{\shortmonthnamecroatian}[1][\month]{%
\@orgargctr=#1\relax
\ifcase\@orgargctr
\PackageError{datetime}{Invalid Month number \the\@orgargctr}{%
Month numbers should go from 1 (sijecnja) to 12 (prosinca)}%
\or sije\v{c}nja%
\or velja\v{c}e%
\or o\v{z}ujka%
\or travnja%
\or svibnja%
\or lipnja%
\or srpnja%
\or kolovoza%
\or rujna%
\or listopada%
\or studenog%
\or prosinca%
\else
\PackageError{datetime}{Invalid Month number \the\@orgargctr}{%
Month numbers should go from 1 (sijecnja) to 12 (prosinca)}%
\fi}
%    \end{macrocode}
% (Re)Define declaration to switch to this format.
%    \begin{macrocode}
\DeclareRobustCommand*\datecroatian{%
\renewcommand{\formatdate}[3]{%
\@day=##1\relax\@month=##2\relax\@year=##3\relax
\number\@day.~\monthnamecroatian[\@month]\space \number\@year.}}
%    \end{macrocode}
%\iffalse
%    \begin{macrocode}
%</dt-croatian.def>
%    \end{macrocode}
%\fi
%\iffalse
%    \begin{macrocode}
%<*dt-czech.def>
%    \end{macrocode}
%\fi
%\subsubsection{Czech}
% Identify file
%    \begin{macrocode}
\ProvidesFile{dt-czech.def}[2004/10/31]
%    \end{macrocode}
% Define month names.
%    \begin{macrocode}
\providecommand{\monthnameczech}[1][\month]{%
\@orgargctr=#1\relax
\ifcase\@orgargctr
\PackageError{datetime}{Invalid Month number \the\@orgargctr}{%
Month numbers should go from 1 to 12}%
\or ledna%
\or \'unora%
\or b\v{r}ezna%
\or dubna%
\or kv\v{e}tna%
\or \v{c}ervna%
\or \v{c}ervence%
\or srpna%
\or z\'a\v{r}\'{\i}%
\or \v{r}\'{\i}jna%
\or listopadu%
\or prosince%
\else
\PackageError{datetime}{Invalid Month number \the\@orgargctr}{%
Month numbers should go from 1 to 12}%
\fi}
%    \end{macrocode}
% Define abbreviated month names. This currently does the full 
% name, because I don't know the abbreviated versions.
%    \begin{macrocode}
\providecommand{\shortmonthnameczech}[1][\month]{%
\@orgargctr=#1\relax
\ifcase\@orgargctr
\PackageError{datetime}{Invalid Month number \the\@orgargctr}{%
Month numbers should go from 1 to 12}%
\or ledna%
\or \'unora%
\or b\v{r}ezna%
\or dubna%
\or kv\v{e}tna%
\or \v{c}ervna%
\or \v{c}ervence%
\or srpna%
\or z\'a\v{r}\'{\i}%
\or \v{r}\'{\i}jna%
\or listopadu%
\or prosince%
\else
\PackageError{datetime}{Invalid Month number \the\@orgargctr}{%
Month numbers should go from 1 to 12}%
\fi}
%    \end{macrocode}
% (Re)Define declaration to switch to this format.
%    \begin{macrocode}
\DeclareRobustCommand*\dateczech{%
\renewcommand{\formatdate}[3]{%
\@day=##1\relax\@month=##2\relax\@year=##3\relax
\number\@day.~\monthnameczech[\@month]\space \number\@year}}
%    \end{macrocode}
%\iffalse
%    \begin{macrocode}
%</dt-czech.def>
%    \end{macrocode}
%\fi
%\iffalse
%    \begin{macrocode}
%<*dt-danish.def>
%    \end{macrocode}
%\fi
%\subsubsection{Danish}
% Identify file
%    \begin{macrocode}
\ProvidesFile{dt-danish.def}[2004/10/31]
%    \end{macrocode}
% Define month names.
%    \begin{macrocode}
\providecommand{\monthnamedanish}[1][\month]{%
\@orgargctr=#1\relax
\ifcase\@orgargctr
\PackageError{datetime}{Invalid Month number \the\@orgargctr}{%
Month numbers should go from 1 to 12}%
\or januar%
\or februar%
\or marts%
\or april%
\or maj%
\or juni%
\or juli%
\or august%
\or september%
\or oktober%
\or november%
\or december%
\else
\PackageError{datetime}{Invalid Month number \the\@orgargctr}{%
Month numbers should go from 1 to 12}%
\fi}
%    \end{macrocode}
% (Re)Define declaration to switch to this format.
%    \begin{macrocode}
\DeclareRobustCommand*\datedanish{%
\renewcommand{\formatdate}[3]{%
\@day=##1\relax\@month=##2\relax\@year=##3\relax
\number\@day.~\monthnamedanish[\@month]\space \number\@year}}
%    \end{macrocode}
%\iffalse
%    \begin{macrocode}
%</dt-danish.def>
%    \end{macrocode}
%\fi
%\iffalse
%    \begin{macrocode}
%<*dt-dutch.def>
%    \end{macrocode}
%\fi
%\subsubsection{Dutch}
% Identify file
%    \begin{macrocode}
\ProvidesFile{dt-dutch.def}[2004/10/31]
%    \end{macrocode}
% Define month names.
%    \begin{macrocode}
\providecommand{\monthnamedutch}[1][\month]{%
\@orgargctr=#1\relax
\ifcase\@orgargctr
\PackageError{datetime}{Invalid Month number \the\@orgargctr}{%
Month numbers should go from 1 to 12}%
\or januari%
\or februari%
\or maart%
\or april%
\or mei%
\or juni%
\or juli%
\or augustus%
\or september%
\or oktober%
\or november%
\or december%
\else
\PackageError{datetime}{Invalid Month number \the\@orgargctr}{%
Month numbers should go from 1 to 12}%
\fi}
%    \end{macrocode}
% (Re)Define declaration to switch to this format.
%    \begin{macrocode}
\DeclareRobustCommand*\datedutch{%
\renewcommand{\formatdate}[3]{%
\@day=##1\relax\@month=##2\relax\@year=##3\relax
\number\@day~\monthnamedutch[\@month]\space \number\@year}}
%    \end{macrocode}
%\iffalse
%    \begin{macrocode}
%</dt-dutch.def>
%    \end{macrocode}
%\fi
%\iffalse
%    \begin{macrocode}
%<*dt-esperanto.def>
%    \end{macrocode}
%\fi
%\subsubsection{Esperanto}
% Identify file
%    \begin{macrocode}
\ProvidesFile{dt-esperanto.def}[2004/10/31]
%    \end{macrocode}
% Define month names.
%    \begin{macrocode}
\providecommand{\monthnameesperanto}[1][\month]{%
\@orgargctr=#1\relax
\ifcase\@orgargctr
\PackageError{datetime}{Invalid Month number \the\@orgargctr}{%
Month numbers should go from 1 to 12}%
\or januaro%
\or februaro%
\or marto%
\or aprilo%
\or majo%
\or junio%
\or julio%
\or a\u{u}gusto%
\or septembro%
\or oktobro%
\or novembro%
\or decembro%
\else
\PackageError{datetime}{Invalid Month number \the\@orgargctr}{%
Month numbers should go from 1 to 12}%
\fi}
%    \end{macrocode}
% (Re)Define declaration to switch to this format.
%    \begin{macrocode}
\DeclareRobustCommand*\dateesperanto{%
\renewcommand{\formatdate}[3]{%
\@day=##1\relax\@month=##2\relax\@year=##3\relax
\number\@day{--a}~de~\monthnameesperanto[\@month],\space 
\number\@year}}
%    \end{macrocode}
%\iffalse
%    \begin{macrocode}
%</dt-esperanto.def>
%    \end{macrocode}
%\fi
%\iffalse
%    \begin{macrocode}
%<*dt-estonian.def>
%    \end{macrocode}
%\fi
%\subsubsection{Estonian}
% Identify file
%    \begin{macrocode}
\ProvidesFile{dt-estonian.def}[2004/10/31]
%    \end{macrocode}
% Define month names.
%    \begin{macrocode}
\providecommand{\monthnameestonian}[1][\month]{%
\@orgargctr=#1\relax
\ifcase\@orgargctr
\PackageError{datetime}{Invalid Month number \the\@orgargctr}{%
Month numbers should go from 1 to 12}%
\or jaanuar%
\or veebruar%
\or m"arts%
\or aprill%
\or mai%
\or juuni%
\or juuli%
\or august%
\or september%
\or oktoober%
\or november%
\or detsember%
\else
\PackageError{datetime}{Invalid Month number \the\@orgargctr}{%
Month numbers should go from 1 to 12}%
\fi}
%    \end{macrocode}
% (Re)Define declaration to switch to this format.
%    \begin{macrocode}
\DeclareRobustCommand*\dateestonian{%
\renewcommand{\formatdate}[3]{%
\@day=##1\relax\@month=##2\relax\@year=##3\relax
\number\@day.\space\monthnameestonian[\@month]\space 
\number\@year.\space a.}}
%    \end{macrocode}
%\iffalse
%    \begin{macrocode}
%</dt-estonian.def>
%    \end{macrocode}
%\fi
%\iffalse
%    \begin{macrocode}
%<*dt-finnish.def>
%    \end{macrocode}
%\fi
%\subsubsection{Finnish}
% Identify file
%    \begin{macrocode}
\ProvidesFile{dt-finnish.def}[2004/10/31]
%    \end{macrocode}
% Define month names.
%    \begin{macrocode}
\providecommand{\monthnamefinnish}[1][\month]{%
\@orgargctr=#1\relax
\ifcase\@orgargctr
\PackageError{datetime}{Invalid Month number \the\@orgargctr}{%
Month numbers should go from 1 to 12}%
\or tammikuuta%
\or helmikuuta%
\or maaliskuuta%
\or huhtikuuta%
\or toukokuuta%
\or kes\"akuuta%
\or hein\"akuuta%
\or elokuuta%
\or syyskuuta%
\or lokakuuta%
\or marraskuuta%
\or joulukuuta%
\else
\PackageError{datetime}{Invalid Month number \the\@orgargctr}{%
Month numbers should go from 1 to 12}%
\fi}
%    \end{macrocode}
% (Re)Define declaration to switch to this format.
%    \begin{macrocode}
\DeclareRobustCommand*\datefinnish{%
\renewcommand{\formatdate}[3]{%
\@day=##1\relax\@month=##2\relax\@year=##3\relax
\number\@day.~\monthnamefinnish[\@month]\space \number\@year}}
%    \end{macrocode}
%\iffalse
%    \begin{macrocode}
%</dt-finnish.def>
%    \end{macrocode}
%\fi
%\iffalse
%    \begin{macrocode}
%<*dt-french.def>
%    \end{macrocode}
%\fi
%\subsubsection{French}
% Identify file
%    \begin{macrocode}
\ProvidesFile{dt-french.def}[2004/10/31]
%    \end{macrocode}
% Define week day names.
%    \begin{macrocode}
\providecommand{\dayofweeknameidfrench}[1]{%
\ifcase#1\relax
\or dimanche%
\or lundi%
\or mardi%
\or mercredi%
\or jeudi%
\or vendredi%
\or samedi%
\fi}
%    \end{macrocode}
% Define abbreviated week day names (are these correct?)
%    \begin{macrocode}
\providecommand{\shortdayofweeknameidfrench}[1]{%
\ifcase#1\relax
\or dim%
\or lun%
\or mar%
\or mer%
\or jeu%
\or ven%
\or sam%
\fi}
%    \end{macrocode}
% Define month names.
%    \begin{macrocode}
\providecommand{\monthnamefrench}[1][\month]{%
\@orgargctr=#1\relax
\ifcase\@orgargctr
\PackageError{datetime}{Invalid Month number \the\@orgargctr}{%
Month numbers should go from 1 (janvier) to 12 (decembre)}%
\or janvier%
\or f\'evrier%
\or mars%
\or avril%
\or mai%
\or juin%
\or juillet%
\or ao\^ut%
\or septembre%
\or octobre%
\or novembre%
\or d\'ecembre%
\else
\PackageError{datetime}{Invalid Month number \the\@orgargctr}{%
Month numbers should go from 1 (janvier) to 12 (decembre)}%
\fi}
%    \end{macrocode}
% (Re)Define declaration to switch to this format.
%    \begin{macrocode}
\DeclareRobustCommand*\datefrench{%
\renewcommand{\formatdate}[3]{%
\@day=##1\relax\@month=##2\relax\@year=##3\relax
\number\@day\ifnum\@day=1{\ier}\fi\space
\monthnamefrench[\@month]\space \number\@year}}
%    \end{macrocode}
%\iffalse
%    \begin{macrocode}
%</dt-french.def>
%    \end{macrocode}
%\fi
%\iffalse
%    \begin{macrocode}
%<*dt-galician.def>
%    \end{macrocode}
%\fi
%\subsubsection{Galician}
% Identify file
%    \begin{macrocode}
\ProvidesFile{dt-galician.def}[2004/10/31]
%    \end{macrocode}
% Define month names.
%    \begin{macrocode}
\providecommand{\monthnamegalician}[1][\month]{%
\@orgargctr=#1\relax
\ifcase\@orgargctr
\PackageError{datetime}{Invalid Month number \the\@orgargctr}{%
Month numbers should go from 1 to 12}%
\or xaneiro%
\or febreiro%
\or marzo%
\or abril%
\or maio%
\or xu\~no%
\or xullo%
\or agosto%
\or setembro%
\or outubro%
\or novembro%
\or decembro%
\else
\PackageError{datetime}{Invalid Month number \the\@orgargctr}{%
Month numbers should go from 1 to 12}%
\fi}
%    \end{macrocode}
% (Re)Define declaration to switch to this format.
%    \begin{macrocode}
\DeclareRobustCommand*\dategalician{%
\renewcommand{\formatdate}[3]{%
\@day=##1\relax\@month=##2\relax\@year=##3\relax
\number\@day~de\space\monthnamegalician[\@month]\space
de~\number\@year}}
%    \end{macrocode}
%\iffalse
%    \begin{macrocode}
%</dt-galician.def>
%    \end{macrocode}
%\fi
%\iffalse
%    \begin{macrocode}
%<*dt-german.def>
%    \end{macrocode}
%\fi
%\subsubsection{German}
% Identify file
%    \begin{macrocode}
\ProvidesFile{dt-german.def}[2004/10/31]
%    \end{macrocode}
% Define month names.
%    \begin{macrocode}
\providecommand{\monthnamegerman}[1][\month]{%
\@orgargctr=#1\relax
\ifcase\@orgargctr
\PackageError{datetime}{Invalid Month number \the\@orgargctr}{%
Month numbers should go from 1 to 12}%
\or Januar%
\or Februar%
\or M\"arz%
\or April%
\or Mai%
\or Juni%
\or Juli%
\or August%
\or September%
\or Oktober%
\or November%
\or Dezember%
\else
\PackageError{datetime}{Invalid Month number \the\@orgargctr}{%
Month numbers should go from 1 (janvier) to 12 (decembre)}%
\fi}
%    \end{macrocode}
% The following week day names were supplied by Uwe Bieling:
%    \begin{macrocode}
\providecommand{\dayofweeknameidgerman}[1]{%
\ifcase#1\relax
\or Sonntag%
\or Montag%
\or Dienstag%
\or Mittwoch%
\or Donnerstag%
\or Freitag%
\or Samstag%
\fi}

\providecommand{\shortdayofweeknameidgerman}[1]{%
\ifcase#1\relax
\or So%
\or Mo%
\or Di%
\or Mi%
\or Do%
\or Fr%
\or Sa%
\fi}
%    \end{macrocode}
% (Re)Define declaration to switch to this format.
%    \begin{macrocode}
\DeclareRobustCommand*\dategerman{%
\renewcommand{\formatdate}[3]{%
\@day=##1\relax\@month=##2\relax\@year=##3\relax
\number\@day.~\monthnamegerman[\@month]\space\number\@year}}
%    \end{macrocode}
%\iffalse
%    \begin{macrocode}
%</dt-german.def>
%    \end{macrocode}
%\fi
%\iffalse
%    \begin{macrocode}
%<*dt-greek.def>
%    \end{macrocode}
%\fi
%\subsubsection{Greek}
% Identify file
%    \begin{macrocode}
\ProvidesFile{dt-greek.def}[2004/10/31]
%    \end{macrocode}
% Define month names.
%    \begin{macrocode}
\providecommand{\monthnamegreek}[1][\month]{%
\@orgargctr=#1\relax
\ifcase\@orgargctr
\PackageError{datetime}{Invalid Month number \the\@orgargctr}{%
Month numbers should go from 1 to 12}%
\or Ianouar'iou%
\or Febrouar'iou%
\or Mart'iou%
\or April'iou%
\or Ma'"iou%
\or Ioun'iou%
\or Ioul'iou%
\or Augo'ustou%
\or Septembr'iou%
\or Oktwbr'iou%
\or Noembr'iou%
\or Dekembr'iou%
\else
\PackageError{datetime}{Invalid Month number \the\@orgargctr}{%
Month numbers should go from 1 to 12}%
\fi}
%    \end{macrocode}
% (Re)Define declaration to switch to this format.
%    \begin{macrocode}
\DeclareRobustCommand*\dategreek{%
\renewcommand{\formatdate}[3]{%
\@day=##1\relax\@month=##2\relax\@year=##3\relax
\number\@day\space\monthnamegreek[\@month]\space\number\@year}}
%    \end{macrocode}
%\iffalse
%    \begin{macrocode}
%</dt-greek.def>
%    \end{macrocode}
%\fi
%\iffalse
%    \begin{macrocode}
%<*dt-hebrew.def>
%    \end{macrocode}
%\fi
%\subsubsection{Hebrew}
% Identify file
%    \begin{macrocode}
\ProvidesFile{dt-hebrew.def}[2004/10/31]
%    \end{macrocode}
% Babel already provides Hebrew month names, so just provide a
% synonym.
%    \begin{macrocode}
\let\monthnamehebrew=\hebmonth
%    \end{macrocode}
% Redefine declaration to switch to this format. (This uses
% |\hebdate| which is defined by babel.)
%    \begin{macrocode}
\DeclareRobustCommand*\datehebrew{%
\renewcommand{\formatdate}[3]{%
\@day=##1\relax\@month=##2\relax\@year=##3\relax
\hebdate\@day\@month\@year}}
%    \end{macrocode}
%\iffalse
%    \begin{macrocode}
%</dt-hebrew.def>
%    \end{macrocode}
%\fi
%\iffalse
%    \begin{macrocode}
%<*dt-icelandic.def>
%    \end{macrocode}
%\fi
%\subsubsection{Icelandic}
% Identify file
%    \begin{macrocode}
\ProvidesFile{dt-icelandic.def}[2004/10/31]
%    \end{macrocode}
% Define month names.
%    \begin{macrocode}
\providecommand{\monthnameicelandic}[1][\month]{%
\@orgargctr=#1\relax
\ifcase\@orgargctr
\PackageError{datetime}{Invalid Month number \the\@orgargctr}{%
Month numbers should go from 1 to 12}%
\or jan�ar%
\or febr�ar%
\or mars%
\or apr�l%
\or ma�%
\or j�n�%
\or j�l�%
\or �g�st%
\or september%
\or okt�ber%
\or n�vember%
\or desembe%
\else
\PackageError{datetime}{Invalid Month number \the\@orgargctr}{%
Month numbers should go from 1 to 12}%
\fi}
%    \end{macrocode}
% (Re)Define declaration to switch to this format.
%    \begin{macrocode}
\DeclareRobustCommand*\dateicelandic{%
\renewcommand{\formatdate}[3]{%
\@day=##1\relax\@month=##2\relax\@year=##3\relax
\number\@day.~\monthnameicelandic[\@month]\space\number\@year}}
%    \end{macrocode}
%\iffalse
%    \begin{macrocode}
%</dt-icelandic.def>
%    \end{macrocode}
%\fi
%\iffalse
%    \begin{macrocode}
%<*dt-irish.def>
%    \end{macrocode}
%\fi
%\subsubsection{Irish}
% Identify file
%    \begin{macrocode}
\ProvidesFile{dt-irish.def}[2004/10/31]
%    \end{macrocode}
% Define month names.
%    \begin{macrocode}
\providecommand{\monthnameirish}[1][\month]{%
\@orgargctr=#1\relax
\ifcase\@orgargctr
\PackageError{datetime}{Invalid Month number \the\@orgargctr}{%
Month numbers should go from 1 to 12}%
\or  Ean\'air%
\or Feabhra%
\or M\'arta%
\or Aibre\'an%
\or Bealtaine%
\or Meitheamh%
\or I\'uil%
\or L\'unasa%
\or Me\'an F\'omhair%
\or Deireadh F\'omhair%
\or M\'{\i} na Samhna%
\or M\'{\i} na Nollag%
\else
\PackageError{datetime}{Invalid Month number \the\@orgargctr}{%
Month numbers should go from 1 to 12}%
\fi}
%    \end{macrocode}
% (Re)Define declaration to switch to this format.
%    \begin{macrocode}
\DeclareRobustCommand*\dateirish{%
\renewcommand{\formatdate}[3]{%
\@day=##1\relax\@month=##2\relax\@year=##3\relax
\number\@day\space\monthnameirish[\@month]\space\number\@year}}
%    \end{macrocode}
%\iffalse
%    \begin{macrocode}
%</dt-irish.def>
%    \end{macrocode}
%\fi
%\iffalse
%    \begin{macrocode}
%<*dt-italian.def>
%    \end{macrocode}
%\fi
%\subsubsection{Italian}
% Identify file
%    \begin{macrocode}
\ProvidesFile{dt-italian.def}[2004/10/31]
%    \end{macrocode}
% Define month names.
%    \begin{macrocode}
\providecommand{\monthnameitalian}[1][\month]{%
\@orgargctr=#1\relax
\ifcase\@orgargctr
\PackageError{datetime}{Invalid Month number \the\@orgargctr}{%
Month numbers should go from 1 to 12}%
\or gennaio%
\or febbraio%
\or marzo%
\or aprile%
\or maggio%
\or giugno%
\or luglio%
\or agosto%
\or settembre%
\or ottobre%
\or novembre%
\or dicembre%
\else
\PackageError{datetime}{Invalid Month number \the\@orgargctr}{%
Month numbers should go from 1 to 12}%
\fi}
%    \end{macrocode}
% (Re)Define declaration to switch to this format.
%    \begin{macrocode}
\DeclareRobustCommand*\dateitalian{%
\renewcommand{\formatdate}[3]{%
\@day=##1\relax\@month=##2\relax\@year=##3\relax
\number\@day\space\monthnameitalian[\@month]\space\number\@year}}
%    \end{macrocode}
%\iffalse
%    \begin{macrocode}
%</dt-italian.def>
%    \end{macrocode}
%\fi
%\iffalse
%    \begin{macrocode}
%<*dt-latin.def>
%    \end{macrocode}
%\fi
%\subsubsection{Latin}
% Identify file
%    \begin{macrocode}
\ProvidesFile{dt-latin.def}[2004/10/31]
%    \end{macrocode}
% Define month names.
%    \begin{macrocode}
\providecommand{\monthnamelatin}[1][\month]{%
\@orgargctr=#1\relax
\ifcase\@orgargctr
\PackageError{datetime}{Invalid Month number \the\@orgargctr}{%
Month numbers should go from 1 to 12}%
\or Ianuarii%
\or Februarii%
\or Martii%
\or Aprilis%
\or Maii%
\or Iunii%
\or Iulii%
\or Augusti%
\or Septembris%
\or Octobris%
\or Novembris%
\or Decembris%
\else
\PackageError{datetime}{Invalid Month number \the\@orgargctr}{%
Month numbers should go from 1 to 12}%
\fi}
%    \end{macrocode}
% (Re)Define declaration to switch to this format.
%    \begin{macrocode}
\DeclareRobustCommand*\datelatin{%
\renewcommand{\formatdate}[3]{%
\@day=##1\relax\@month=##2\relax\@year=##3\relax
\check@mathfonts\fontsize\sf@size\z@\math@fontsfalse\selectfont
\uppercase\expandafter{\romannumeral\@day}%
~\monthnamelatin[\@month]\space
{\uppercase\expandafter{\romannumeral\@year}}}}
%    \end{macrocode}
%\iffalse
%    \begin{macrocode}
%</dt-latin.def>
%    \end{macrocode}
%\fi
%\iffalse
%    \begin{macrocode}
%<*dt-lsorbian.def>
%    \end{macrocode}
%\fi
%\subsubsection{LSorbian}
% Identify file
%    \begin{macrocode}
\ProvidesFile{dt-lsorbian.def}[2004/10/31]
%    \end{macrocode}
% Define new month names.
%    \begin{macrocode}
\providecommand{\monthnamenewlsorbian}[1][\month]{%
\@orgargctr=#1\relax
\ifcase\@orgargctr
\PackageError{datetime}{Invalid Month number \the\@orgargctr}{%
Month numbers should go from 1 to 12}%
\or januara%
\or februara%
\or m\v erca%
\or apryla%
\or maja%
\or junija%
\or julija%
\or awgusta%
\or septembra%
\or oktobra%
\or nowembra%
\or decembra%
\else
\PackageError{datetime}{Invalid Month number \the\@orgargctr}{%
Month numbers should go from 1 to 12}%
\fi}
%    \end{macrocode}
% Define old month names.
%    \begin{macrocode}
\providecommand{\monthnameoldlsorbian}[1][\month]{%
\@orgargctr=#1\relax
\ifcase\@orgargctr
\PackageError{datetime}{Invalid Month number \the\@orgargctr}{%
Month numbers should go from 1 to 12}%
\or wjelikego ro\v zka%
\or ma\l ego ro\v zka%
\or nal\v etnika%
\or jat\v sownika%
\or ro\v zownika%
\or sma\v znika%
\or pra\v znika%
\or \v znje\'nca%
\or po\v znje\'nca%
\or winowca%
\or nazymnika%
\or godownika%
\else
\PackageError{datetime}{Invalid Month number \the\@orgargctr}{%
Month numbers should go from 1 to 12}%
\fi}
%    \end{macrocode}
% Set the default month names.
%    \begin{macrocode}
\let\monthnamelsorbian=\monthnamenewlsorbian
%    \end{macrocode}
% (Re)Define declaration to switch to new format.
%    \begin{macrocode}
\DeclareRobustCommand*\newdatelsorbian{%
\renewcommand{\formatdate}[3]{%
\@day=##1\relax\@month=##2\relax\@year=##3\relax
\number\@day.~\monthnamenewlsorbian[\@month]\space\number\@year}}
%    \end{macrocode}
% (Re)Define declaration to switch to old format.
%    \begin{macrocode}
\DeclareRobustCommand*\olddatelsorbian{%
\renewcommand{\formatdate}[3]{%
\@day=##1\relax\@month=##2\relax\@year=##3\relax
\number\@day.~\monthnameoldlsorbian[\@month]\space\number\@year}}
%    \end{macrocode}
% Set the default date format.
%    \begin{macrocode}
\let\datelsorbian\newdatelsorbian
%    \end{macrocode}
%\iffalse
%    \begin{macrocode}
%</dt-lsorbian.def>
%    \end{macrocode}
%\fi
%\iffalse
%    \begin{macrocode}
%<*dt-magyar.def>
%    \end{macrocode}
%\fi
%\subsubsection{Magyar}
% Identify file
%    \begin{macrocode}
\ProvidesFile{dt-magyar.def}[2004/10/31]
%    \end{macrocode}
% Define month names.
%    \begin{macrocode}
\providecommand{\monthnamemagyar}[1][\month]{%
\@orgargctr=#1\relax
\ifcase\@orgargctr
\PackageError{datetime}{Invalid Month number \the\@orgargctr}{%
Month numbers should go from 1 to 12}%
\or janu\'ar%
\or febru\'ar%
\or m\'arcius%
\or \'aprilis%
\or m\'ajus%
\or j\'unius%
\or j\'ulius%
\or augusztus%
\or szeptember%
\or okt\'ober%
\or november%
\or december%
\else
\PackageError{datetime}{Invalid Month number \the\@orgargctr}{%
Month numbers should go from 1 to 12}%
\fi}
%    \end{macrocode}
% (Re)Define declaration to switch to this format.
%    \begin{macrocode}
\DeclareRobustCommand*\datemagyar{%
\renewcommand{\formatdate}[3]{%
\@day=##1\relax\@month=##2\relax\@year=##3\relax
\number\@year.~\monthnamemagyar[\@month]\space\number\@day.}}
%    \end{macrocode}
%\iffalse
%    \begin{macrocode}
%</dt-magyar.def>
%    \end{macrocode}
%\fi
%\iffalse
%    \begin{macrocode}
%<*dt-naustrian.def>
%    \end{macrocode}
%\fi
%\subsubsection{NAustrian}
% Identify file
%    \begin{macrocode}
\ProvidesFile{dt-naustrian.def}[2004/10/31]
%    \end{macrocode}
% Define month names.
%    \begin{macrocode}
\providecommand{\monthnamenaustrian}[1][\month]{%
\@orgargctr=#1\relax
\ifcase\@orgargctr
\PackageError{datetime}{Invalid Month number \the\@orgargctr}{%
Month numbers should go from 1 to 12}%
\or J\"anner%
\or Februar%
\or M\"arz%
\or April%
\or Mai%
\or Juni%
\or Juli%
\or August%
\or September%
\or Oktober%
\or November%
\or Dezember%
\else
\PackageError{datetime}{Invalid Month number \the\@orgargctr}{%
Month numbers should go from 1 to 12}%
\fi}
%    \end{macrocode}
% (Re)Define declaration to switch to this format.
%    \begin{macrocode}
\DeclareRobustCommand*\datenaustrian{%
\renewcommand{\formatdate}[3]{%
\@day=##1\relax\@month=##2\relax\@year=##3\relax
\number\@day.~\monthnamenaustrian[\@month]\space\number\@year}}
%    \end{macrocode}
%\iffalse
%    \begin{macrocode}
%</dt-naustrian.def>
%    \end{macrocode}
%\fi
%\iffalse
%    \begin{macrocode}
%<*dt-ngerman.def>
%    \end{macrocode}
%\fi
%\subsubsection{NGerman}
% Identify file
%    \begin{macrocode}
\ProvidesFile{dt-ngerman.def}[2004/10/31]
%    \end{macrocode}
% Define month names.
%    \begin{macrocode}
\providecommand{\monthnamengerman}[1][\month]{%
\@orgargctr=#1\relax
\ifcase\@orgargctr
\PackageError{datetime}{Invalid Month number \the\@orgargctr}{%
Month numbers should go from 1 to 12}%
\or Januar%
\or Februar%
\or M\"arz%
\or April%
\or Mai%
\or Juni%
\or Juli%
\or August%
\or September%
\or Oktober%
\or November%
\or Dezember%
\else
\PackageError{datetime}{Invalid Month number \the\@orgargctr}{%
Month numbers should go from 1 to 12}%
\fi}
%    \end{macrocode}
% The following week day names were supplied by Uwe Bieling:
%    \begin{macrocode}
\providecommand{\dayofweeknameidngerman}[1]{%
\ifcase#1\relax
\or Sonntag%
\or Montag%
\or Dienstag%
\or Mittwoch%
\or Donnerstag%
\or Freitag%
\or Samstag%
\fi}

\providecommand{\shortdayofweeknameidngerman}[1]{%
\ifcase#1\relax
\or So%
\or Mo%
\or Di%
\or Mi%
\or Do%
\or Fr%
\or Sa%
\fi}
%    \end{macrocode}
% (Re)Define declaration to switch to this format.
%    \begin{macrocode}
\DeclareRobustCommand*\datengerman{%
\renewcommand{\formatdate}[3]{%
\@day=##1\relax\@month=##2\relax\@year=##3\relax
\number\@day.~\monthnamengerman[\@month]\space\number\@year}}
%    \end{macrocode}
%\iffalse
%    \begin{macrocode}
%</dt-ngerman.def>
%    \end{macrocode}
%\fi
%\iffalse
%    \begin{macrocode}
%<*dt-norsk.def>
%    \end{macrocode}
%\fi
%\subsubsection{Norsk}
% Identify file
%    \begin{macrocode}
\ProvidesFile{dt-norsk.def}[2004/10/31]
%    \end{macrocode}
% Define month names.
%    \begin{macrocode}
\providecommand{\monthnamenorsk}[1][\month]{%
\@orgargctr=#1\relax
\ifcase\@orgargctr
\PackageError{datetime}{Invalid Month number \the\@orgargctr}{%
Month numbers should go from 1 to 12}%
\or januar%
\or februar%
\or mars%
\or april%
\or mai%
\or juni%
\or juli%
\or august%
\or september%
\or oktober%
\or november%
\or desember%
\else
\PackageError{datetime}{Invalid Month number \the\@orgargctr}{%
Month numbers should go from 1 to 12}%
\fi}
%    \end{macrocode}
% (Re)Define declaration to switch to this format.
%    \begin{macrocode}
\DeclareRobustCommand*\datenorsk{%
\renewcommand{\formatdate}[3]{%
\@day=##1\relax\@month=##2\relax\@year=##3\relax
\number\@day.~\monthnamenorsk[\@month]\space\number\@year}}
%    \end{macrocode}
%\iffalse
%    \begin{macrocode}
%</dt-norsk.def>
%    \end{macrocode}
%\fi
%\iffalse
%    \begin{macrocode}
%<*dt-polish.def>
%    \end{macrocode}
%\fi
%\subsubsection{Polish}
% Identify file
%    \begin{macrocode}
\ProvidesFile{dt-polish.def}[2004/10/31]
%    \end{macrocode}
% Define month names.
%    \begin{macrocode}
\providecommand{\monthnamepolish}[1][\month]{%
\@orgargctr=#1\relax
\ifcase\@orgargctr
\PackageError{datetime}{Invalid Month number \the\@orgargctr}{%
Month numbers should go from 1 to 12}%
\or stycznia%
\or lutego%
\or marca%
\or kwietnia%
\or maja%
\or czerwca%
\or lipca%
\or sierpnia%
\or wrze\'snia%
\or pa\'zdziernika%
\or listopada%
\or grudnia%
\else
\PackageError{datetime}{Invalid Month number \the\@orgargctr}{%
Month numbers should go from 1 to 12}%
\fi}
%    \end{macrocode}
% (Re)Define declaration to switch to this format.
%    \begin{macrocode}
\DeclareRobustCommand*\datepolish{%
\renewcommand{\formatdate}[3]{%
\@day=##1\relax\@month=##2\relax\@year=##3\relax
\number\@day~\monthnamepolish[\@month]\space\number\@year}}
%    \end{macrocode}
%\iffalse
%    \begin{macrocode}
%</dt-polish.def>
%    \end{macrocode}
%\fi
%\iffalse
%    \begin{macrocode}
%<*dt-portuges.def>
%    \end{macrocode}
%\fi
%\subsubsection{Portuges}
% Identify file
%    \begin{macrocode}
\ProvidesFile{dt-portuges.def}[2004/10/31]
%    \end{macrocode}
% Define week day names.
%    \begin{macrocode}
\providecommand{\dayofweeknameidportuges}[1]{%
\ifcase#1\relax
\or domingo%
\or segunda-feira%
\or ter\c{c}a-feira%
\or quarta-feira%
\or quinta-feira%
\or sexta-feira%
\or sabado%
\fi}
%    \end{macrocode}
% Define month names.
%    \begin{macrocode}
\providecommand{\monthnameportuges}[1][\month]{%
\@orgargctr=#1\relax
\ifcase\@orgargctr
\PackageError{datetime}{Invalid Month number \the\@orgargctr}{%
Month numbers should go from 1 to 12}%
\or Janeiro%
\or Fevereiro%
\or Mar\c{c}o%
\or Abril%
\or Maio%
\or Junho%
\or Julho%
\or Agosto%
\or Setembro%
\or Outubro%
\or Novembro%
\or Dezembro%
\else
\PackageError{datetime}{Invalid Month number \the\@orgargctr}{%
Month numbers should go from 1 to 12}%
\fi}
%    \end{macrocode}
% (Re)Define declaration to switch to this format.
%    \begin{macrocode}
\DeclareRobustCommand*\dateportuges{%
\renewcommand{\formatdate}[3]{%
\@day=##1\relax\@month=##2\relax\@year=##3\relax
\number\@day\space de\space\monthnameportuges[\@month]\space
de\space\number\@year}}
%    \end{macrocode}
%\iffalse
%    \begin{macrocode}
%</dt-portuges.def>
%    \end{macrocode}
%\fi
%\iffalse
%    \begin{macrocode}
%<*dt-romanian.def>
%    \end{macrocode}
%\fi
%\subsubsection{Romanian}
% Identify file
%    \begin{macrocode}
\ProvidesFile{dt-romanian.def}[2004/10/31]
%    \end{macrocode}
% Define month names.
%    \begin{macrocode}
\providecommand{\monthnameromanian}[1][\month]{%
\@orgargctr=#1\relax
\ifcase\@orgargctr
\PackageError{datetime}{Invalid Month number \the\@orgargctr}{%
Month numbers should go from 1 to 12}%
\or ianuarie%
\or februarie%
\or martie%
\or aprilie%
\or mai%
\or iunie%
\or iulie%
\or august%
\or septembrie%
\or octombrie%
\or noiembrie%
\or decembrie%
\else
\PackageError{datetime}{Invalid Month number \the\@orgargctr}{%
Month numbers should go from 1 to 12}%
\fi}
%    \end{macrocode}
% (Re)Define declaration to switch to this format.
%    \begin{macrocode}
\DeclareRobustCommand*\dateromanian{%
\renewcommand{\formatdate}[3]{%
\@day=##1\relax\@month=##2\relax\@year=##3\relax
\number\@day~\monthnameromanian[\@month]\space\number\@year}}
%    \end{macrocode}
%\iffalse
%    \begin{macrocode}
%</dt-romanian.def>
%    \end{macrocode}
%\fi
%\iffalse
%    \begin{macrocode}
%<*dt-russian.def>
%    \end{macrocode}
%\fi
%\subsubsection{Russian}
% Identify file
%    \begin{macrocode}
\ProvidesFile{dt-russian.def}[2004/10/31]
%    \end{macrocode}
% Define month names.
%    \begin{macrocode}
\providecommand{\monthnamerussian}[1][\month]{%
\@orgargctr=#1\relax
\ifcase\@orgargctr
\PackageError{datetime}{Invalid Month number \the\@orgargctr}{%
Month numbers should go from 1 to 12}%
\or \cyrya\cyrn\cyrv\cyra\cyrr\cyrya\or
    \cyrf\cyre\cyrv\cyrr\cyra\cyrl\cyrya\or
    \cyrm\cyra\cyrr\cyrt\cyra\or
    \cyra\cyrp\cyrr\cyre\cyrl\cyrya\or
    \cyrm\cyra\cyrya\or
    \cyri\cyryu\cyrn\cyrya\or
    \cyri\cyryu\cyrl\cyrya\or
    \cyra\cyrv\cyrg\cyru\cyrs\cyrt\cyra\or
    \cyrs\cyre\cyrn\cyrt\cyrya\cyrb\cyrr\cyrya\or
    \cyro\cyrk\cyrt\cyrya\cyrb\cyrr\cyrya\or
    \cyrn\cyro\cyrya\cyrb\cyrr\cyrya\or
    \cyrd\cyre\cyrk\cyra\cyrb\cyrr\cyrya%
\else
\PackageError{datetime}{Invalid Month number \the\@orgargctr}{%
Month numbers should go from 1 to 12}%
\fi}
%    \end{macrocode}
% (Re)Define declaration to switch to this format.
%    \begin{macrocode}
\DeclareRobustCommand*\daterussian{%
\renewcommand{\formatdate}[3]{%
\@day=##1\relax\@month=##2\relax\@year=##3\relax
\number\@day~\monthnamerussian[\@month]\ \number\@year~\cyrg.}}
%    \end{macrocode}
%\iffalse
%    \begin{macrocode}
%</dt-russian.def>
%    \end{macrocode}
%\fi
%\iffalse
%    \begin{macrocode}
%<*dt-samin.def>
%    \end{macrocode}
%\fi
%\subsubsection{Samin}
% Identify file
%    \begin{macrocode}
\ProvidesFile{dt-samin.def}[2004/10/31]
%    \end{macrocode}
% Define month names.
%    \begin{macrocode}
\providecommand{\monthnamesamin}[1][\month]{%
\@orgargctr=#1\relax
\ifcase\@orgargctr
\PackageError{datetime}{Invalid Month number \the\@orgargctr}{%
Month numbers should go from 1 to 12}%
\or o\dj{}\dj{}ajagem\'anu\or
    guovvam\'anu\or
    njuk\v cam\'anu\or
    cuo\ng{}om\'anu\or
    miessem\'anu\or
    geassem\'anu\or
    suoidnem\'anu\or
    borgem\'anu\or
    \v cak\v cam\'anu\or
    golggotm\'anu\or
    sk\'abmam\'anu\or
    juovlam\'anu%
\else
\PackageError{datetime}{Invalid Month number \the\@orgargctr}{%
Month numbers should go from 1 to 12}%
\fi}
%    \end{macrocode}
% (Re)Define declaration to switch to this format.
%    \begin{macrocode}
\DeclareRobustCommand*\datesamin{%
\renewcommand{\formatdate}[3]{%
\@day=##1\relax\@month=##2\relax\@year=##3\relax
\monthnamesamin[\@month]\space\number\@day.~b.\space
\number\@year}}
%    \end{macrocode}
%\iffalse
%    \begin{macrocode}
%</dt-samin.def>
%    \end{macrocode}
%\fi
%\iffalse
%    \begin{macrocode}
%<*dt-scottish.def>
%    \end{macrocode}
%\fi
%\subsubsection{Scottish}
% Identify file
%    \begin{macrocode}
\ProvidesFile{dt-scottish.def}[2004/10/31]
%    \end{macrocode}
% Define month names.
%    \begin{macrocode}
\providecommand{\monthnamescottish}[1][\month]{%
\@orgargctr=#1\relax
\ifcase\@orgargctr
\PackageError{datetime}{Invalid Month number \the\@orgargctr}{%
Month numbers should go from 1 to 12}%
\or am Faoilteach%
\or an Gearran%
\or am M\`art%
\or an Giblean%
\or an C\`eitean%
\or an t-\`Og mhios%
\or an t-Iuchar%
\or L\`unasdal%
\or an Sultuine%
\or an D\`amhar%
\or an t-Samhainn%
\or an Dubhlachd%
\else
\PackageError{datetime}{Invalid Month number \the\@orgargctr}{%
Month numbers should go from 1 to 12}%
\fi}
%    \end{macrocode}
% (Re)Define declaration to switch to this format.
%    \begin{macrocode}
\DeclareRobustCommand*\datescottish{%
\renewcommand{\formatdate}[3]{%
\@day=##1\relax\@month=##2\relax\@year=##3\relax
\number\@day\space\monthnamescottish[\@month]\space \number\@year}}
%    \end{macrocode}
%\iffalse
%    \begin{macrocode}
%</dt-scottish.def>
%    \end{macrocode}
%\fi
%\iffalse
%    \begin{macrocode}
%<*dt-serbian.def>
%    \end{macrocode}
%\fi
%\subsubsection{Serbian}
% Identify file
%    \begin{macrocode}
\ProvidesFile{dt-serbian.def}[2004/10/31]
%    \end{macrocode}
% Define month names.
%    \begin{macrocode}
\providecommand{\monthnameserbian}[1][\month]{%
\@orgargctr=#1\relax
\ifcase\@orgargctr
\PackageError{datetime}{Invalid Month number \the\@orgargctr}{%
Month numbers should go from 1 to 12}%
\or januar%
\or februar%
\or mart%
\or april%
\or maj%
\or juni%
\or juli%
\or avgust%
\or septembar%
\or oktobar%
\or novembar%
\or decembar%
\else
\PackageError{datetime}{Invalid Month number \the\@orgargctr}{%
Month numbers should go from 1 to 12}%
\fi}
%    \end{macrocode}
% (Re)Define declaration to switch to this format.
%    \begin{macrocode}
\DeclareRobustCommand*\dateserbian{%
\renewcommand{\formatdate}[3]{%
\@day=##1\relax\@month=##2\relax\@year=##3\relax
\number\@day.~\monthnameserbian[\@month]\space \number\@year}}
%    \end{macrocode}
%\iffalse
%    \begin{macrocode}
%</dt-serbian.def>
%    \end{macrocode}
%\fi
%\iffalse
%    \begin{macrocode}
%<*dt-slovak.def>
%    \end{macrocode}
%\fi
%\subsubsection{Slovak}
% Identify file
%    \begin{macrocode}
\ProvidesFile{dt-slovak.def}[2004/10/31]
%    \end{macrocode}
% Define month names.
%    \begin{macrocode}
\providecommand{\monthnameslovak}[1][\month]{%
\@orgargctr=#1\relax
\ifcase\@orgargctr
\PackageError{datetime}{Invalid Month number \the\@orgargctr}{Month 
numbers should go from 1 to 12}%
\or janu\'ara%
\or febru\'ara%
\or marca%
\or apr\'{\i}la%
\or m\'aja%
\or j\'una%
\or j\'ula%
\or augusta%
\or septembra%
\or okt\'obra%
\or novembra%
\or decembra%
\else \PackageError{datetime}{Invalid Month number \the\@orgargctr}{%
Month numbers should go from 1 to 12}%
\fi}
%    \end{macrocode}
% (Re)Define declaration to switch to this format.
%    \begin{macrocode}
\DeclareRobustCommand*\dateslovak{%
\renewcommand{\formatdate}[3]{%
\@day=##1\relax\@month=##2\relax\@year=##3\relax
\number\@day.~\monthnameslovak[\@month]\space \number\@year}}
%    \end{macrocode}
%\iffalse
%    \begin{macrocode}
%</dt-slovak.def>
%    \end{macrocode}
%\fi
%\iffalse
%    \begin{macrocode}
%<*dt-slovene.def>
%    \end{macrocode}
%\fi
%\subsubsection{Slovene}
% Identify file
%    \begin{macrocode}
\ProvidesFile{dt-slovene.def}[2004/10/31]
%    \end{macrocode}
% Define month names.
%    \begin{macrocode}
\providecommand{\monthnameslovene}[1][\month]{%
\@orgargctr=#1\relax
\ifcase\@orgargctr
\PackageError{datetime}{Invalid Month number \the\@orgargctr}{%
Month numbers should go from 1 to 12}%
\or januar%
\or februar%
\or marec%
\or april%
\or maj%
\or junij%
\or julij%
\or avgust%
\or september%
\or oktober%
\or november%
\or december%
\else \PackageError{datetime}{Invalid Month number \the\@orgargctr}{%
Month numbers should go from 1 to 12}%
\fi}
%    \end{macrocode}
% (Re)Define declaration to switch to this format.
%    \begin{macrocode}
\DeclareRobustCommand*\dateslovene{%
\renewcommand{\formatdate}[3]{%
\@day=##1\relax\@month=##2\relax\@year=##3\relax
\number\@day.~\monthnameslovene[\@month]\space \number\@year}}
%    \end{macrocode}
%\iffalse
%    \begin{macrocode}
%</dt-slovene.def>
%    \end{macrocode}
%\fi
%\iffalse
%    \begin{macrocode}
%<*dt-spanish.def>
%    \end{macrocode}
%\fi
%\subsubsection{Spanish}
% Identify file
%    \begin{macrocode}
\ProvidesFile{dt-spanish.def}[2004/10/31]
%    \end{macrocode}
% Define week day names.
%    \begin{macrocode}
\providecommand{\dayofweeknameidspanish}[1]{%
\ifcase#1\relax
\or domingo%
\or lunes%
\or martes%
\or mi\'ercoles%
\or jueves%
\or viernes%
\or s\'abado%
\fi}
%    \end{macrocode}
% Define abbreviated week day names (is this correct?)
%    \begin{macrocode}
\providecommand{\shortdayofweeknameidspanish}[1]{%
\ifcase#1\relax
\or dom%
\or lun%
\or mar%
\or mi\'e%
\or jue%
\or vie%
\or s\'ab%
\fi}
%    \end{macrocode}
% Define month names.
%    \begin{macrocode}
\providecommand{\monthnamespanish}[1][\month]{%
\@orgargctr=#1\relax
\ifcase\@orgargctr
\PackageError{datetime}{Invalid Month number \the\@orgargctr}{%
Month numbers should go from 1 to 12}%
\or enero%
\or febrero%
\or marzo%
\or abril%
\or mayo%
\or junio%
\or julio%
\or agosto%
\or septiembre%
\or octubre%
\or noviembre%
\or diciembre%
\else \PackageError{datetime}{Invalid Month number \the\@orgargctr}{%
Month numbers should go from 1 to 12}%
\fi}
%    \end{macrocode}
% (Re)Define declaration to switch to this format.
%    \begin{macrocode}
\DeclareRobustCommand*\datespanish{%
\renewcommand{\formatdate}[3]{%
\@day=##1\relax\@month=##2\relax\@year=##3\relax
\number\@day~de \monthnamespanish[\@month]\ de~\number\@year}}
%    \end{macrocode}
%\iffalse
%    \begin{macrocode}
%</dt-spanish.def>
%    \end{macrocode}
%\fi
%\iffalse
%    \begin{macrocode}
%<*dt-swedish.def>
%    \end{macrocode}
%\fi
%\subsubsection{Swedish}
% Identify file
%    \begin{macrocode}
\ProvidesFile{dt-swedish.def}[2004/10/31]
%    \end{macrocode}
% Define month names.
%    \begin{macrocode}
\providecommand{\monthnameswedish}[1][\month]{%
\@orgargctr=#1\relax
\ifcase\@orgargctr
\PackageError{datetime}{Invalid Month number \the\@orgargctr}{%
Month numbers should go from 1 to 12}%
\or januari%
\or februari%
\or mars%
\or april%
\or maj%
\or juni%
\or juli%
\or augusti%
\or september%
\or oktober%
\or november%
\or december%
\else \PackageError{datetime}{Invalid Month number \the\@orgargctr}{%
Month numbers should go from 1 to 12}%
\fi}
%    \end{macrocode}
% (Re)Define declaration to switch to this format 
% (day monthname year).
%    \begin{macrocode}
\DeclareRobustCommand*\dateswedish{%
\renewcommand{\formatdate}[3]{%
\@day=##1\relax\@month=##2\relax\@year=##3\relax
\number\@day~\monthnameswedish[\@month]\space\number\@year}}
%    \end{macrocode}
% (Re)Define declaration to switch to this format
% (two-digit numerical).
%    \begin{macrocode}
\DeclareRobustCommand*\datesymd{%
\renewcommand{\formatdate}[3]{%
\@day=##1\relax\@month=##2\relax\@year=##3\relax
\number\@year-\two@digits\@month-\two@digits\@day}
}
%    \end{macrocode}
% (Re)Define declaration to switch to this format
% (numerical).
%    \begin{macrocode}
\DeclareRobustCommand*\datesdmy{%
\renewcommand{\formatdate}[3]{%
\@day=##1\relax\@month=##2\relax\@year=##3\relax
\number\@day/\number\@month\space\number\@year}
}
%    \end{macrocode}
%\iffalse
%    \begin{macrocode}
%</dt-swedish.def>
%    \end{macrocode}
%\fi
%\iffalse
%    \begin{macrocode}
%<*dt-turkish.def>
%    \end{macrocode}
%\fi
%\subsubsection{Turkish}
% Identify file
%    \begin{macrocode}
\ProvidesFile{dt-turkish.def}[2004/10/31]
%    \end{macrocode}
% Define month names.
%    \begin{macrocode}
\providecommand{\monthnameturkish}[1][\month]{%
\@orgargctr=#1\relax
\ifcase\@orgargctr
\PackageError{datetime}{Invalid Month number \the\@orgargctr}{%
Month numbers should go from 1 to 12}%
\or  Ocak%
\or \c Subat%
\or Mart%
\or Nisan%
\or May\i{}s%
\or Haziran%
\or Temmuz%
\or A\u gustos%
\or Eyl\"ul%
\or Ekim%
\or Kas\i{}m%
\or Aral\i{}k%
\else \PackageError{datetime}{Invalid Month number \the\@orgargctr}{%
Month numbers should go from 1 to 12}%
\fi}
%    \end{macrocode}
% (Re)Define declaration to switch to this format.
%    \begin{macrocode}
\DeclareRobustCommand*\dateturkish{%
\renewcommand{\formatdate}[3]{%
\@day=##1\relax\@month=##2\relax\@year=##3\relax
\number\@day~\monthnameturkish[\@month]\space\number\@year}}
%    \end{macrocode}
%\iffalse
%    \begin{macrocode}
%</dt-turkish.def>
%    \end{macrocode}
%\fi
%\iffalse
%    \begin{macrocode}
%<*dt-UKenglish.def>
%    \end{macrocode}
%\fi
%\subsubsection{UKenglish}
% Identify file
%    \begin{macrocode}
\ProvidesFile{dt-UKenglish.def}[2007/06/11]
%    \end{macrocode}
\let\dateUKenglish\longdate

\let\monthnameUKenglish\monthnameenglish
\let\shortmonthnameUKenglish\shortmonthnameenglish

\let\dayofweeknameidUKenglish\dayofweeknameidenglish
\let\shortdayofweeknameidUKenglish\shortdayofweeknameidenglish
%\iffalse
%    \begin{macrocode}
%</dt-UKenglish.def>
%    \end{macrocode}
%\fi
%\iffalse
%    \begin{macrocode}
%<*dt-ukraineb.def>
%    \end{macrocode}
%\fi
%\subsubsection{Ukraine}
% Identify file
%    \begin{macrocode}
\ProvidesFile{dt-ukraineb.def}[2004/10/31]
%    \end{macrocode}
% Define month names.
%    \begin{macrocode}
\providecommand{\monthnameukraineb}[1][\month]{%
\@orgargctr=#1\relax
\ifcase\@orgargctr
\PackageError{datetime}{Invalid Month number \the\@orgargctr}{%
Month numbers should go from 1 to 12}%
\or \cyrs\cyrii\cyrch\cyrn\cyrya\or
    \cyrl\cyryu\cyrt\cyro\cyrg\cyro\or
    \cyrb\cyre\cyrr\cyre\cyrz\cyrn\cyrya\or
    \cyrk\cyrv\cyrii\cyrt\cyrn\cyrya\or
    \cyrt\cyrr\cyra\cyrv\cyrn\cyrya\or
    \cyrch\cyre\cyrr\cyrv\cyrn\cyrya\or
    \cyrl\cyri\cyrp\cyrn\cyrya\or
    \cyrs\cyre\cyrr\cyrp\cyrn\cyrya\or
    \cyrv\cyre\cyrr\cyre\cyrs\cyrn\cyrya\or
    \cyrzh\cyro\cyrv\cyrt\cyrn\cyrya\or
    \cyrl\cyri\cyrs\cyrt\cyro\cyrp\cyra\cyrd\cyra\or
    \cyrg\cyrr\cyru\cyrd\cyrn\cyrya%
\else \PackageError{datetime}{Invalid Month number \the\@orgargctr}{%
Month numbers should go from 1 to 12}%
\fi}
%    \end{macrocode}
% (Re)Define declaration to switch to this format.
%    \begin{macrocode}
\DeclareRobustCommand*\dateukraineb{%
\renewcommand{\formatdate}[3]{%
\@day=##1\relax\@month=##2\relax\@year=##3\relax
\number\@day~\monthnameukraineb[\@month]\space\number\@year~\cyrr.}}
%    \end{macrocode}
%\iffalse
%    \begin{macrocode}
%</dt-ukraineb.def>
%    \end{macrocode}
%\fi
%\iffalse
%    \begin{macrocode}
%<*dt-USenglish.def>
%    \end{macrocode}
%\fi
%\subsubsection{USenglish}
% Identify file
%    \begin{macrocode}
\ProvidesFile{dt-USenglish.def}[2007/06/11]
%    \end{macrocode}
\let\dateUSenglish\usdate

\let\monthnameUSenglish\monthnameenglish
\let\shortmonthnameUSenglish\shortmonthnameenglish

\let\dayofweeknameidUSenglish\dayofweeknameidenglish
\let\shortdayofweeknameidUSenglish\shortdayofweeknameidenglish
%\iffalse
%    \begin{macrocode}
%</dt-USenglish.def>
%    \end{macrocode}
%\fi
%\iffalse
%    \begin{macrocode}
%<*dt-usorbian.def>
%    \end{macrocode}
%\fi
%\subsubsection{USorbian}
% Identify file
%    \begin{macrocode}
\ProvidesFile{dt-usorbian.def}[2004/10/31]
%    \end{macrocode}
% Define (new) month names.
%    \begin{macrocode}
\providecommand{\monthnamenewusorbian}[1][\month]{%
\@orgargctr=#1\relax
\ifcase\@orgargctr
\PackageError{datetime}{Invalid Month number \the\@orgargctr}{%
Month numbers should go from 1 to 12}%
\or januara%
\or februara%
\or m\v erca%
\or apryla%
\or meje%
\or junija%
\or julija%
\or awgusta%
\or septembra%
\or oktobra%
\or nowembra%
\or decembra%
\else \PackageError{datetime}{Invalid Month number \the\@orgargctr}{%
Month numbers should go from 1 to 12}%
\fi}
%    \end{macrocode}
% Define (old) month names.
%    \begin{macrocode}
\providecommand{\monthnameoldusorbian}[1][\month]{%
\@orgargctr=#1\relax
\ifcase\@orgargctr
\PackageError{datetime}{Invalid Month number \the\@orgargctr}{%
Month numbers should go from 1 to 12}%
\or wulkeho r\'o\v zka%
\or ma\l eho r\'o\v zka%
\or nal\v etnika%
\or jutrownika%
\or r\'o\v zownika%
\or  sma\v znika%
\or pra\v znika%
\or \v znjenca%
\or po\v znjenca%
\or winowca%
\or nazymnika%
\or hodownika%
\else \PackageError{datetime}{Invalid Month number \the\@orgargctr}{Month numbers should go from 1 to 12}%
\fi}
%    \end{macrocode}
% Set up default
%    \begin{macrocode}
\let\monthnameusorbian=\monthnamenewusorbian
%    \end{macrocode}
% (Re)Define declaration to switch to (new) format.
%    \begin{macrocode}
\DeclareRobustCommand*\newdateusorbian{%
\renewcommand{\formatdate}[3]{%
\@day=##1\relax\@month=##2\relax\@year=##3\relax
\number\@day.~\monthnamenewusorbian[\@month]\space\number\@year}}
%    \end{macrocode}
% (Re)Define declaration to switch to (old) format.
%    \begin{macrocode}
\DeclareRobustCommand*\olddateusorbian{%
\renewcommand{\formatdate}[3]{%
\@day=##1\relax\@month=##2\relax\@year=##3\relax
\number\@day.~\monthnameoldusorbian[\@month]\space\number\@year}}
%    \end{macrocode}
% Set up default
%    \begin{macrocode}
\let\dateusorbian\newdateusorbian
%    \end{macrocode}
%\iffalse
%    \begin{macrocode}
%</dt-usorbian.def>
%    \end{macrocode}
%\fi
%\iffalse
%    \begin{macrocode}
%<*dt-welsh.def>
%    \end{macrocode}
%\fi
%\subsubsection{Welsh}
% Identify file
%    \begin{macrocode}
\ProvidesFile{dt-welsh.def}[2004/10/31]
%    \end{macrocode}
% Define month names.
%    \begin{macrocode}
\providecommand{\monthnamewelsh}[1][\month]{%
\@orgargctr=#1\relax
\ifcase\@orgargctr
\PackageError{datetime}{Invalid Month number \the\@orgargctr}{%
Month numbers should go from 1 to 12}%
\or Ionawr%
\or Chwefror%
\or Mawrth%
\or Ebrill%
\or Mai%
\or Mehefin%
\or Gorffennaf%
\or Awst%
\or Medi%
\or Hydref%
\or Tachwedd%
\or Rhagfyr%
\else \PackageError{datetime}{Invalid Month number \the\@orgargctr}{%
Month numbers should go from 1 to 12}%
\fi}
%    \end{macrocode}
% (Re)Define declaration to switch to this format.
%    \begin{macrocode}
\DeclareRobustCommand*\datewelsh{%
\renewcommand{\formatdate}[3]{%
\@day=##1\relax\@month=##2\relax\@year=##3\relax
\ifnum\@day=1\relax 1\/$^{\mathrm{a\tilde{n}}}$\else
\number\@day\fi \space a\space viz\space
\monthnamewelsh[\@month]\space\number\@year}}
%    \end{macrocode}
% \subsection{LaTeX2HTML Perl Script}
%\iffalse
%    \begin{macrocode}
%</dt-welsh.def>
%    \end{macrocode}
%\fi
%\Finale
\endinput
