% \iffalse 
% 
% register.dtx   
% Register diagrams with field descriptions
%
% Author: Matthew Lovell (lovell@indra.com)
% Copyright 2001 by Matthew B. Lovell 
%
% Run LaTeX on the file `register.ins' to get a .sty-file and 
% then on `register.dtx' to obtain instructions.
%
% This program is provided under the terms of the
% LaTeX Project Public License distributed from CTAN
% archives in directory macros/latex/base/lppl.txt.
%
% This program consists of register.dtx, register.ins,
% reg_reset.pl, and reg_table.pl.
%
%<*dtx>
\ProvidesFile{register.dtx}
%</dtx>
% \fi
% \CheckSum{589}
%
% \changes{v0.1}{1999/03/29}{First release}
% \changes{v0.2}{2000/01/29}{Added regdesc environment, for use under LyX}
% \changes{v0.21}{2000/03/12}{Fixed minute problem shown in regs with lots of 1-bit fields}
% \changes{v0.3}{2000/06/19}{Added reglabelb macro, for setting label without a drop}
% \changes{v0.31}{2000/07/11}{Modified names of lengths used within macros}
% \changes{v0.32}{2000/08/02}{Modified regdesc env to use regdescsep for vertical separation}
% \changes{v0.4}{2001/08/11}{Added macros for marginal Tape Release flags}
% \changes{v0.5}{2001/08/12}{Added package options for handling LyX, hyperref, and TR flags}
% \changes{v0.6}{2001/08/19}{Moved chapter redefinition into package; improved TR flags}
% \changes{v1.0}{2001/09/06}{Created .dtx documentation; first public release}
% \changes{v1.01}{2003/01/17}{Removed automatic label in register diagrams}
% \changes{v1.1}{2003/02/15}{Changed length manipulation to work with parskip package}
% \changes{v1.2}{2003/03/31}{Removed requirement to specify register offset}
% \changes{v1.3}{2003/11/24}{Changed how center is handled within register environment}
% \changes{v1.4}{2004/08/16}{Now provide a boolean set only in register context}
% \changes{v1.5}{2007/03/08}{Corrected use of conditionals used in spreading register reset value}
%
% \def\fileversion{v1.5}
% \def\filedate{2007/03/08}
% \title{Register diagrams with field descriptions}
% \author{Matthew Lovell\\
%  \texttt{lovell@indra.com}
% }
% \date{\filedate}
% \maketitle
%
% \providecommand{\LyX}{L\kern-.1667em\lower.25em\hbox{Y}\kern-.125emX\@}
%
% \begin{abstract}
% The |register| package is designed for typesetting
% the programmable elements in digital hardware, i.e., registers.
% Such registers typically have many fields and can be quite wide;
% they are thus a challenge to typeset in a consistent manner.
% This package attempts to address such issues.  It is similar
% in some aspects to the |bitfield| package by Reuben Thomas and the
% |bytefield| package by Scott Pakin.
% \end{abstract}
%
% \StopEventually
%
% \iffalse
%<*driver>
\documentclass{ltxdoc}
\RequirePackage{textcomp}
\RequirePackage[TRflags,TRboxed]{register}
\RequirePackage{amsmath}
\RequirePackage{amssymb}
\setlength{\regWidth}{0.98\textwidth}
\setlength{\TRwidth}{40pt}
% Uncomment the next two lines for a change history. 
% \AtBeginDocument{\RecordChanges}
% \AtEndDocument{\PrintChanges}
\begin{document}
\DocInput{register.dtx}
%% \RecordChanges
\end{document}
%</driver>
% \fi
%
% \listofregisters
%
% \section{Introduction} \label{sec:background}
%
% My group at work designs the memory and I/O controllers for HP's
% servers and workstations.  Historically, our chip documentation was
% done with FrameMaker or, more recently, Microsoft Word.  While these
% approaches have various disadvantages, one of the most egregious was
% register documentation.
%
% Our recent chips have had 64-bit wide control-status registers (CSR)
% or, more simply, registers.  Throw in the fact that many of these
% registers have a large number of single-bit fields, and you get a
% typesetting challenge.  The typical solution was to describe such
% registers using a table, typing field names vertically if space
% became a problem.  For a complicated register, these tables became
% quite complex and filled a large portion of a page.
%
% When we decided to evaluate \LaTeX{} for the latest round of
% documentation, we had three goals in mind with respect to registers:
% \begin{enumerate}
% \item Create a method of documenting registers which was consistent
%      and easy to read, regardless of the number of fields within a
%      register.
% \item Automate the creation of lists of registers, both in order
%      of appearance within the text and in memory order.
% \item Enable the automatic extraction of documented register reset
%      values in order to verify register functionality in the chip
%      itself.
% \end{enumerate}
% The |register| package is my attempt at meeting all three goals.  It
% has been in use by my group since April 1999; it may be not be
% pretty to all eyes, but it certainly has proven itself stable.
%
% In order to promote \LaTeX{} within our group at the time, we
% adopted \LyX{}.  The |register| package thus attempts to work well
% within that environment.  All \LyX{}-specific code, however, is
% controlled via package options.
%
% \subsection{Feedback}
%
% As |register| is my first \LaTeX{} package, I would welcome
% feedback regarding problems, new features, or more proper means of
% implementing existing features.  For example, is there an
% ``official'' method for determining the current font's maximum
% height and depth (see Section~\ref{sec:reg_fields} for details)?  I
% would also appreciate knowing whether anyone ever actually uses it. 
%
% It might also be worthwhile to investigate combining |register| with
% |bitfield| or |bytefield|.
%
% \section{Examples}

% Register~\ref{Diagnostic Control} depicts a register with many
% single-bit fields.  Registers such as this stress any scheme of
% typesetting, since the field names are so much longer than the space
% available for them within the diagram.  By typesetting the field
% names at an angle, |register| manages to be consistent regardless of
% the number or width of fields.  This rotation, however, means that
% documents which use |register| must be produced using PostScript.
%
% \begin{register}{H}{Diagnostic Control}{0x1F28}^^A  NAME=DIAG_CTRL1
% \label{Diagnostic Control}^^A
% \regfield{reserved}{1}{63}{0}^^A 
% \regfield{cmp\_ack64}{1}{62}{0}^^A 
% \regfield{cmp\_req64}{1}{61}{0}^^A 
% \regfield{cmp\_devsel}{1}{60}{0}^^A 
% \regfield{cmp\_stop}{1}{59}{0}^^A 
% \regfield{cmp\_trdy}{1}{58}{0}^^A 
% \regfield{cmp\_irdy}{1}{57}{0}^^A 
% \regfield{cmp\_frame}{1}{56}{0}^^A 
% \regfield{reserved}{1}{55}{0}^^A 
% \regfield{mask\_ack64}{1}{54}{0}^^A 
% \regfield{mask\_req64}{1}{53}{0}^^A 
% \regfield{mask\_devsel}{1}{52}{0}^^A 
% \regfield{mask\_stop}{1}{51}{0}^^A 
% \regfield{mask\_trdy}{1}{50}{0}^^A 
% \regfield{mask\_irdy}{1}{49}{0}^^A 
% \regfield{mask\_frame}{1}{48}{0}^^A 
% \regfield{cmp\_cbe}{8}{40}{0}^^A 
% \regfield{mask\_cbe}{8}{32}{0}^^A 
% \reglabel{Reset}^^A 
% \regnewline^^A 
% \regfield{reserved}{1}{31}{0}^^A 
% \regfield{cmp\_arb}{7}{24}{0}^^A 
% \regfield{reserved}{1}{23}{0}^^A 
% \regfield{mask\_arb}{7}{16}{0}^^A 
% \regfield{reserved}{5}{11}{0}^^A 
% \regfield{grp\_en\_cntl}{1}{10}{0}^^A 
% \regfield{grp\_en\_cbe}{1}{9}{0}^^A 
% \regfield{grp\_en\_arb}{1}{8}{0}^^A 
% \regfield{grp\_en\_fifo}{1}{7}{0}^^A 
% \regfield{dly\_frame}{1}{6}{0}^^A 
% \regfield{level\_frame}{1}{5}{0}^^A 
% \regfield{level\_cntl}{1}{4}{0}^^A 
% \regfield{level\_cbe}{1}{3}{0}^^A 
% \regfield{level\_arb}{1}{2}{0}^^A 
% \regfield{hlt\_trg\_on\_1}{1}{1}{0}^^A 
% \regfield{alwys\_cnt\_1}{1}{0}{0}^^A 
% \reglabel{Reset}^^A
% \regnewline^^A
% \end{register}
%
% Register~\ref{Function Class} is an example of a register one could
% derive from the PCI specification.  It is a miscellaneous register
% in a PCI card which controls device independent functions.  Note the
% field descriptions which, in this case, are actually part of the
% float.
%
% \begin{register}{htbp}{Function Class}{0x0008}^^A NAME=FC
% \label{Function Class}^^A
% \regfield{BIST}{8}{56}{0}^^A  READ_ONLY
% \regfield{Header Type}{8}{48}{0}^^A  READ_ONLY
% \regfield{Latency Timer}{8}{40}{0}^^A 
% \regfield{Line Size}{8}{32}{0}^^A 
% \reglabel{Reset}^^A 
% \regnewline^^A 
% \regfield{Class Code}{24}{8}{{0x060000}}^^A  READ_ONLY
% \regfield{Revision}{8}{0}{{0x10}}^^A  READ_ONLY
% \reglabel{Reset}^^A 
% \regnewline^^A
% \begin{regdesc}^^A
% 
% \begin{reglist}[Latency Timer]
% \item [BIST](Read only) Always returns 0.
% \item [Header~Type](Read only) Always returns 0.
% \item [Latency~Timer]PCI Latency Timer value (PCI 2.2 spec, Section 6.2.4).
% \item [Line~Size]PCI Cache Line Size (PCI 2.2 spec, Section 6.2.4). Valid
% values are listed below; any other value will be treated as indicating
% 64 byte cache lines.
% \begin{description}
% \item [0010\_0000]128 bytes 
% \item [0001\_0000]64 bytes 
% \end{description}
% \item [Class~Code](Read only) PCI Class Code (PCI 2.2 spec, Section 6.2.1).
% Chip identifies itself as a Host bridge.
% \item [Revision](Read only) Chip revision number. Bits 4--7 provide the major
% revision number, and bits 0--3 provide the minor
% revision number.
% \end{reglist}
% \end{regdesc}
% \end{register}
%
% A final example is shown in Figure~\ref{fig:regbus_cycle}.  This
% example shows that the register drawing macros don't have to be used
% solely within a |register| float.
%
% \begin{figure}[H]
% \begin{center}^^A 
%  \regfieldb{write\_en}{1}{15}^^A 
%  \regfieldb{reserved}{2}{13}^^A 
%  \regfieldb{function}{1}{12}^^A 
%  \regfieldb{block}{4}{8}^^A 
%  \regfieldb{register}{5}{3}^^A 
%  \regfieldb{reserved}{3}{0}^^A 
%  \reglabelb{Address phase} \\
%  \regfieldb{reserved}{8}{8}^^A 
%  \regfieldb{BE}{8}{0}^^A 
%  \reglabelb{BE phase}^^A 
% \end{center}^^A 
% \caption{Address and BE phases for register access}\label{fig:regbus_cycle}
% \end{figure}
%
%
% \section{User Interface} \label{sec:user}
%
% The user interface for register diagrams could be regarded as 
% overly verbose.  The main idea behind the parameters was to allow
% \LaTeX{} \emph{and} accompanying Perl scripts to decipher register
% fields and reset values.  The simplest way to show the interface
% is via another example.  Below is a 64-bit register.
% \begin{register}{H}{Example}{0x250}^^A name=example
% \label{example}^^A
% \regfield{FIFO depth}{6}{58}{{random}}^^A 
% \regfield{Something}{4}{54}{1100}^^A 
% \regfield{Status}{21}{33}{{uninitialized}}^^A 
% \regfield{Enable}{1}{32}{1}^^A 
% \reglabel{Reset}\regnewline^^A 
% \regfield{Counter}{10}{22}{{0x244}}^^A READ_ONLY 
% \regfield{Howdy}{5}{17}{1_1010}^^A 
% \regfield{Control}{1}{16}{-}^^A 
% \regfield{Hardfail}{1}{15}{1}^^A 
% \regfield{Data}{15}{0}{{uninitialized}}^^A 
% \reglabel{Reset}\regnewline^^A 
% \end{register}
%
% It was created with the following commands:
%
% \begin{verbatim}
% \begin{register}{H}{Example}{0x250}% name=example
%   \label{example}%
%   \regfield{FIFO depth}{6}{58}{{random}}%
%   \regfield{Something}{4}{54}{1100}%
%   \regfield{Status}{21}{33}{{uninitialized}}%
%   \regfield{Enable}{1}{32}{1}%
%   \reglabel{Reset}\regnewline%
%   \regfield{Counter}{10}{22}{{0x244}}% READ_ONLY
%   \regfield{Howdy}{5}{17}{1_1010}%
%   \regfield{Control}{1}{16}{-}%
%   \regfield{Hardfail}{1}{15}{1}%
%   \regfield{Data}{15}{0}{{uninitialized}}%
%   \reglabel{Reset}%\regnewline%
%  \end{register}
% \end{verbatim}
%
% \begin{macro}{register}
% \begin{macro}{\listofregisters}
% The register environment begins with |\begin{register}|.  The three
% parameters to the environment are the float specification (|h|, |H|,
% |t|, |b|, or |p|); the register name; and the register's offset in
% memory space.  These parameters become the register caption as well
% as the entry in the list of registers, which is typeset using
% |\listofregisters|.
% \end{macro}
% \end{macro}
% \hspace*{\MacroIndent}|\begin{register}{|\meta{placement}|}{|^^A
%  \meta{register name}|}{|\meta{register offset}|}|
%
% \begin{macro}{\regfield}
% Each register field is then typeset using |\regfield|.  The fields
% must be specified\footnote{Provided you're a little-endian person!}
% starting with the most-significant bit.  The |\regfield| macro takes
% four parameters: field name, field length (in bits), starting bit
% number, and reset value.  The reset value, unless you pass it as a
% single token (see example), will be spread apart to fill up the
% width of the field.  The reset value, since it could be just a
% very long binary number, can be interrupted with underscores to
% improve readability.
% \end{macro} 
% \hspace*{\MacroIndent}|\regfield{|\meta{name}|}{|\meta{length}|}{|^^A
%   \meta{start bit}|}{|\meta{reset value}|}|
%
% \begin{macro}{\regBitWidth}
% \begin{macro}{\regnewline}
% \begin{macro}{\reglabel}
% The macro |\regBitWidth| specifies the maximum number of bits to be
% typeset on a single line.  Once a line has been completed, a new line
% can be inserted via |\regnewline|.  Before starting a new line,
% however, an appropriate label for the reset value should be typed.
% This label is generated via |\reglabel|, which takes the string to
% typeset as a single parameter.  The package, via the macro
% |\regResetName|, assumes that |Reset| will be used in order to set
% some lengths appropriately; it really only matters for very full
% lines.  In a future version, it may be possible to make
% |\regResetName| be the default parameter to |\reglabel|.
% \end{macro}
% \end{macro}
% \end{macro}
% \hspace*{\MacroIndent}|\reglabel{|\meta{reset label}|}|\\[2pt]
% 
%
% As a final example, consider Register~\ref{Configuration}, which has
% field descriptions as part of the float.  This diagram was created 
% with the following code:
% \begin{verbatim}
% \begin{register}{htbp}{Configuration}{0x2848}% name=CONFIG
%   \label{Configuration}%
%   \regfield{soft reset perf}{1}{63}{0}% STATUS
%   \regfield{reserved}{30}{33}{0}%
%   \regfield{Test mode}{1}{32}{0}%
%   \reglabel{Reset}\regnewline%
%   \regfield{reserved}{13}{19}{0}%
%   \regfield{\parbox[b]{0pt}{Request Depth}}{7}{12}{1}%
%   \regfield{reserved}{3}{9}{0}%
%   \regfield{line\_2x\_L}{1}{8}{?}% STATUS
%   \regfield{reserved}{1}{7}{0}%
%   \regfield{ill\_cmd\_enable}{1}{6}{0}%
%   \regfield{LPCE}{1}{5}{0}%
%   \regfield{DVI disable}{1}{4}{0}%
%   \regfield{SBA enable}{1}{3}{0}%
%   \regfield{reserved}{2}{1}{0}%
%   \regfield{line\_2x\_enable}{1}{0}{0}% READ_ONLY
%   \reglabel{Reset}\regnewline%
%   \begin{regdesc}\begin{reglist}[Request~Depth]
%    \item [line\_2x\_enable]Setting this bit enables the chip 
%       to utilize a second connected data line.
%    \item [SBA~enable]Setting this bit activates the sideband-address port.
%       The SBA port is only useable in a double-line configuration.
%    \item [DVI~disable]Setting this bit \emph{turns off} DVI extraction for
%       DMA requests.
%    \item [LPCE]Line Parity Check Enable.
%    \item [ill\_cmd\_enable]\TR{3}Illegal Command enable.
%       This bit is new for TR3.
%    \item [line\_2x\_L](Read only) Indicates whether this chip is connected
%       to a second data line. When this bit is 0, a second line 
%       is available.
%    \item [Request~Depth]Controls number of outstanding DMA requests.
%    \item [Test~mode]Activates data line test mode.
%    \item [soft~reset~perf]\TR{4}Indicates that a soft reset has been
%       performed.  This bit is new for TR4.
% \end{reglist}\end{regdesc}\end{register}
% \end{verbatim}
%
%
% \begin{register}{htbp}{Configuration}{0x2848}^^A  NAME=ROPE_CONFIG
% \label{Configuration}^^A
% \regfield{soft reset perf}{1}{63}{0}^^A  STATUS
% \regfield{reserved}{30}{33}{0}^^A 
% \regfield{Test mode}{1}{32}{0}^^A 
% \reglabel{Reset}\regnewline^^A 
% \regfield{reserved}{13}{19}{0}^^A 
% \regfield{\parbox[b]{0pt}{Request Depth}}{7}{12}{1}^^A 
% \regfield{reserved}{3}{9}{0}^^A 
% \regfield{line\_2x\_L}{1}{8}{?}^^A  STATUS
% \regfield{reserved}{1}{7}{0}^^A 
% \regfield{ill\_cmd\_enable}{1}{6}{0}^^A 
% \regfield{PCE}{1}{5}{0}^^A 
% \regfield{DVI disable}{1}{4}{0}^^A 
% \regfield{SBA enable}{1}{3}{0}^^A 
% \regfield{reserved}{2}{1}{0}^^A 
% \regfield{line\_2x\_enable}{1}{0}{0}^^A  READ_ONLY
% \reglabel{Reset}\regnewline^^A 
% 
% \begin{regdesc}\begin{reglist}[Request~Depth]
% \item [line\_2x\_enable]Setting this bit enables this chip to utilize a second
% connected data line.
% \item [SBA~enable]Setting this bit activates the sideband-address port.
%   The SBA port is only useable in a double-line configuration.
% \item [DVI~disable]Setting this bit \emph{turns off} DVI extraction for
% DMA requests.
% \item [LPCE]Line Parity Check Enable.
% \item [ill\_cmd\_enable]\TR{3}Illegal OCMD enable.
% This bit is new for TR3.
% \item [line\_2x\_L](Read only) Indicates whether this chip is connected
% to a second data line. When this bit is 0, a second line is available.
% \item [Request~Depth]Controls number of outstanding DMA requests.
% \item [Test~mode]Activates data line test mode.
% \item [soft~reset~perf]\TR{4}Indicates that a soft reset has been
%       performed.  This bit is new for TR4.
% \end{reglist}\end{regdesc}\end{register}
%
% \begin{macro}{regdesc}
% \begin{macro}{reglist}
% \begin{macro}{lyxlist}

% This example uses the |regdesc| environment, which can either be
% part of a register float or used standalone.  The purpose of
% |regdesc| is really just to turn on centering and, when the |LyX|
% package option is specified, redefine the |lyxlist| environment.  If
% one is using \LyX{}, then register field descriptions can be entered
% using its List type.  If one is typing \LaTeX{} directly, just use
% the |reglist| list definition. 
% \end{macro}
% \end{macro}
% \end{macro}
%
% The |regdesc| and |reglist| environments both take optional
% arguments.  For |regdesc|, the optional argument specifies how wide
% the register description body should be.  The default value is 90\%
% of |\textwidth|.  For |reglist|, the optional argument is the
% longest field name, which allows the list to be spaced appropriately
% (like examples given in \textit{The \LaTeX{} Companion}).
%
% \begin{macro}{\regfieldb}
% Finally, if a register with no reset values is desired, one can use
% the |\regfieldb| macro.  It takes the same first three arguments as
% |\regfield|.  Figure~\ref{fig:regbus_cycle} was produced using |\regfieldb|.
% \end{macro}
%
%
%
% \subsection{TR Flags}
%
% \begin{macro}{\TR}
% The last example also shows the TR flags.  These marginal notes can
% be used to denote features that are specific to a particular release
% of a chip.  These macros were created since \LaTeX{}'s |\marginpar|
% is itself a float; as such it cannot be nested inside other floats.
% The |\TR| macros thus implement something very similar to the
% solution for Problem~14.28 in \textit{The \TeX{}book}.  

% TR flags are turned on via the |TRflags| package option.  If the
% option is not specified, then the |\TR| macro still exists but is
% empty.  The TR flags are placed inside an |\fbox| if the |TRboxed|
% package option is specified.
%
% For two-sided documents, the TR flags are placed in the outer
% margins.  The placement decision is made by placing labels for each
% TR flag and then determining even/odd page numbering in a subsequent
% \LaTeX{} run.
% \end{macro}
%
% 
% \section{Helper Scripts} \label{sec:scripts}
%
% A Perl module and a perl script are included as part of the
% |register| package.  The module augments the package by
% providing the ability to parse the register macro information into
% Perl.  This information consists of the parameters used within
% the macros themselves, as well as the comments which follow.  If
% you examine the last register example, you will notice that some
% fields specify |READ\_ONLY| or |STATUS| in a comment.  The
% ``directives'' can be used to indicate that particular fields are
% unaffected by writes or unaffected by reset, respectively.
%
% The script, |reg_list.pl|, utilizes the Perl module to produce lists
% of registers in offset-order or to generate reset value and mask
% information for each register.  The later data can be used in
% simulation to ensure that registers behave as documented.
%
%
% ^^A  -------------------------------------------------------------------------
% \newpage
%
% \section{Implementation} \label{sec:docstrip}
%
% \subsection{Identification \& Options} \label{sec:id}
%
%    The |register| document class can only be used with \LaTeXe, so we make
%    sure that an appropriate message is displayed when another \TeX{}
%    format is used.
%    \begin{macrocode}
%<*package>\NeedsTeXFormat{LaTeX2e}
%    \end{macrocode}
%
%    Announce the name and load required packages:
%    \begin{macrocode}
\ProvidesPackage{register} [2007/03/08 v1.5 Register macros with
hyperref/LyX support]

\RequirePackage{ifthen}[1997/11/02]
\RequirePackage{graphicx}[1997/06/09]
\RequirePackage{float}[2001/07/25]
\RequirePackage{calc}[1998/06/07]
%    \end{macrocode}
%
%    Declare the package options and some booleans.  There are options indicating
%    whether |register| is being used with \LyX{} or |hyperref|.  The marginal
%    TR flags can also be turned on or off with an option.
%    \begin{macrocode}
\DeclareOption{LyX}{\setboolean{RegisterLyX}{true}}
\DeclareOption{hyperref}{\setboolean{RegisterHyperref}{true}}
\DeclareOption{TRflags}{\setboolean{RegisterTRFlags}{true}}
\DeclareOption{TRboxed}{\setboolean{RegisterTRBoxed}{true}}

\DeclareOption*{%  Emit a warning for other options
  \PackageWarning{register}{Unknown option '\CurrentOption'}%
}

\newboolean{RegisterLyX}
\newboolean{RegisterHyperref}
\newboolean{RegisterTRFlags}
\newboolean{RegisterTRBoxed}

\setboolean{RegisterLyX}{false}
\setboolean{RegisterHyperref}{false}
\setboolean{RegisterTRFlags}{false}
\setboolean{RegisterTRBoxed}{false}

\ProcessOptions\relax  % Process package options
%    \end{macrocode}
%
% Also, provide a flag indicating when one is inside a register
% context.  Some users may wish to define macros, for example,
% which make use of |\marginpar|.  Those cannot be used inside
% of a floating register.  So, this flag is provided to
% provide something to test against.
%    \begin{macrocode}
\newboolean{RegisterContext}
\setboolean{RegisterContext}{false}
%    \end{macrocode}
%
% \subsection{Float Declaration \& Lengths} \label{sec:float}
%
%    Declare the new register float type and the lengths which the
%    macros will use.  The register diagrams tend to look best with
%    the caption at the top of the float.  If |\chapter| is defined in
%    the current document class, then the register diagrams will count
%    by chapter, otherwise they will count by section.  
%
%    \begin{macrocode}
\floatstyle{plaintop} \@ifundefined{chapter}
{\newfloat{Regfloat}{tbp}{rdf}[section]}
{\newfloat{Regfloat}{tbp}{rdf}[chapter]}
\floatname{Regfloat}{Register}
%    \end{macrocode}
%
% \subsection{Style Parameters} \label{sec:style}
%    
% \begin{macro}{\regWidth}
% \begin{macro}{\regBitWidth}
%    The |\regWidth| length controls how physically wide the register
%    diagrams are, while |\regBitWidth| specifies how the maximum
%    number of bits per line in the diagram.  The default value
%    for |\regWidth| is 95\% of the width of the main text body. 
% \end{macro}
% \end{macro}
%
% \begin{macro}{\regResetName}
% \begin{macro}{\regBitSize}
% \begin{macro}{\regBitFamily}
% \begin{macro}{\regResetSize}
% \begin{macro}{\regLabelSize}
% \begin{macro}{\regLabelFamily}
% \begin{macro}{\regDescFamily}
%    The string with which to label reset values is
%    |\regResetName|. The |\regBitSize| command controls the font size
%    used for bit values, |\regResetSize| controls the font size used
%    for reset values, and |\regBitFamily| controls which font family
%    is used to denote bit positions.  The font size for field names
%    and for the |\regResetName| is controlled by |\regLabelSize|.
%    Finally, |\regDescFamily| controls the appearance of field names
%    within register descriptions.
% \end{macro}
% \end{macro}
% \end{macro}
% \end{macro}
% \end{macro}
% \end{macro}
% \end{macro}
%
%    \begin{macrocode}
\newlength{\regWidth}
\newlength{\regFieldLen}
\newlength{\regLabelAdjust}
\newlength{\regResetHeight}
\newlength{\regResetDepth}
\newlength{\regResetDrop}
\newlength{\regDescSkip} 
\setlength{\regWidth}{0.95\textwidth}
\newcommand{\regBitWidth}{32}
\newcommand{\regResetName}{Reset}
\newcommand{\regBitSize}{\tiny}
\newcommand{\regBitFamily}{\sffamily}
\newcommand{\regResetSize}{\scriptsize}
\newcommand{\regLabelSize}{\footnotesize}
\newcommand{\regLabelFamily}{\rmfamily}
\newcommand{\regDescFamily}{\bf}
%    \end{macrocode}
%    The lengths below control the vertical spacing between a 
%    register diagrams and the register field descriptions.  This
%    distance must change when the |\regdesc| is part of the |\Regfloat|
%    or standalone.
%    \begin{macrocode}
\newlength{\regdescsep}   
\newlength{\oldregdescsep}
\setlength{\regdescsep}{-\medskipamount}

\newsavebox{\Label}
\newsavebox{\RotatedLabel}
\newcounter{upperbit}
\newcounter{lowerbit}
%    \end{macrocode}
%
% \subsection{Register macros} \label{sec:regdesc}
%
% \begin{macro}{\reglist}
%    Define a new environment for register field descriptions,
%    beneath a register diagram.  These descriptions can indicate what
%    a particular field is used for, whether it is read-only, etc.  The
%    field descriptions themselves are in a |\reglist| list environment,
%    unless \LyX{} is being used.
% \end{macro}
%    \begin{macrocode}
\newenvironment{reglist}[1][M]
  {\begin{list}{}
    {\settowidth{\labelwidth}{\regDescFamily #1}
     \addtolength{\labelwidth}{\labelsep}
     \setlength{\leftmargin}{\labelwidth}
     \addtolength{\leftmargin}{\labelsep}
     \addtolength{\leftmargin}{0.5\regDescSkip}
     \addtolength{\rightmargin}{0.5\regDescSkip}
     \setlength{\topsep}{0pt}
     \setlength{\itemsep}{0pt}
     \setlength{\parsep}{0.5\baselineskip}
     \renewcommand{\makelabel}[1]{\regDescFamily ##1 \hfill}}}
  {\end{list}}  
%    \end{macrocode}
%
%
% \begin{macro}{\regdesc}
% Next is the |\regdesc| environment.  If \LyX{} is being used,
% the environment temporarily redefines the |lyxlist| environment, which
% is what \LyX{} provides as the default list structure.  Otherwise,
% the user is left to use |\reglist|.  The redefinition should look very
% similar to that for |\reglist|.
%
% In all cases, |\regdesc| starts by attempting to place a small
% amount of vertical space between it and the presumed register
% diagram immediately above it.
% \end{macro}
%    \begin{macrocode}
\newenvironment{regdesc}[1][0.90\textwidth]%
% ---------------------------------------------
{%
 \setlength{\regDescSkip}{\textwidth - #1}%
 \vspace{\regdescsep}%
\ifthenelse{\boolean{RegisterLyX}}{%
  \renewenvironment{lyxlist}[1]
    {\begin{list}{}
      {\settowidth{\labelwidth}{\regDescFamily ##1}
       \addtolength{\labelwidth}{\labelsep}
       \setlength{\leftmargin}{\labelwidth}
       \addtolength{\leftmargin}{\labelsep}
       \addtolength{\leftmargin}{0.5\regDescSkip}
       \addtolength{\rightmargin}{0.5\regDescSkip}
       \setlength{\topsep}{0pt}
       %\setlength{\partopsep}{0pt}
       \renewcommand{\makelabel}[1]{\regDescFamily ####1 \hfill}}}
    {\end{list}}}{}% 
 % endif
 % set spacing appropriately
 \leftskip 0.5\regDescSkip%
 \rightskip 0.5\regDescSkip%
 \parfillskip=\z@ plus 1fil%
 \parskip=0.5\baselineskip \advance\parskip by 0pt plus 2pt%
}% end begin{regdesc}
% --------------------------------------------
{\vskip\baselineskip}
%    \end{macrocode}
%
% \begin{macro}{\regnewline}
% This command allows a register diagram to be broken into multiple lines
% \end{macro}
%    \begin{macrocode}
\newcommand{\regnewline}{\\*}
%    \end{macrocode}
%
%  
% \begin{macro}{\register}
%     This environment declares the floating environment in which a
%     register diagram and, optionally, its field descriptions can be
%     typeset.  It takes as arguments the float type (h, t, p, H); the 
%     register name; and the register offset/address.
% \end{macro}
%    \begin{macrocode}
\newenvironment{register}[3]
{\begin{Regfloat}[#1]%
  \setlength{\leftskip}{0pt}%
  \setlength{\oldregdescsep}{\regdescsep}%
  \setlength{\regdescsep}{0.5\baselineskip}%
  \setlength{\partopsep}{0pt}%
  \setlength{\topsep}{0pt}%
  \setboolean{RegisterContext}{true}%
  \ifthenelse{\equal{#3}{}}%
  {\caption[#2]{\textsc{#2}}}% else
  {\caption[#2]{\textsc{#2} ({#3})}}%
  \centering}
{% restore lengths
  \leftskip\z@%
  \rightskip\z@%
  \parfillskip=\z@ plus 1fil%
  \setlength{\regdescsep}{\oldregdescsep}%
  \setboolean{RegisterContext}{false}%
  \end{Regfloat}}
%    \end{macrocode}
%
%
% \begin{macro}{\regUnderScore}
% \begin{macro}{\regFiller}
% \begin{macro}{\regSpread}
%    The next definitions provide a method of spreading out
%    an argument, i.e., placing |\hfill|'s between each character.
%    They can be used to space the reset value of a register field
%    appropriately.  Previous versions of these macros used \TeX{}'s
%    conditionals incorrectly.  
%
%    If you are using the |underscore| package, you should
%    \emph{repeat} the definition of |\regUnderScore|, after loading
%    |underscore| in order to observe the changes made by that package. 
%    The spreading macros below strive to strip out underscores.
% \end{macro}
% \end{macro}
% \end{macro}
%    \begin{macrocode}
\def\regUnderScore{_}%
\def\regFiller#1{\def\regInner{#1}%
\ifx\regInner\regUnderScore%
\else%
\ifnum\count0>0%
\hfill#1%
\else#1\fi%
\fi%
\advance\count0 by 1%
}
\def\regSpread#1{\count0=0{}\regSpreadaux#1\empty}
\def\regSpreadaux#1#2\empty{\def\aux{#1}%\show#1%
\ifx\aux\empty%
\else%
\def\aux{#2}%
\regFiller{#1}%
\ifx\aux\empty%
\else%
\regSpreadaux#2\empty%
\fi\fi}
%    \end{macrocode}
%
%
% \subsubsection{Register fields} \label{sec:reg_fields}
%
% Define some internal utility macros which get called repeatedly.
% These macros figure out some of the dimensions of the current font
% (is there a better way to do this?), construct and rotate label
% boxes, and typeset bit positions.
%    \begin{macrocode}
\newcommand{\setRegLengths}{%
  % Compute basic width of a single bit
  \settowidth{\regFieldLen}{\regLabelSize \regResetName}%
  \setlength{\regFieldLen}{\regWidth - \regFieldLen}%
  \setlength{\regFieldLen}{\regFieldLen / \regBitWidth}%
  % Figure out height and depth of reset field in current font
  % Is there a more ``official'' method to do this?
  \settodepth{\regResetDepth}{\regResetSize ()jgpq}%
  \settoheight{\regResetHeight}{\regResetSize ()ABCjkl}%
  \addtolength{\regResetHeight}{\regResetDepth}%
  % Compute how far to drop the reset fields down.  The value at
  % the end is effectively the separation between the bit position
  % box and the reset value box.
  \setlength{\regResetDrop}{\regResetHeight + 2\fboxsep - 2\fboxrule + 3pt}%
}
%    \end{macrocode}
%    These macros assembly, rotate, and typset the register field name.
%    It is the use of |\rotatebox| below which makes |register| require
%    PostScript processing.
%    \begin{macrocode}
\newcommand{\regMakeFieldName}[1]{%
  % Create box to hold label
  \savebox{\Label}{\regLabelSize\regLabelFamily #1}%
}
\newcommand{\regRotateFieldName}{%
  \savebox{\RotatedLabel}{\rotatebox[origin=lB]{45}{\usebox{\Label}}}%
  \makebox[0pt][l]{\raisebox{\regResetHeight + \fboxsep + \depth + 1pt}%
        {\hspace{\regLabelAdjust}\usebox{\RotatedLabel}}}%
}
%    \end{macrocode}
%    The |\typesetRegBits| macro constructs the frame for a particular
%    register field and sets the starting and ending bit positions
%    within that frame.
%    \begin{macrocode}
\newcommand{\typesetRegBits}[1]{%
  \ifthenelse{#1 > 1}%
    {\framebox[\regFieldLen][c]%
        {\regBitSize \rule[-1\regResetDepth]{0pt}{\regResetHeight}%
           \regBitFamily\arabic{upperbit} \hfill \arabic{lowerbit}}}%
    {\framebox[\regFieldLen][c]%
        {\regBitSize \rule[-1\regResetDepth]{0pt}{\regResetHeight}%
           \regBitFamily\arabic{lowerbit}}}%
}
%    \end{macrocode}
%    This macro constructs a frame, below the bit position frame,
%    displaying the reset (or some other) value for a register field.
%    \begin{macrocode}
\newcommand{\typesetRegReset}[1]{%
 % Typeset reset value in a framebox
  \makebox[0pt][l]{\raisebox{-1\regResetDrop}{\framebox[\regFieldLen][c]%
        % Place an invisible rule to control the box 
        % surrounding the reset field
        {\regResetSize \rule[-1\regResetDepth]{0pt}{\regResetHeight}\regSpread{#1}}}}%
}
%    \end{macrocode}
%
%
%
% \begin{macro}{\regfield}
% The macros below are the user interface to the register drawing
% routines.  The first of these, |\regfield| takes four arguments:
% the field name, field length (in bits), the starting bit number,
% and the field's reset value.
% \end{macro}
%    \begin{macrocode}
\newcommand{\regfield}[4]{%
  % Compute overall field length
  \setRegLengths%
  \setlength{\regFieldLen}{#2\regFieldLen + \fboxrule}%
  % Figure out bit positions
  \setcounter{lowerbit}{#3}%
  \setcounter{upperbit}{#3 + #2 - 1}%
  \regMakeFieldName{#1}%
  % Figure out how far over to place label, accounting for height
  \setlength{\regLabelAdjust}{0.5\regFieldLen - 0.707107\ht\Label}%
  % Now, rotate and type the label
  \regRotateFieldName%
  \typesetRegReset{#4}%
  % Typeset bit positions in a framebox
  \typesetRegBits{#2}%
  \hspace{-1\fboxrule}%
}
%    \end{macrocode}
%
%
%
% \begin{macro}{\regfieldb}
%   The |\regfieldb| macro can be used to construct a register diagram
%   without reset values.  It's parameters are thus the same
%   as for |\regfield| with the omission of reset value.
% \end{macro}
%    \begin{macrocode}
\newcommand{\regfieldb}[3]{%
  % Compute overall field length
  \setRegLengths%
  \setlength{\regFieldLen}{#2\regFieldLen + \fboxrule}%
  % Figure out bit positions
  \setcounter{lowerbit}{#3}%
  \setcounter{upperbit}{#3 + #2 - 1}%
  % Create box to hold label
  \regMakeFieldName{#1}%
  % Figure out how far over to place label, accounting for height
  \setlength{\regLabelAdjust}{0.5\regFieldLen - 0.707107\ht\Label}%
  % Now, rotate and typeset the label
  \regRotateFieldName%
  % Typeset bit positions
  \typesetRegBits{#2}%
  \hspace{-1\fboxrule}%
}
%    \end{macrocode}
%
%
%
% \begin{macro}{\regbits}
%    The |\regbits| macro typesets a register field without
%    showing bit positions.  As such, it only takes three
%    parameters: field name, field bit length, and field value.
% \end{macro}
%    \begin{macrocode}
\newcommand{\regbits}[3]{%
  % Compute overall field length
  \setRegLengths%
  \setlength{\regFieldLen}{#2\regFieldLen + \fboxrule}%
  % Figure out bit positions
  \setcounter{lowerbit}{#3}%
  \setcounter{upperbit}{#3 + #2 - 1}%
  % Create box to hold label
  \regMakeFieldName{#1}%
  % Figure out how far over to place label, accounting for height
  \setlength{\regLabelAdjust}{0.5\regFieldLen - 0.707107\ht\Label}%
  % Now, rotate and typeset the label
  \regRotateFieldName%
  % Typeset field value
  \framebox[\regFieldLen][c]%
        {\tiny\regSpread{#3}}%
  \hspace{-1\fboxrule}%
}
%    \end{macrocode}
%
%
%
% \begin{macro}{\regspace}
% \begin{macro}{\reglabel}
% \begin{macro}{\reglabelb}
%    Finally, define some additional utility macros.  An invisible box
%    with the same width as a specified number of bits is typeset by
%    |\regspace|.  It takes as its sole parameter the desired
%    bitwidth.  The |\reglabel| macro typesets the label for the reset
%    field in register diagrams.  Finally, |\reglabelb| performs the
%    same function, but without any drop.
% \end{macro}
% \end{macro}
% \end{macro}
%    \begin{macrocode}
\newcommand{\regspace}[1]{%
  % Compute overall field length
  \setRegLengths%
  \setlength{\regFieldLen}{#1\regFieldLen + \fboxrule}%
  \makebox[\regFieldLen]{}%
}

\newcommand{\reglabel}[1]{%
  \settowidth{\regFieldLen}{\regLabelSize \regResetName}%
  \setlength{\regResetDrop}{\regResetDrop + 0.5\fboxsep}%
  $\,$\raisebox{-1\regResetDrop}{\makebox[\regFieldLen][l]%
    {\regLabelSize\regBitFamily #1}}%
}

\newcommand{\reglabelb}[1]{%
  \settowidth{\regFieldLen}{\regLabelSize \regResetName}%
  $\,$\raisebox{-0.5\fboxsep}{\makebox[\regFieldLen][l]%
    {\regLabelSize\regBitFamily #1}}%
}
%    \end{macrocode}
%
%
%
% \subsubsection{Hyperref Interface} \label{sec:hyperref}
%
% This code helps with the hyperlinks to register floats
% when producing linked PDF.
%    \begin{macrocode}
\ifthenelse{\boolean{RegisterHyperref}}{%
  % Define a counter for the hyperref package.  Otherwise,
  % the hyperlinks to registers don't work correctly
  \@namedef{theHRegfloat}{\theRegfloat}

  % Define a bookmark level for Regfloats (for hyperref package)
  \def\toclevel@Regfloat{0}
}{}
%    \end{macrocode}
%
%
%
%
% \subsubsection{List of Registers} \label{sec:list}
%
% The code below redefines what the |float| package specifies
% in order to provide a hook for changing the format of the
% ToC line for registers.
%    \begin{macrocode}
\newcommand{\listofregisters}{%
  \@ifundefined{ext@Regfloat}{\float@error{Regfloat}}{%
    \@ifundefined{chapter}{\def\@tempa{\section*}}%
      {\def\@tempa{\chapter*}}%
    \@tempa{List of Registers\@mkboth{\uppercase{List of Registers}}%
       {\uppercase{List of Registers}}}%
    \addcontentsline{toc}{chapter}{List of Registers}%
    \@starttoc{\@nameuse{ext@Regfloat}}}}
\newcommand\l@Regfloat{\@dottedtocline{1}{1.5em}{2.3em}}
%    \end{macrocode}
%
%
%
% The |\chapter| command, if present, is also redefined in order
% to get identical chapter-related spacing in the list of registers
% as appears in the LoF and LoT.
%    \begin{macrocode}
\@ifundefined{chapter}{}
{% Adjust chapter definition slightly for Regfloats
\def\@chapter[#1]#2{\ifnum \c@secnumdepth >\m@ne
  \if@mainmatter
    \refstepcounter{chapter}%
    \typeout{\@chapapp\space\thechapter.}%
    \addcontentsline{toc}{chapter}%
    {\protect\numberline{\thechapter}#1}%
  \else
    \addcontentsline{toc}{chapter}{#1}%
  \fi
  \else
    \addcontentsline{toc}{chapter}{#1}%
  \fi
  \chaptermark{#1}%
    \addtocontents{lof}{\protect\addvspace{10\p@}}%
    \addtocontents{lot}{\protect\addvspace{10\p@}}%
    \addtocontents{rdf}{\protect\addvspace{10\p@}}% --- Add space ---
  \if@twocolumn
    \@topnewpage[\@makechapterhead{#2}]%
  \else
    \@makechapterhead{#2}%
    \@afterheading
  \fi}
}
%    \end{macrocode}
%
%
%
% \subsection{TR flag macros} \label{sec:TRflags}%
%
% Define a mechanism for including TR flags in register descriptions,
% as well as possibly in the main body of the text.  One cannot just
% use |\marginpar|'s within a register description, since they are
% typically part of a register float.  \LaTeX{}'s |\marginpar| command is
% \emph{also} considered a float, so they cannot be nested.
%
% So, we resort to using some lower level \LaTeX{}.  This certainly
% isn't the most elegant way to get the flags, but I think it will
% work for the majority of cases.
%
% The |\TR{x}| command is intended to be placed at the start of
% a register field description, so that the alignment is 
% obvious on the page.  To also get decent alignment in
% main body text, it is necessary to have a second definition.
% So, we'll see these commands get redefined inside regdesc,
% below.
%
% The TR flags also decide which page they are on by placing
% labels and testing the resulting pagenumber.  This enables
% the flags to be placed in the outer margin in two-sided
% documents.
%
% \begin{macro}{\TRfamily}
% \begin{macro}{\TRwidth}
% The font family used for the TR flags can be controlled via
% |\TRfamily|.  The width available for TR flags is controlled
% by |\TRwidth|, which by default is set to |\marginparwidth|.
% Setting the width may be important if you are using the
% |TRboxed| package option.
% \end{macro}
% \end{macro}
%
%    \begin{macrocode}
% Define a pageref which will work with \isodd
\newcommand\@GetTRSecondParam{}
\long\def\@GetTRSecondParam#1#2#3\@nil{#2}
\newcommand*{\GetTRPageRef}[1]{%
  \expandafter\expandafter\expandafter\@GetTRSecondParam
    \csname r@#1\endcsname
    0% dummy, if the reference is undefined
    \@nil
}

\newcommand{\TRfamily}{\sffamily}
\newcounter{T@peReleaseTag}
\newlength{\TRwidth}
\setlength{\TRwidth}{\marginparwidth}
\newlength{\T@peReleaseDepth}

% Internal command to form the actual TR margin note.
\newcommand{\TRwriteout}[1]{%
  \makebox[\TRwidth][c]{%   
    \raisebox{\T@peReleaseDepth}{%
     \ifthenelse{\boolean{RegisterTRBoxed}}%
       {\fbox{\TRfamily #1}}%
       {\TRfamily #1}}}%
}

% Internal command to typeset a TR flag in the right margin
\newcommand{\TRrightlabel}[1]{%
  % Place a strut in order to set line depth
  \mbox{}\strut\settodepth{\T@peReleaseDepth}{\strut}%
   \vadjust{\hspace{\textwidth}\hspace{\marginparsep}%
   \smash{\rlap{\TRwriteout{#1}}}}}

% Internal command to typeset a TR flag in the left margin
\ifthenelse{\boolean{@twoside}}{
  % Two-sided document 
  \newcommand{\TRleftlabel}[1]{%
    % Place a strut in order to set line depth
    \mbox{}\strut\settodepth{\T@peReleaseDepth}{\strut}%
    \vadjust{\smash{\llap{\TRwriteout{#1}}\kern\marginparsep}}}
}{
  % Otherwise, the command is the same as rightlabel
  \newcommand{\TRleftlabel}[1]{%
    \TRrightlabel{#1}}
}
%    \end{macrocode}
%
%
% \begin{macro}{\TR}
%   Finally, the user-level command is simply |\TR|.
% \end{macro}
%    \begin{macrocode}
\ifthenelse{\boolean{RegisterTRFlags}}{
\newcommand{\TR}[1]{%
  \stepcounter{T@peReleaseTag}%
  \label{TapeReleaseTag-\theT@peReleaseTag}%
  \ifthenelse{\isodd{\GetTRPageRef{TapeReleaseTag-\theT@peReleaseTag}}}%
   {\TRrightlabel{TR\hspace{1pt}#1}}%
   {\TRleftlabel{TR\hspace{1pt}#1}}%
}}
{\newcommand{\TR}[1]{}}
%    \end{macrocode}
%
% 
%
% \Finale
%
\endinput

%% \CharacterTable
%%  {Upper-case    \A\B\C\D\E\F\G\H\I\J\K\L\M\N\O\P\Q\R\S\T\U\V\W\X\Y\Z
%%   Lower-case    \a\b\c\d\e\f\g\h\i\j\k\l\m\n\o\p\q\r\s\t\u\v\w\x\y\z
%%   Digits        \0\1\2\3\4\5\6\7\8\9
%%   Exclamation   \!     Double quote  \"     Hash (number) \#
%%   Dollar        \$     Percent       \%     Ampersand     \&
%%   Acute accent  \'     Left paren    \(     Right paren   \)
%%   Asterisk      \*     Plus          \+     Comma         \,
%%   Minus         \-     Point         \.     Solidus       \/
%%   Colon         \:     Semicolon     \;     Less than     \<
%%   Equals        \=     Greater than  \>     Question mark \?
%%   Commercial at \@     Left bracket  \[     Backslash     \\
%%   Right bracket \]     Circumflex    \^     Underscore    \_
%%   Grave accent  \`     Left brace    \{     Vertical bar  \|
%%   Right brace   \}     Tilde         \~}

%%% Local Variables:
%%% mode: LaTeX
%%% End:

